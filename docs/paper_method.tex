\documentclass[aps,showpacs,twocolumn,
prd,superscriptaddress,nofootinbib]{revtex4-1}

\usepackage{amsmath}
\usepackage{amsfonts}
\usepackage{amssymb}
\usepackage{latexsym}
\usepackage{graphicx}
\usepackage{bm}
\usepackage{color}
\usepackage{enumerate}
\usepackage{ulem}
\usepackage{tabularx}

\usepackage{graphicx}
\usepackage[caption=false]{subfig}
%\usepackage{caption}
%\usepackage{subcaption}
%\captionsetup{compatibility=false}

\usepackage{color}
\usepackage[usenames,dvipsnames,svgnames,table]{xcolor}
%\usepackage[colorlinks=true,
%            linkcolor=green,
%            urlcolor=blue,
%            citecolor=red]{hyperref}
\usepackage[colorlinks=true,
            linkcolor=YellowOrange,
            urlcolor=RoyalBlue,
            citecolor=RedViolet]{hyperref}
            
%\usepackage{showlabels}

\newcommand{\be}{\begin{equation}}
\newcommand{\ee}{\end{equation}}
\newcommand\ud{{\mathrm{d}}}
\newcommand\uD{{\mathrm{D}}}
\newcommand\calO{{\mathcal{O}}}
\newcommand\calM{{\mathcal{M}}}
\newcommand\calF{{\mathcal{F}}}
\newcommand\calT{{\mathcal{T}}}
\newcommand\calD{{\mathcal{D}}}
\newcommand\calA{{\mathcal{A}}}
\newcommand\bfx{\mathbf{x}}
\newcommand{\ov}[1]{\overline{#1}}
\newcommand{\ph}[1]{\phantom{#1}}
\newcommand{\cte}{\mathrm{cte}}
\newcommand{\nn}{\nonumber}
%\newcommand{\hatk}{\hat{k}}
\newcommand{\hatk}{k}
\newcommand{\Hz}{\,\mathrm{Hz}}
\newcommand{\yr}{\,\mathrm{yr}}
\newcommand{\sinc}{\,\mathrm{sinc}}
\newcommand{\Msol}{M_{\odot}}
\newcommand{\Mchirp}{M_{c}}
\newcommand{\tf}{t_{f}}
\newcommand{\Tf}{T_{f}}
\newcommand{\tfd}{t_{f}^{d}}
\newcommand{\tfSPA}{t_{f}^{\rm SPA}}
\newcommand{\TfSPA}{T_{f}^{\rm SPA}}

\newcolumntype{C}[1]{>{\centering\arraybackslash}p{#1}}
\newcolumntype{L}[1]{>{\raggedright\arraybackslash}p{#1}}

\newcommand{\SM}[1]{{\color{Red} #1}}
\newcommand{\jgb}[1]{{\color{DarkGreen} #1}}

\begin{document}

\title{Fourier-domain modulations and delays of gravitational-wave signals}

\author{Sylvain Marsat}
\affiliation{Max Planck Institute for Gravitational Physics (Albert Einstein Institute), Am M\"uhlenberg 1, Potsdam-Golm, 14476, Germany}
\author{John G. Baker}
\affiliation{Gravitational Astrophysics Laboratory, NASA Goddard Space Flight Center, 8800 Greenbelt Rd., Greenbelt, MD 20771, USA}
\affiliation{Joint Space-Science Institute, University of Maryland, College Park, MD 20742, USA}

\date{\today}

\begin{abstract}

We present a Fourier-domain based approach modulations and delays of gravitational wave signals, a problem which arises in two different contexts. For space-based detectors like LISA, the orbital motion of the detector introduces a time-dependency in the response of the detector, consisting of both a modulation and a varying delay. In the context of signals from precessing spinning binary systems, a useful tool for building models of the waveform consists in representing the signal as a time-dependent rotation of a quasi-non-precessing waveform. In both cases, being able to compute transfer functions for these effects directly in the Fourier domain may enable performance improvements for data analysis applications by using fast frequency-domain waveforms. Our results generalize previous approaches based on the stationary phase approximation, extending them by including delays and higher-order corrections while being applicable to a broader class of signals including full inspiral-merger-ringdown waveforms. In the LISA case, we find that higher-order corrections remain small except at the two extremes of the detector's frequency band, where we show that including them reduces the errors. In the case of precessing binaries, we perform a limited exploration of the merger-ringdown range based on a toy model for the precession. We find that, depending on the post-merger frame velocity, higher-order corrections can become quantitatively large to the point of challenging our perturbative formalism, while affecting mainly the amplitude and giving rise to limited unfaithfulness. We further present an alternative direct convolution approach to accurately represent these post-merger features.

\end{abstract}

\pacs{
04.25.D-, % numerical relativity
04.70.Bw, % classical black holes
04.80.Nn, % Gravitational wave detectors and experiments
95.30.Sf, % relativity and gravitation
95.55.Ym, % Gravitational radiation detectors
97.60.Lf  % black holes (astrophysics)
}

\maketitle

%%%%%%%%%%%%%%%%%%%%%%%%%%%%%%%%%%%%
%%%%%%%%%%%%%%%%%%%%%%%%%%%%%%%%%%%%

\section{Introduction}
\label{sec:intro}

\jgb{With the unprecedented recent gravitational-wave detections of coalescencing binary black holes and binary neutron stars, announced by the LIGO-Virgo collaboration~\cite{LIGO-theevent-2016,LIGO-christmasevent-2016,LIGO-O1BBH-2016}[update refs], gravitational-wave astronomy has entered its observational era. As LIGO prepares for even more sensitive observation runs, and with the recent expansion of the ground-based detectors network with Virgo~\cite{Virgo} ( and eventually also KAGRA~\cite{KAGRA} and LIGO-India~\cite{INDIGO}), observations of such compact object coalescences are expected at ever-increasing rate, with binary black holes likely to be the dominant sources~\cite{} \SM{[C]}.}

  \jgb{Moreover, the European Space Agency has recently selected the Laser Interferometer Space Antenna (LISA)\cite{LISA17} to realize the ``Gravitational Universe'' science theme\cite{GravitationalUniverse2013} as the 3rd large space mission of its Cosmic Vision program, with a tentative launch around 2034. LISA will be able to detect and characterize, among several important gravitational wave source targets, comparable-mass binary black hole coalescences from cosmological distances over a wide range of masses. These will range from high-redshift observations of supermassive black hole binaries with $M\sim 10^{7} \Msol$ down to the observation of LIGO-type sources with $M\sim 10^{2} \Msol$~\cite{Sesana16}.}

  Data analysis for gravitational-wave observations of compact binary coalescences require accurate models (or templates) for the signals, both for ensuring efficient detections of signals that may be buried in instrumental noise, and to extract the physical parameters of the source in a subsequent analysis. Bayesian analysis for parameter estimation of gravitational-wave signals, as was performed for the LIGO detections~\cite{LIGO-theeventPE-2016,LIGO-O1BBH-2016}, may require millions of evaluations of the likelihood function to sample the posterior probability distribution. \SM{[objective: accuracy, not necessary necessary for LIGO now but in the future yes]} The greater sensitivity of LISA and other future instruments will require further increases in the accuracy and computational efficiency of signal templates.

  %This motivates an important and ongoing effort to improve gravitational wave templates. Two important focus areas for template improvements are need to cover all phases of the evolution of the system, including the inspiral, merger and ringdown (IMR), which became possible thanks to the breakthrough of numerical relativity~\cite{Pretorius05,Baker+06,Campanelli+06}, and the inclusion of the effects of the spins of the compact objects, that can lead to a precession of the orbital plane~\cite{ACST94,K95}. Besides accuracy, the computational efficiency of template generation is crucial. 

\jgb{The GW community is making progress assembling higher-fidelity tools for these coming challenges.
State-of-the-art IMR templates combine information from the perturbative results of post-Newtonian theory covering the inspiral (see e.g.~\cite{BlanchetLiving}) and from numerical relativity simulations covering the end of the inspiral and the merger-ringdown phase (see e.g.~\cite{Pfeiffer12}).  Approaches to template construction include phenomenological templates postulating an analytic ansatz for the Fourier-domain amplitude and phase~\cite{Husa+15, Khan+15,Hannam+13}, and the more physically motived Effective-One-Body (EOB) approach~\cite{BD99,Taracchini+13, Pan+13, Bohe+16}. Where needed, Reduced Order Models (ROM), also called surrogate models, have been developed to considerably speed up waveform generation, without losing accuracy~\cite{Field+13, Puerrer14, Bohe+16}. These tools provide efficient non-precessing IMR Fourier-domain waveforms. Importantly for this work, the resulting waveforms can be represented by an amplitude and phase for each mode, with only a few hundred samples~\cite{Puerrer14}.[JB:I cut this down significantly.]} Our challenge will be to accurately complete the representation of instrumental signals including precession effects and the LISA instrumental response and precession effects without sacrificing the computational efficiency enabled by these fast waveform templates.

A complete representation of spin effects in fast IMR templates still remains a frontier of gravitational wave signal modelling. In presence of misaligned spins, the system will endure precession of the orbital plane as it evolves, leading to modulations of the signal as seen by the observer~\cite{ACST94}.
%Precessing EOB templates incorporate all six parameters of the two spins~\cite{Pan+13, BTB16} but still require a costly time-domain integration; while precessing phenomenological templates~\cite{Hannam+13} decrease the dimensionnality by using effective spin parameters representing only the dominant effects of the spins.
\jgb{[JB:I cut this down too.]}

A promising approach to modeling the effect of precession on the emitted waveform, it has been proposed~\cite{BCV03b, BCPTV05, Schmidt+10, OShaughnessy+11, Boyle+11} to decompose precessing waveforms by performing a time-dependent rotation, following the precession of the orbital plane. The resulting waveform in the rotated frame can then be modelled by a non-precessing waveform, an approximation which is used both in the construction of the inspiral part of precessing EOB waveforms~\cite{Pan+13} and in the construction of precessing phenomenological waveforms~\cite{Hannam+13}. To follow this modelling approach and efficiently create Fourier-domain waveforms, one needs to understand how to translate the time-domain modulations created by the frame rotation into a Fourier-domain transfer function, which is our first motivation.

Beyond a fast representation of the incident waveform, observational analyses also require transforming the signals through some instrumental response. For short duration mergers such as LIGO  has detected, this can be treated by a simple multiplier.  For future instruments though, the instrumental response will be more complicated.

%Prospective parameter recovery studies for LISA observations of comparable-mass binary black hole coalescences aim at characterizing the scientific performance of the future experiment. Although LISA~\cite{LISA17} is still at least a decade from launch, the ESA L3 mission is rapidly progressing into its early stages where mission design choices must be informed by science impact analysis. However, the recent progresses in waveform modelling, most notably the availability of fast and accurate IMR waveforms, have not yet made their way to the field of LISA data analysis studies. In the past, many such studies used post-Newtonian, inspiral-only models, and were limited to the cheaper Fisher-matrix formalism instead of a full Bayesian analysis (among numerous works, see e.g. \cite{Cutler97,LISAPE09}). Refs.~\cite{McWilliams+09,McWilliams+10,McWilliams+11} explored the effects of including the merger and ringdown, keeping the mass parameters fixed. However, the Fisher matrix analysis, although considerably cheaper to perform, is valid only in the limit of a high signal to noise ratio (SNR). Several implementations of Bayesian analyses were developed~\cite{CP06,Wickham+06,Babak+08,GP09,Feroz+09,Petiteau+10,PC13} (see also~\cite{MLDC4} and references therein), with Ref.~\cite{PC15} proposing a direct comparison of Fisher and Bayes analyses, but they were limited to inspiral-only waveforms. One exception was Ref.~\cite{Babak+08} where IMR signals were used in a Bayesian study, but there the masses of the black holes were kept fixed. In the recent Ref.~\cite{Klein+15}, the limitation to inspiral-only precessing post-Newtonian waveforms was compensated by an SNR reweighting procedure. Thus, due mainly to their high computational cost, full-fledged Bayesian parameter estimation simulations for IMR signals are still missing. \jgb{[Maybe there is too much detail in this paragraph?]}

%Prospective parameter recovery studies for LISA observations of comparable-mass binary black hole coalescences aim at characterizing the scientific performance of the future experiment. Although LISA~\cite{LISA17} is still at least a decade from launch, the ESA L3 mission is rapidly progressing into its early stages where mission design choice must be informed by science impact analysis. However, the recent progresses in waveform modelling, most notably the availability of cheap IMR waveforms, have not yet made their way to the field of LISA data analysis studies. In the past, such studies used mainly post-Newtonian, inspiral-only models, and were limited to the Fisher-matrix formalism instead of full Bayesian analysis (among numerous works, see e.g. \cite{Cutler97,LISAPE09}). Refs.~\cite{McWilliams+09,McWilliams+10,McWilliams+11} explored the effects of including the merger and ringdown, keeping the mass parameters fixed. However, the Fisher matrix analysis, although considerably cheaper to perform, is valid only in the limit of a high signal to noise ratio (SNR) and has known shortcomings~\cite{Vallisneri08}. Several implementations of Bayesian analyses were developed~\cite{CP06,Wickham+06,Babak+08,GP09,Feroz+09,Petiteau+10,PC13} (see also~\cite{MLDC4} and references therein), with Ref.~\cite{PC15} proposing a direct comparison of Fisher and Bayes analyses, but they were limited to inspiral-only waveforms. One exception was Ref.~\cite{Babak+08} where IMR signals were used in a Bayesian study, but there the masses of the black holes were kept fixed. In the recent Ref.~\cite{Klein+15}, the limitation to inspiral-only precessing post-Newtonian waveforms was compensated by an SNR reweighting procedure. Thus, due mainly to their high computational cost, full-fledged Bayesian parameter estimation simulations for IMR signals are still missing. \jgb{[Maybe there is too much detail in this paragraph?]}

\jgb{Whereas LIGO and Virgo are typically sensitive to chirping binaries for a minute or less, LISA signals may accumulating over months or years.
The response of a LISA-type instrument is thus time-dependent~\cite{Cutler97}.
The motion and change of orientation of the detector constellation along its orbit lead to significant time variability in the form of a modulation and a varying delay. These effects then convey information about the localization of the gravitational-wave source in the sky. Direct implementation of the detector response is straightforward\cite{SynthLISA,LISACode,LISASim} but at prohibitive computational cost for parameter-estimation analyses. To leverage the performance of state-of-the-art Fourier-domain IMR templates~\cite{BTB16,Khan+15}, we must efficiently process the signals through the time-dependent response of the dectector while staying in the Fourier domain.}

The purpose of this paper is to introduce a formalism for efficiently processing signals through a time-domain modulation and delay within the Fourier domain, while retaining the compactness of a Fourier-domain amplitude and phase representation of the signals. This will allow us to address both the issue of the Fourier-domain response of the LISA instrument, as well as the issue of Fourier-domain precession modulation for IMR signals from precessing binaries.

In previous works focused on gravitational-wave inspirals, the Stationary Phase Approximation (SPA) (see e.g.~\cite{Thorne300, CF94}) has been often used for this purpose. While the SPA is a common approximation to compute the Fourier transform of non-precessing signals during the inspiraling phase, it is not applicable for IMR waveforms.

\jgb{In the case of precessing binaries, applying the SPA directly to the modulated signal is also prone to pathologies. In Ref.~\cite{KCY13,KCY14}, a formalism (called shifted uniform asymptotics or SUA) was introduced to go beyond the SPA and compute more accurately the modulation in the Fourier domain; however, this formalism still relies on the SPA for the underlying precessing-frame signal, and is as such limited to inspiraling signals. \textbf{[JB: Did I capture your thoughts correctly in the following.]}
The simplified treatment of the precession response in the phenomenological waveforms in Ref.~\cite{Hannam+13} takes another approach, treating the precession modulation in the frequency domain by directly associating the Fourier frequency with (a multiple of) the post-Newtonain orbital frequency. As we will explain, this corresponds to the zeroth-order approximation of the SUA.}

In the case of the LISA response, the SPA provides a natural map from time-domain (for the orbit) to Fourier-domain (for the signal)~\cite{Cutler97}. Ref.~\cite{Klein+15} included the orbital motion of the detector in the SUA treatment. Consistently extending these previous approaches to the merger-ringdown part of the signals, including the delays in the LISA case, is part of the objectives of this paper.

We seek to overcome two limitations in such previous approaches.  The first is that  SPA-based methods are not applicable to IMR waveforms. Second, there is often no clear way to improve the accuracy of these methods beyond the intuitive leading order treatment in order to meet the high-accuracy needs of future detectors. Our approach exploits separation of time-scales approximations, based on a general treatment directly in the Fourier domain of slowly varying delays and amplitude modulations for chirping waveforms with slowly varying amplitude and phase.

The plan of the paper is as follows. In Sec.~\ref{sec:motivation}, we provide a general presentation of problem of Fourier-domain modulation and introduce the relevant timescales for both the response of LISA-type detectors and the modulation of precessing signals. In Sec.~\ref{sec:formalism}, we present our general formalism, give its leading order approximation as well as higher-order corrections, introduce new timescales based on the Fourier-domain signal, refine the previous results for both the quadratic-in-phase corrections and the treatment of the delays, and make the connection to previous approaches of the problem. We then apply our formalism to the response of the LISA detector in Sec.~\ref{sec:LISA}, and to the case of signals from precessing binaries in Sec~\ref{sec:precession}. We discuss and summarize our results in Sec.~\ref{sec:discussion}.

\SM{[say that we reverse point of view: waveform is given FD, modulation TD, instead of waveform+modulation TD]}

\SM{[say a word of how to gauge relevance of a given error: 1/SNR in direct error ? We won't give anything but waveform reconstruction errors...]}

\SM{[style: below use texttt for models and codes (like \texttt{SpEC})]}

%%%%%%%%%%%%%%%%%%%%%%%%%%%%%%%%%%%%
%%%%%%%%%%%%%%%%%%%%%%%%%%%%%%%%%%%%

\section{GW signals in the frequency domain}
\label{sec:motivation}

%%%%%%%%%%%%%%%%%%%%%%%%%%%%%%%%%%%%

%\subsection{Basic considerations}
%\label{subsec:basic}

\jgb{To motivate our signal treatment we consider challenges of both the LISA response and GW precession applications in terms of a more general problem which covers both domains. In each case must determine the Fourier transform a signal $\tilde s(f)$ which results from processing some incident waveform $h(t)$ through some combination of modulations and delays.}

In the general problem, we apply a time-varying delay $d(t)$ to the waveform followed by a multiplicative modulation function $F(t)$,
\be
\label{eq:delay-mod-defs}
	h_{d}(t) = h(t+d(t)) \,, \quad s(t) = F(t)h_{d}(t) \,.
\ee
We then seek an \textit{efficient} way to compute the Fourier transform $\tilde{s}(f)$, expressed by means of a Fourier-domain transfer function $\calT$ such that
\be\label{eq:deftransfer}
	\tilde{s}(f) \equiv \calT(f) \tilde{h}(f) \,. 
\ee

As is well known, GW signals decomposed in spin-weighted spherical harmonic components are smoothly varying functions of frequency in amplitude/phase form\cite{smoothAmpPhaseRefs}. An important consequence is that these signal components can be accurately represented in terms of a relatively coarsely sampled frequency grid. This property will also extend to the transfer functions, allowing us to keep a compact representation of the full signals. (See Appendix~\ref{app:notation} for details of our notation and conventions for Fourier transforms and spherical harmonics.)

In the case of precessing binaries, the delays will be absent and the modulation functions will be the time-dependent Wigner coefficients applied to rotate the waveform from an \jgb{precessing orbital frame, in which waveform modes exhibit smooth amplitude and phase variation}, to an inertial frame where the observations take place. In the case of a LISA-type detector, the signal $h(t)$ will simply be the waveform in a fixed heliocentric frame, and the delays will come from the motion of each detector against the wave front, while the modulation will represent the time-variation in the detector orientation.

In full generality computing the transfer function $\calT(f)$ would require a convolution, a costly (discretized) integral over the full frequency domain for each value of $f$. For our context though, there are some properties of $h(t)$, $d(t)$ and $F(t)$ which we can exploit for a more efficient computation. In particular, we will be able to exploit the separation of the different timescales in the problem.

First, the gravitational waveforms present a clear separation between the timescales of orbital motion and radiation-reaction. This leads to a general feature of GW signals from compact binaries, during the inspiral phase but also for black hole mergers, that the signal is relatively localized in time-frequency. The localization is particularly clear during the inspiral phase, where the SPA (see Sec.~\ref{subsec:SPA} below) provides an unambiguous time-to-frequency correspondence, however we will find that the applicability of the SPA will not be a limiting factor of our approach.

Second, the delay and modulation functions we consider are much more slowly varying than the GW signal. The relevant timescales are either the precession timescale for precessing binaries or the fixed annual-orbital motion timescale for the LISA response. This means that the modulation and delay have relatively compact support in the Fourier domain, hence the convolution with the signal will be localized in frequency, which justifies writing its output as a transfer function as in~\eqref{eq:deftransfer}.

Together, these observations lead us to expect that, for a given $f$, only a limited range of times should be relevant in $d(t)$ and $F(t)$ so that we may expect to find a treatment for the transfer function that would be local in time for the modulation and delay. This general idea has already been applied for inspiral signals in the limit of an extremely slowly varying $d(t)$ and $F(t)$, where intuitively we should be able to simply evaluate them at the time given by the time-to-frequency correspondence of the SPA.

Furthermore, a natural quantitative criterion for the applicability of this idea is given by the comparison of the radiation-reaction timescale of the signal with the modulation timescale. In other words, the change in frequency of the signal over a characeteristic time of the modulation should be small for the separation of timescales to work.

Our objective is to find an approximate treatment for $\calT(f)$ which allows us to exploit these properties without relying on unnecessary limiting assumptions for the signal (like the limitation of the SPA to inspiral signals), which is extensible to the high-accuracies which will be required by LISA and other future gravitational-wave instruments, while being computationally efficient and widely applicable to GW analysis. In preparation for developing our formalism we first review the salient features of our application problems.

\subsection{Instrumental modulations and delays for LISA-type detectors}
\label{subsec:modulationLISA}

The response of a detector of the LISA type to an incident gravitational wave can be written in two different, equivalent forms, in terms of phase or frequency measurements. Here we will work with the second representation, which will prove more convenient for our purposes. Moreover, various notation and conventions have been used in the literature to label the spacecraft and describe their orbits. We refer the reader to~\cite{Vallisneri04} for a comparative account on these various conventions. In this work, we will keep close to the conventions of~\cite{Vallisneri04}, which were also used in the Mock LISA Data Challenges (MLDC)~\cite{MLDC4}.

We use a coordinate system centered on the solar system barycenter (SSB), and represent the center of the constellation by the vector $p_{0}$. We introduce the notation $\hatk$ for the propagation vector of the gravitational wave, which we denote by $h_{ij}^{\rm TT}(t)$ in transverse-traceless matrix form, as measured at the SSB (thus, at position $p$, $h_{ij}^{\rm TT}(t, p) = h_{ij}^{\rm TT}(t - \hatk \cdot p)$). We denote by $p_{A}$ ($A=1,2,3$) the position of the individual spacecraft and $n_{l}$ the unit vectors of the three links, with the convention that $n_{3}$ points from 1 to 2.

Written in terms of the fractional laser frequency shifts between two spacecraft, the elementary response of the detector reads~\cite{EW75, RCP04, Vallisneri04}
\begin{align}\label{eq:yslr}
	y_{slr} &\equiv \frac{\nu_{r} - \nu_{s}}{\nu} \nn\\
	&= \frac{1}{2} \frac{n_{l}^{i}n_{l}^{j}}{1 - \hatk\cdot n_{l}} \left[ h_{ij}^{\rm TT}(t - L - \hatk\cdot p_{s}) - h_{ij}^{\rm TT}(t - \hatk\cdot p_{r}) \right] \,.
\end{align}
We will use a rigid instantaneous model, approximating the geometry of the constellation by a moving equilateral triangle (neglecting the flexing of the arms induced by corrections in the orbits), and evaluating all geometric factors in~\eqref{eq:yslr} at a single time $t$ (neglecting point-ahead corrections). The additional delay $L$ (we use $c=1$) in the first term represents the light propagation time along the arm between spacecraft $s$ and $l$.

The first- and second-generation TDI observables are then built as combinations of these basic building blocks, evaluated at delayed times. Since in our rigid approximation these delays will take a simple form in Fourier domain, we will focus in this paper on the $y_{slr}$ observables. Sec.~\ref{subsec:modelLISA} provides more details on the response of LISA-type detectors, on the TDI observables, and on the approximations that enter the derivation of the basic response~\eqref{eq:yslr} above.

The structure of~\eqref{eq:yslr} is such that one can build the full signal from individual contributions of the form
\be
	s(t) = F(t) h(t + d(t)) \,.
\ee
Here, $h(t)$ represents one of the individual modes building the full gravitational wave signal (see Eq.~\eqref{eq:defmodes}), $d(t)$ represents the time-varying delays of the form $-\hatk\cdot p_{A}(t)$, and $F(t)$ incorporates all the relevant geometric prefactors.

For the LISA response, the functions $F(t)$ and $d(t)$ vary on a timescale of one year, with a frequency $f_{0} = 1/\mathrm{yr} \simeq 3.169\times10^{-8} \mathrm{Hz}$ (we will also use $\Omega_{0} = 2\pi f_{0}$). When neglecting small non-periodic orbital perturbations (such as those caused by the influence of the Earth and of the other planets on the orbits), both $F(t)$ and $d(t)$ are periodic. It will also be useful to separate the delays in two types of terms: the first, $d_{0} = -\hatk\cdot p_{0}$, relates the waveform at the SSB to the waveform at the center of the constellation, whereas the second, $d_{c}$, represents the various delays between the spacecraft of the constellation. \jgb{Concretely, we assume the basic reference design parameters of the LISA mission\cite{LISA17} recently selected by the European Space Agency(ESA), with orbit radius $R=1\,\mathrm{au}$ and armlength $L=2.5\times10^{6}\mathrm{km}$, and with $d_{0} \sim R/c \simeq 500\mathrm{s}$ and $d_{c} \sim L/c \simeq 8\mathrm{s}$.}

%%%%%%%%%%%%%%%%%%%%%%%%%%%%%%%%%%%%

\subsection{Precession modulation for spinning binaries}
\label{subsec:modulationPrec}

For spinning compact objects, \jgb{angular momentum interactions typically} lead to the precession of the orbit~\cite{Apostolatos+94, Kidder95}, which can have large effects on the waveform. In particular, it breaks the planar symmetry of the gravitational wave emission, causing modulations that are especially important for systems that are observed edge-on.

A number of authors~\cite{BCV03b, BCPTV05, Schmidt+10, OShaughnessy+11, Boyle+11} have suggested that the effect of the precession can be modeled, to a good approximation, by a time-dependent rotation of a effectively non-precessing waveform. This allows for a modelling approach where one separately models a precessing frame following the evolution of the plane of the orbit, and approximates the waveform in this precessing frame by using a effective non-precessing model. Different prescriptions have been proposed for the construction of a precessing frame from the waveform itself~\cite{Schmidt+10, OShaughnessy+11, Boyle+11}.

If $(\alpha, \beta, \gamma)$ are the Euler angles relating the precessing frame to the inertial frame in the $(z,y,z)$ convention, the modes in the inertial frame $h_{\ell m}^{\rm I}$ are then related to the modes in the precessing frame $h_{\ell m}^{\rm P}$ by~\cite{Goldberg+67}
\be\label{eq:wignerrotintro}
	h_{\ell m}^{\rm I} = \sum\limits_{m=-\ell}^{\ell} \calD^{\ell *}_{mm'} (\alpha,\beta,\gamma) h_{\ell m'}^{\rm P} \,,
\ee
where the $\calD^{\ell}_{mm'}$ are Wigner matrices~\cite{} \SM{[C]}. Notice that there is no mixing of the modes with different values of $\ell$.

If we make the assumption that the precessing-frame waveform $h^{\rm P}$ is approximated by a non-precessing model that provides us with a smooth Fourier-domain amplitude and phase, then the problem reduces to computing the Fourier transform of the signal
\be\label{eq:defmodulationprec}
	s(t) = F(t) h(t) \,,
\ee
where the modulation function $F(t)$ is given by a Wigner matrix and depends on time through the Euler angles $(\alpha, \beta, \gamma)(t)$. We will discuss more in Sec.~\ref{subsec:modelPrec} \SM{[R]} the limitations of the assumptions used to perform the decomposition above.

In Eq.~\eqref{eq:defmodulationprec} above, the modulation function $F$ has time variations on the precessional timescale, which evolves throughout the inspiral. The separation between the precessional timescale and the orbital timescale will be crucial for our analysis. In the limit of low frequencies, we will see in Sec.~\ref{} \SM{[R]} below that, although these timescales become more and more separated, the decrease in the chirping rate gives raise to a corrective contribution that does not vanish in this limit. In the other limit, as the system gets close to merger, the separation of timescales becomes weaker, and we will explore in Sec.~\ref{} \SM{[R]} the application of our formalism to the merger and ringdown phase.

%%%%%%%%%%%%%%%%%%%%%%%%%%%%%%%%%%%%

\subsection{Stationary phase approximation}
\label{subsec:SPA}

As a preliminary step, we recall here an approximation widely used for similar purposes in treating gravitational waves signals emitted by inspiraling binaries, the Stationary Phase Approximation (thereafter SPA, and also sometimes called the steepest-descent method). It applies in general to chirping signals, and we refer the reader to~\cite{FC93, CF94} for details. One first writes the time-domain signal in amplitude and phase form as $h(t) = a(t) e^{-2i\varphi(t)}$, where for gravitational wave signals $\varphi$ will correspond to the orbital phase, with the orbital frequency being $\omega = \dot{\varphi}$. In order to keep close to the notation used in the gravitational wave literature, we introduced a factor of 2 in the phase of the wave, which is appropriate for the dominant 22 harmonic of the signal. The approximation then applies to signals verifying the conditions\footnote{Note that the third condition is not always written explicitly, in particular not in Refs.~\cite{FC93, CF94}. It becomes important if we generalize $a$ to be an envelope function incorporating a modulation.} \SM{[say that the third is sometimes missing ?]} \SM{[add reference to Finn-Chernoff, Cutler-Flanagan where the third one is missing]}
\be\label{eq:conditionsSPA}
	\left| \frac{\dot{a}/a}{\omega} \right| \ll 1\,, \quad \left|\frac{\dot{\omega}}{\omega^{2}} \right| \ll 1\,, \quad \left| \frac{(\dot{a}/a)^{2}}{\dot{\omega}} \right| \ll 1 \,.
\ee
Since the integral~\eqref{eq:defFT} defining the Fourier transform is rapidly oscillatory unless the term $2\pi f t$ cancels the evolution of $-2\varphi(t)$, its support is well centered around the point of stationary phase. This determines the time-to-frequency correspondence in the stationary phase approximation, and leads to the definition of this time as an implicit function of frequency by the relation
\be\label{eq:deftfSPA}
	\omega(\tfSPA) = \pi  f \,.
\ee
For a chirping signal of increasing phase, $\omega>0$ and $\dot{\omega}>0$, there is a unique point of stationary phase, located in the positive frequency range $f>0$. Using the conditions above, one can formally expand the signal around $\tfSPA$ to quadratic order in time, according to
\begin{align}
	\tilde{h}_{\rm SPA} (f) &\simeq a(\tfSPA) \exp\left[2i\pi f \tfSPA-2i\varphi(\tfSPA) \right] \nn\\
	& \qquad \cdot \int \ud t \, e^{-i \dot{\omega} (\tfSPA) (t-\tfSPA)^{2}} \,.
\end{align}
Note that to be able to treat the amplitude as a constant in the integral above, we used the third condition in~\eqref{eq:conditionSPA}. The resulting complex Gaussian integral yields\footnote{The expressions given here are valid for the dominant mode $h_{22}$ of the waveform. They can be generalized to other modes $h_{\ell m}$, $m\neq 0$ with phase $e^{-im\varphi}$ as follows: $\Psi$ acquires a factor $m/2$, $A$ a factor $\sqrt{2/m}$, and $v$ (see~\eqref{eq:SPAN}) has to be replaced by $(2/m)^{1/3} v$.}
\begin{subequations}
\begin{align}
	\tilde{h}_{\rm SPA}(f) &= A_{\rm SPA}(f) e^{-i\Psi_{\rm SPA}(f)} \,, \\
	A_{\rm SPA}(f) &= a(\tfSPA) \sqrt{\frac{\pi}{\dot{\omega}(\tfSPA)}} \,, \\
	\Psi_{\rm SPA}(f) &= 2\varphi(\tfSPA) - 2\pi f \tfSPA + \frac{\pi}{4} \,. \label{eq:PsiSPA}
\end{align}
\end{subequations}
Ref.~\cite{Droz+99} evaluated the first correction to this approximation, within the context of post-Newtonian signals, and found that it can be considered as a term of the fifth post-Newtonian order, beyond the accuracy level of our current best models~\cite{BlanchetLiving}.

To understand the separation of timescales in the problem, it will be useful to have at hand the leading-order scaling laws for an inspiral (labeled the Newtonian order in the PN language). Although inaccurate for the purpose of waveform modelling, these leading-order estimates will give useful orders of magnitude of the relevant timescales in the inspiral. For a binary with masses $m_{1}, m_{2}$, we define the total mass $M=m_{1}+m_{2}$ and the symmetric mass ratio $\nu = m_{1}m_{2}/M^{2}$. Introducing a time of coalescence $t_{c}$, the relations between the orbital frequency and phase and the time to coalescence $t_{c} - t$ are then given by
\begin{subequations}\label{eq:omegaphiN}
\begin{align}
	\omega(t) &= \left[ \frac{256\nu}{5c^{5}} (GM)^{5/3} (t_{c}-t) \right]^{-3/8} \,, \\
	\varphi(t) &= -\left[ \frac{c^{3}}{5 G M \nu^{3/5}} (t_{c}-t) \right]^{5/8} \,.
\end{align}
\end{subequations}
As for the leading-order time-domain amplitude of the $22$ mode, we have~\cite{BlanchetLiving}
\be\label{eq:a22N}
	a_{22}^{\rm N} (t) = \frac{2 G M \nu v^{2}}{D c^{2}} \sqrt{\frac{16 \pi}{5}} \,,
\ee
where we set $v = (G M\omega/c^{3})^{1/3}$ and where $D$ is the luminosity distance to the observer. For this Newtonian inspiral, applying the SPA gives for the $22$ mode $\tilde{h}(f) = A_{\rm N}(f)e^{-i\Psi_{\rm N}(f)}$ with \SM{[check that total mass convention is $M$ everywhere]}
\begin{subequations}\label{eq:SPAN}
\begin{align}
	A_{N}(f) &= \frac{G^{2}M^{2} \pi}{Dc^{5}} \sqrt{\frac{2\nu}{3}} v^{-7/2}\,, \label{eq:ASPAN}\\
	\Psi_{\rm SPA}^{\rm N}(f) &= \phi_{0} - 2\pi f t_{0} - \frac{3}{128\nu v^{5}} \,, \label{eq:PsiSPAN} 
\end{align}
\end{subequations}
where $v=(G M \pi f/c^{3})^{1/3}$ according to the SPA correspondence~\eqref{eq:deftfSPA}, and where $t_{0}, \phi_{0}$ are constants. It is also customary to rewrite the above relations in terms of the chirp mass $\Mchirp \equiv M\nu^{3/5}$, which is the only mass combination characterizing the signal at the leading PN order.

In this Newtonian, low-frequency limit one can check that each of the combinations \eqref{eq:conditionsSPA} indeed vanish at $\calO{(v^5)}$. but as the system approaches merger, these condition are no-longer satisfied. Near and after the merger, the SPA treatment is not applicable.
%, as can be seen by the breakdown of the relation~\eqref{eq:deftfSPA} as $\omega$ asymptotes the quasi-normal mode frequency in the ringdown. Because of this limitation, we aim at developing a formalism that would not rely on this approximation.
Nonetheless, the SPA has been a useful workhorse in many gravitational analyses involving frequency domain transfer functions of the form~\eqref{eq:deftransfer}. The usual approach is simply to replace any time dependencies appearing the transfer function using \eqref{eq:deftfSPA}. \jgb{We will reference the SPA treatment, as a familiar touchstone, as we develop a more general formalism.}

%%%%%%%%%%%%%%%%%%%%%%%%%%%%%%%%%%%%
%%%%%%%%%%%%%%%%%%%%%%%%%%%%%%%%%%%%

\section{Perturbative Fourier-domain approach to modulations and delays}
\label{sec:formalism}

%%%%%%%%%%%%%%%%%%%%%%%%%%%%%%%%%%%%

In this section we develop a perturbative Fourier-domain formalism for treating time-delays and temporally multiplicative signal transformations, exploiting the separation of timescales in the problem.

\subsection{Fourier transform of a modulated and delayed signal}
\label{subsec:FTgeneral}

We begin by expression our delay and modulation function definitions in \eqref{eq:delay-mod-defs} using the Fourier transform (note our unusal convention~\eqref{eq:defFT}),
\begin{align}
h_{d}(t) & =  h(t+d(t)) \nonumber\\
&=\int \ud f \, e^{-2i\pi f (t+d(t))}\tilde{h}(f) \,,
\end{align}
and
\begin{align}
  \tilde{s}(f) &= \mathrm{FT} \left[ F h_{d}\right] (f) \nn \\
  &= \int \ud t \, e^{2i\pi f t} F(t)  \int \ud f' \, e^{-2i\pi f' (t+d(t))}\tilde{h}(f') \nn\\
	&= \int \ud f' \, \tilde{h}(f-f') \int \ud t \, e^{2i\pi f' t} e^{-2i\pi (f-f') d(t)} F(t) \,.
\end{align}
The last equation can then be rewritten as a generalized convolution integral with a frequency-dependent Kernel, according to
\be\label{eq:FDkernel}
	\tilde{s}(f) = \int \ud f' \, \tilde{h}(f-f') \tilde{G}(f-f',f') \,,
\ee
where we introduced the frequency-dependent function of time $G(f,t)$, and its Fourier transform in the auxiliary frequency $f'$, denoted by $\tilde{G}(f,f')$, as
\begin{subequations}\label{eq:defG}
\begin{align}
	G(f,t) &= e^{-2i\pi f d(t)} F(t) \,, \\
	\tilde{G}(f,f') &= \int \ud t \, e^{2i\pi f' t} G(f,t) \,.
\end{align}
\end{subequations}
In the absence of delays $d(t)$, as in the case of precessing binaries, the function $G$ loses its frequency-dependence and becomes a modulation in the form of a function of time $F(t)$, and the result~\eqref{eq:FDkernel} above reduces to the familiar convolution theorem for the Fourier transform.

A direct computation of the generalized convolution~\eqref{eq:FDkernel} using~\eqref{eq:defG} will generally be computationally demanding, but we can exploit the separation of timescales in the problem to seek an accurate but efficient approximation.  \jgb{We will explore approximations which allow the integral in~\eqref{eq:FDkernel} to be computed, so that the signal may be expressed the form of~\eqref{eq:deftransfer}, with the modulation and delay reduced to a multiplicative transfer function $\mathcal T(f)$.}

\jgb{For example, under appropriate conditions with slowly varying modulations and delays, we can expect that} the Fourier transform $\tilde{G}(f,f')$ should have a compact support, limited to $f' \in [-f_{\rm max}, f_{\rm max}]$ with $f_{\rm max}$ a maximal frequency for the modulation, roughly the inverse of its characteristic timescale. This makes the convolution integral~\eqref{eq:FDkernel} localized in frequency, as the waveform is to be evaluated only at frequencies close to $f$. We can then approximate $\tilde{h}(f-f')$ by $\tilde{h}(f)$, with a simple Taylor expansion for a perturbative treatment of the difference between the two. Doing so, we will recover at leading order a locality in time: the response is approximately reduced to an evaluation of the modulation and delay at a representative signal-dependent time $\tf$, which will be equivalent to $\tfSPA$ for inspiral signals.

Now consider the limitations to this straightforward argument. The first is that we did not yet specify what $f_{\rm max}$ should be compared with, in order for the approximation to work. As shown in Sec.~\ref{subsec:SPA}, the Fourier-domain expressions for the amplitude and phases inspiral signals are steep power-laws. We will therefore need a quantitative criterion to ensure $\tilde{h}(f-f')$ does not vary too much on the range $f'\in [-f_{\rm max}, f_{\rm max}]$. Second, although clear when considering a fixed characteristic timescale for the modulation and delay (like in the LISA case), the above argument does not apply as such to modulations with a varying timescale (like in the case of precessing binaries, where the precession goes faster when arriving at merger). We will show below that what will be relevant is the characteristic timescale at the time $t_{f}$ associated to $f$. 

In presence of delays, the above is also complicated by the additional delay phases in the signal Fourier transform. Depending on the frequency $f$, this can lead $G(f,t)$ to have faster variations than its nominal timescale ($1\mathrm{yr}$ for LISA). As we will see in Sec.~\ref{} \SM{[R]}, \jgb{to derive useful approximations we will not only need to compare modulation timescales, but also to constrain dimensionless factors of the form $2\pi f d$.}

\jgb{In the next subsections we will examine the approximations necessary to allow a local-in-frequency response treatment, and consider low-order corrections which can improve accuracy in cases where the approriate conditions are more marginally realized.}

%%%%%%%%%%%%%%%%%%%%%%%%%%%%%%%%%%%%

\subsection{Leading order: the local-in-frequency approximation}
\label{subsec:LLP}

\begin{figure}
  \centering
  \includegraphics[width=.98\linewidth]{plots/tf_py.pdf}
  \caption{Time-to-frequency correspondence $t_{f}$, as defined directly from the phase of the Fourier-domain signal as in~\eqref{eq:deftf}, in geometric units. We show only the high-frequency part of the signal, corresponding to the merger region. The PhenomD waveforms have been aligned such that the time-domain amplitude (obtained by an IFFT \SM{[check acronym]}) peaks at $t=0$. The two aligned spins are equal, with components of $0.95$ ($++$, blue), $0$ ($00$, red), and $-0.95$ ($--$, yellow), for mass ratios $q=1$ (full line) and $q=8$ (dashed). The vertical lines shows the time-domain instantaneous frequency at the peak $\omega^{22}_{\rm peak}/(2\pi)$, again measured from an IFFT, and the fact that the curves $t_{f}$ do not pass exactly by the crossing with the $t=0$ horizontal line reflects the fact that the link between time-domain frequency and Fourier frequency $f$ is only approximate at merger. Note that $t_{f}$ increases to at most $\sim 30M$ after the peak, and is not monotonous. \SM{[add color to the vertical lines]} \SM{[add dots to highlights the crossings]}}
  \label{fig:tf}
\end{figure}

As a first step, we perform a formal leading-order expansion of the signal $\tilde{h}(f-f')$ around $f$. We use the amplitude/phase decomposition~\eqref{eq:defAPsi}, treat the Fourier-domain amplitude $A$ as a constant, expand the Fourier-domain phase $\Psi$ to the first order, and discard the $f'$ dependence in the first argument of $\tilde{G}(f-f', f')$.

For the signal, then, we have
\be
	\tilde{h}(f-f') \simeq A(f) \exp\left[ -i\left( \Psi(f) - f' \frac{\ud \Psi}{\ud f} \right) \right] \,,\label{eq:leadingorderwf}
\ee
where here and in the rest of the paper, such derivatives with respect to the frequency will always be evaluated at $f$. Plugging this relation into~\eqref{eq:FDkernel}, we obtain
\begin{align}
	\tilde{s}(f) &\simeq \tilde{h}(f) \int \ud f' \, \exp\left[ i f' \frac{\ud \Psi}{\ud f} \right] \tilde{G}(f,f') \nn\\
	&= \tilde{h}(f) G\left( f, -\frac{1}{2\pi} \frac{\ud \Psi}{\ud f} \right) \,,\label{eq:leadingorderresponse}
\end{align}
\jgb{which we can think of as a local evaluation of the kernel function $G(f,t)$ at a frequency-dependent effective time}
\be\label{eq:deftf}
	\tf \equiv -\frac{1}{2\pi} \frac{\ud \Psi}{\ud f} \,.
\ee
It is worth noting that a shift in time of the time-domain signal will, by virtue of~\eqref{eq:shifttime}, be appropriately propagated to $t_{f}$. Because of the freedom of adding a linear term to $\Psi(f)$ by simply shifting the signal in time, no assumption can be made on the smallness of the first derivative of the phase, and this is really a leading order approximation.

The definition~\eqref{eq:deftf} is a straightforward generalization of the time-to-frequency correspondence at the heart of the SPA~\eqref{eq:deftfSPA}. Indeed, using~\eqref{eq:deftfSPA} one can verify that the derivative of the SPA phase $\Psi_{\rm SPA}$~\eqref{eq:PsiSPA} with respect to $f$ yields back $\tfSPA$, as
\be
	\tfSPA = -\frac{1}{2\pi} \frac{\ud \Psi_{\rm SPA}}{\ud f} \,.
\ee
However, our defintion~\eqref{eq:deftf} refers only to the Fourier-domain waveform. We do not need to relate the frequency $f$ to a time-domain frequency like the orbital frequency $\omega$, and the defintion is independent of the SPA being valid or not for the underlying signal $\tilde{h}$.

The main advantage of our time-of-frequency function~\eqref{eq:deftf} is that it extends naturally to the merger-ringdown part of the signals. Fig.~\ref{fig:tf} shows the behaviour of the time function $\tf$ around merger for six example waveforms, for mass ratios $q=1$ and $q=8$ and for aligned spin components $\chi=0.95,0.,-0.95$. In particular, one should note that $\tf$ is not monotonically increasing with frequency anymore after reaching in the high-frequency part of the waveform, corresponding to the ringdown. \jgb{As long as the Fourier phase is differentiable $t_{f}$ is a well-defined function of $f$. While its non-monotonicity would forbid an unambiguous definition of a reciprocal frequency-of-time function $f(t)$, like that in the SPA, no such function will be needed in our treatment.}

\jgb{With \eqref{eq:leadingorderresponse} we have brought the modulated and delayed signal in to the form~\eqref{eq:deftransfer} with transfer function}
\be\label{eq:transferlocal}
	\calT_{\rm local}(f) = G(f, \tf) = F(t_{f}) e^{-2i\pi f d(t_{f})}\,.
\ee
The interpretation of this approximation is straightforward: the signal is simply multiplied by the response function evaluated at the time $\tf$, the delay phase becoming the same linear phase contribution as one would have in~\eqref{eq:shifttime} with a time shift $d(t_{f})$ treated like a constant. The locality in frequency at $f$ for $\tilde{h}$ translates into a locality in time at $t_{f}$ for $F,d$.

%%%%%%%%%%%%%%%%%%%%%%%%%%%%%%%%%%%%

\subsection{Taylor expansion in the Fourier domain}
\label{subsec:TaylorFD}

\jgb{If the effective width of of the kernel function $G(f,f’)$ is not quite negligible compared to the scale of significant variations with $f$ it can be useful to extend our approach beyond the leading-order approximation. With the waveform represented in phase-and-amplitude form~\eqref{eq:A9}, the elements of~\eqref{eq:FDkernel} may be Taylor expanded in the variable $f'$: \SM{[change notation, do not use $N$ as sum boundary for $\Psi$]}
\begin{subequations}\label{eq:expandfprime}
\begin{align}
	\Psi(f-f') &= \Psi(f) + 2\pi f' \tf + \sum\limits_{p\geq 2}^{N} \frac{(-1)^{p}}{p!} {f'}^{p} \frac{\ud^{p} \Psi}{\ud f^{p}} \,, \label{eq:expandPsi}\\
	A(f-f') &= A(f)+A(f) \sum\limits_{q\geq 1} \frac{(-1)^{q}}{q!} {f'}^{q} \frac{1}{A}\frac{\ud^{q} A}{\ud f^{q}} \,, \label{eq:expandA}\\
	\tilde{G}(f-f', f') &=\tilde G(f,f')+ \sum\limits_{r\geq 1} \frac{(-1)^{r}}{r!} {f'}^{r} \frac{\partial^{r} }{\partial f^{r}}  \tilde{G}(f,f') \label{eq:expandG} \,,
\end{align}
\end{subequations}
using the definition of $t_{f}$ introduced in~\eqref{eq:deftf}.}

\jgb{The leading order transfer function~\eqref{eq:transferlocal} is obtained by leaving off all the terms in the sums.  In the following, we will consider the resulting transfer functions when keeping some of the next few terms in each of these expansions.} 
Notice that we expand $\tilde{G}(f-f',f')$ in $f'$ only in its first argument, and that we can commute the $f$-derivatives of $G$ with the Fourier transform operation.

\jgb{In some cases, we will also expand the exponential of the phase
\be
\exp\left[ -i\Psi(f) \right] \simeq 1+\sum_{j\geq1}\frac{(-i\Psi(f))^j}{j!}
\label{eq:expandexp}
\ee
to obtain a pure $f'$-expansion. The resulting power series in $f'$ can then be recast as a temporal Taylor series}, by applying the formal derivative rule
\allowdisplaybreaks
\be
	\int \ud f'\, {(-2i\pi f')}^{n} \frac{\partial^{m}}{\partial f^{m}} \tilde{G}(f,f') e^{-2i\pi f' \tf} = \frac{\partial^{m} }{\partial f^{m}} \frac{\partial^{n} }{\partial t^{n}} G (f,\tf) \,.
\ee
The fully expanded result is a rather cumbersome expression with multiple sums, that we will not use directly. Instead, it will be more instructive to separately consider the different expansions in~\eqref{eq:expandfprime}. Later, in Sec.~\ref{subsec:executivesummary} will we come back to combining our results together.

We first consider the effect of the higher-order corrections in~\eqref{eq:expandPsi}. We will find that the third and higher derivatives of the phase are always negligible for our purposes, and we will ignore them. Keeping only the first term of the \jgb{sum in~\eqref{eq:expandPsi}, corresponding to the second derivative of $\Psi$, with just the leading terms from \eqref{eq:expandA} and \eqref{eq:expandG}, and expanding the phase exponential \eqref{eq:expandexp} so that the result can be cast as a Taylor series in time} , we obtain straightforwardly
\be\label{eq:resulttaylorPsi}
	\calT_{\rm phase}(f) = \sum\limits_{p\geq 0} \frac{1}{p!} \left( \frac{i}{8\pi^{2}}\frac{\ud^{2} \Psi}{\ud f^{2}} \right)^{p} \left( \frac{\partial^{2p} }{\partial t^{2p}} G \right)(f, \tf) \,.
\ee
This correction to the transfer function is of particular interest to us. It will be quantitatively dominant over the other ones in most contexts, and it will be shown in Sec.~\ref{subsec:resumquadphase} that it generalizes the previous approach of~\cite{KCY14}. This result shows that the transfer function is signal-dependent, not only through the time-to-frequency correspondence $t_{f}$ but also through the second derivative of the phase $\Psi$.

Similarly, applying the expansion of the Fourier-domain amplitude~\eqref{eq:expandA} while preserving only the leading order terms in \eqref{eq:expandPhi} and \eqref{eq:expandG} gives
\be\label{eq:resulttaylorA}
	\calT_{\rm amp}(f) = \sum\limits_{p\geq 0} \frac{1}{(2i\pi)^{p}p!} \frac{1}{A} \frac{\ud^{p} A}{\ud f ^{p}}  \left( \frac{\partial^{p} }{\partial t^{p}} G \right)(f,\tf) \,.
\ee

Lastly, expanding only the frequency-dependence of $G$ as in~\eqref{eq:expandG} yields
\be\label{eq:resulttaylordelay}
	\calT_{\rm delay}(f) = \sum\limits_{p\geq 0} \frac{1}{(2i\pi)^{p}p!} \left( \frac{\partial^{p} }{\partial f^{p}} \frac{\partial^{p} }{\partial t^{p}} G \right)(f,\tf) \,,
\ee
\jgb{[It seems like it might be better to label this version of the transfer function as ``kernel'' rather than ``delay'' since $G$ here is general. Do we ultimately use this at all?]}
which interestingly looks like a Taylor expansion of $G$, but this time with joint derivatives in frequency and time. This last expansion, taken separately from the other corrections, is signal-independent as it only depends on the kernel function $G$ and not on $A$, $\Psi$.

As will be explained in Sec.~\ref{subsec:executivesummary}, we will also use these formal Taylor expansions to build error measures (constructed as the magnitude of the first term ignored in the series, see~\eqref{eq:deffom}) designed to estimate which of these three types of corrections are important to take into account.

%%%%%%%%%%%%%%%%%%%%%%%%%%%%%%%%%%%%

\subsection{Signal-dependent timescales}
\label{subsec:timescales}

\begin{figure}
  \centering
  %\includegraphics[width=.99\linewidth]{plots/TfTA.pdf}
  \includegraphics[width=.98\linewidth]{plots/TfTA_py.pdf}
  \caption{Fourier-domain amplitude and phase timescales, as defined in~\eqref{eq:defTf} and~\eqref{eq:defTA}, in geometric units for an equal-mass, non-spinning system. The hierarchy of these timescales is roughly the same for higher mass ratio and higher spin systems. The time-domain frequency at merger is represented by the vertical line. The derivatives from which these timescales are built encounter zero-crossings and change sign in the high-frequency range. \SM{[discuss discontinuity of third derivative]}\SM{[add vertical line for RD, make merger thick]}}
  \label{fig:TfTA}
\end{figure}

\jgb{When considering the impact on the transfer function from including the next higher-order phase term, that is the difference between~\eqref{eq:resulttaylorPsi} and~\eqref{eq:transferlocal}, it is natural to define a new timescale, as a function of frequency,
\be\label{eq:defTf}
	\Tf^{2} = \frac{1}{4\pi^{2}}\left| \frac{\ud^{2}\Psi}{\ud f^{2}} \right| \,.
\ee
If the correction is small, this timescale should be small compared to the time-scale of variations in the modulation and delay encoded in $G(f,t)$.}
We can use this notation to rewrite~\eqref{eq:resulttaylorPsi} as
\begin{align}\label{eq:resultdffPsiTf}
	 \calT_{\rm phase}(f) &= \sum\limits_{p\geq 0} \frac{(-i\epsilon)^{p}}{2^{p}p!} \Tf^{2p} \left( \frac{\partial^{2p} }{\partial t^{2p}} G \right)(f, \tf) \,,
\end{align}
where $\epsilon = -\mathrm{sgn}(\ud^{2}\Psi/\ud f^{2} )$ is $1$ in the inspiral.

We can obtain a straighforward physical interpretation of this timescale $\Tf$ by considering inspiral signals for which the SPA is valid.
\jgb{In that case
\be
	\left(\Tf^{\rm SPA} \right)^{2} = -\frac{1}{4\pi^{2}}  \frac{\ud^{2} \Psi_{\rm SPA}}{\ud f^{2}} \,,
\ee
where $\ud^{2}\Psi/\ud f^{2} < 0$ in the SPA with our sign conventions.
Then, taking two derivatives of~\eqref{eq:PsiSPA}, we find
\be\label{eq:TfSPA}
	\Tf^{\rm SPA} \equiv \frac{1}{\sqrt{2\dot{\omega}(\tfSPA)}} \,.
\ee
Thus, when the SPA applies, $\Tf$ corresponds to} the radiation-reaction timescale \SM{[make the connection to $Q$ used for SEOB comparisons in the Nagar group]}: the shorter this timescale, the faster the binary chirps to higher frequencies on its quasi-circular inspiral. 

However, in the same way that our definition~\eqref{eq:deftf} for the time-of-frequency function $t_{f}$ generalizes the SPA definition~\eqref{eq:deftfSPA}, the definition~\eqref{eq:defTf} only refers to the phase of the Fourier-domain signal and does not require introducing a time-domain frequency like $\omega$. This defintion thus extends naturally to the merger-ringdown part of the signal. In this part of the signal, its physical interpretation as the timescale of radiation reaction is obscured, and the second derivative $\ud^{2}\Psi/\ud f^{2}$ can go through zero and change sign, as shown in Fig.~\ref{fig:TfTA}. We include an absolute value in the definition~\eqref{eq:defTf} to allow for this possibility, and keep track of the sign via $\epsilon$.

\jgb{Next consider the impact on the transfer function~\eqref{eq:resulttaylorA} from amplitude corrections beyond leading order.}  This series also leads to the natural introduction of a set of another set of timescales related to the successive derivatives of the amplitude. In an analogous manner to the definition~\eqref{eq:defTf} of the timescale $\Tf$, we can define
\be\label{eq:defTA}
	\left( T_{Ap} \right)^{p} \equiv \frac{1}{(2 \pi)^{p}} \frac{1}{A(f)} \left| \frac{\ud^{p} A}{\ud f^{p}} \right| \,,
\ee
where we included an absolute value to accomodate the possible sign changes in the right-hand side. With this notation, \eqref{eq:resulttaylorA} becomes simply
\be\label{eq:resultATA}
	\calT_{\rm amp}(f) = \sum\limits_{p\geq 0} \frac{1}{p!} (T_{Ap})^{p}  \left( \partial_{t}^{p} G \right) (f,\tf) \,,
\ee
Although the above is written for a generic $p\geq 0$, in practice only the first few of these timescales will be relevant. In this paper we will use only the first two, $T_{A1}$ and $T_{A2}$.

\jgb{By contrast, the impact of higher-order terms in the kernel function on the transfer function}~\eqref{eq:resulttaylordelay} is signal-independent. It does not lead to the introduction of new timescales since the coupled time and frequency derivatives are dimensionless. We will see however in Sec.~\ref{subsec:lisafom} that treating delays \jgb{does require limiting} dimensionless factors of the type $2\pi f d$.

We can obtain useful estimates for these timescales from the leading order post-Newtonian expressions~\eqref{eq:SPAN}, valid for the dominant harmonic $h_{22}$. The leading-order radiation-reaction and amplitude timescales are:
\begin{subequations}\label{eq:timescalesN}
\begin{align}
	\Tf^{\rm N} &= \frac{1}{8} \sqrt{\frac{5}{3\nu}} \frac{G M}{c^{3}} v^{-11/2} \,, \label{eq:TfN}\\
	T_{A1}^{\rm N} &= \frac{7}{6} \frac{1}{2\pi f}\,, \quad T_{A2}^{\rm N} = \frac{\sqrt{91}}{6} \frac{1}{2\pi f} \,. \label{eq:TA1N-TA2N}
\end{align}
\end{subequations}
One can check explicitly that this expression for $\Tf$ agrees with~\eqref{eq:TfSPA}. With a simple power-law amplitude as in~\eqref{eq:ASPAN}, all higher-order amplitude timescales are also simply proportional to $1/f$ and differ only by their numerical factor. For higher harmonics $h_{\ell m}$ with $m\neq 0$, as discussed in Sec.~\ref{subsec:SPA}, $T_{f}$ acquires a factor $m/2$, and $v$ has to be replaced by $(2/m)^{1/3} v$. The amplitude of higher harmonics starts at a higher PN order~\cite{BlanchetLiving}, differing from~\eqref{eq:a22N} by a different constant and an additional scaling $v^{\kappa_{\ell m}}$ with $\kappa_{\ell m} = \ell - 2 + (\ell + m \; \mathrm{mod} \; 2)$, which changes the constant in the amplitude timescale as $7/6 \rightarrow (7-2\kappa_{\ell m})/6$ in $T_{A1}^{\rm N}$ and $\sqrt{91}/6 \rightarrow \sqrt{(7-2\kappa_{\ell m})(13-2\kappa_{\ell m})}/6$ in $T_{A2}^{\rm N}$.

We show in Fig.~\ref{fig:TfTA} the timescales $\Tf$, $T_{A1}$, $T_{A2}$ for an equal-mass and non-spinning system. In the inspiral, they follow the scalings~\eqref{eq:timescalesN} and $\Tf$ is much larger than the other two. For frequencies above the merger frequency, $\Tf$ can go through zero while the amplitude-related timescales become comparable or larger.

In the following, we will compare these signal-dependent timescales to the timescales present in the modulations and delays. In the case of signals from precessing binaries, the precession timescale decreases as the system gets closer to merger, as will be discussed in details in Sec.~\ref{subsec:sizecorrPrec} below. In the case of the response of a LISA-like detectors, the modulation and delay evolve with a fixed timescale of one year, as will be detailed in Sec.~\ref{subsec:lisafom}.

%%%%%%%%%%%%%%%%%%%%%%%%%%%%%%%%%%%%

\subsection{Quadratic term in the phase and relation to the SUA}
\label{subsec:resumquadphase}

If we restrict to the case of a pure undelayed modulation, $d=0$ and $G(f,t) = F(t)$, we can relate our result~\eqref{eq:resultdffPsiTf} \jgb{for the transfer function including up to quadratic phase terms, with treatment} of Ref.~\cite{KCY14}, where the authors extend the SPA in a formalism called the Shifted Uniform Asymptotic expansion (SUA) \SM{[look for other reference beyond KCY]}.
%The derivation includes an expansion of the modulation in Bessel functions, a study of the displacement of the stationary phase point induced by the precession, and a resummation of the result in the form of time derivatives.
The main intermediate result of~\cite{KCY14}, their equation~(34), reads exactly like~\eqref{eq:resultdffPsiTf} with $\epsilon=1$ and with the identifications $\tilde{H}_{corr}(f)\rightarrow \calT(f)$ for the transfer function, $T\rightarrow \Tf$ for the radiation-reaction timescale and $e^{-i\delta\phi} \rightarrow F$ for the modulation function (which is restricted in their framework to a phase, amplitudes being treated jointly with the amplitude of the signal). Thus, our treatment gives a straightforward rederivation of the result of~\cite{KCY14} which corresponds in our framework to the approximation~\eqref{eq:expandPsi}, where in the Fourier-domain convolution the phase of the signal is expanded to quadratic order and the amplitude is not expanded. The main difference is that the approach of~\cite{KCY14} still relies on the SPA being valid for the underlying signal.

The authors of~\cite{KCY14} then proposed a resummation scheme for~\eqref{eq:resultdffPsiTf}, using finite differences for the derivatives. Indeed, the result~\eqref{eq:resultdffPsiTf} looks like a symmetrized Taylor expansion, except for the factors $i^{p}$ and $1/p!$ instead of $1/(2p)!$. Truncating the sum at some finite order $N$, one can write (following~\cite{KCY14})
\be\label{eq:stencilresult}
	\calT_{\rm phase}(f) \simeq \sum\limits_{p = 0}^{N} \frac{(-i\epsilon\Tf^{2})^{p}}{2^{p}p!} \partial_{t}^{2p}F(\tf) \simeq \calF^{N}_{\Tf, \epsilon}[F] (\tf) \,,
\ee
with the operator $\calF_{T, \epsilon}^{N}$ defined as
\be\label{eq:stencilfresnel}
	\calF_{T, \epsilon}^{N}[F] (t) \equiv \frac{1}{2}\sum\limits_{k=0}^{N} a_{N,k}^{\epsilon} \left( F(t + kT) + F(t - k T) \right) \,,
\ee
where the complex coefficients $a_{N,k}^{\epsilon}$ are a solution of the $N+1$-dimensional linear system~\cite{KCY14}
\be\label{eq:stencilsystem}
	(-i\epsilon)^{p} (2p-1)!! = \sum\limits_{k=0}^{N} a_{N,k}^{\epsilon} k^{2p} \quad \text{for } p=0,\dots,N \,.
\ee
In~\cite{KCY14}, only the case $\epsilon=1$ was considered. The two solutions for the stencil coefficients in the two cases $\epsilon = \pm 1$ are simply related by a complex conjugation. Explicit expressions for the stencil coefficients $a_{N,k}^{\epsilon}$ are given in App.~\ref{app:stencil} for the first values of $N$.

An immediate advantage of this reformulation is its improved numerical stability. In waveform modelling applications, it can be bery hard to control high-order numerical derivatives of the modulation. Here, one simply evaluates the original smooth modulation function at shifted times.

\jgb{In practice, our quadratic-in-phase treatment will apply this formalism with [something about the range of N?]. }

%%%%%%%%%%%%%%%%%%%%%%%%%%%%%%%%%%%%

\subsection{Relation to the Fresnel transform}
\label{subsec:fresneltransform}

We can give an alternative interpretation of the \jgb{quadratic-phase} expansion~\eqref{eq:resultdffPsiTf} and of Sec.~\ref{subsec:resumquadphase} in terms of the Fresnel transform. For simplicity of notation, here we keep to the case of a pure modulation $F$ with no delays, $d=0$. We will reintroduce the delays below in Sec.~\ref{subsec:delays}. To obtain~\eqref{eq:resultdffPsiTf}, we expanded the phase exponential using~\eqref{eq:expandexp}. \jgb{Without this expansion, though}, we have
\be\label{eq:integralquadphase}
	\tilde{s}(f)	\simeq \tilde{h}(f) \int \ud t\, F(t) \int\ud f'\, e^{2i\pi f' (t-\tf)} \exp\left[ 2i\pi^{2} \epsilon{f'}^{2} \Tf^{2} \right] \,,
\ee
where we recall that $\epsilon = -\mathrm{sgn}(\ud ^{2} \Psi/\ud f^{2})$. The integral over $f'$ is a simple complex Gaussian integral \jgb{which can be carried out explicitly. The result is a complex Gaussian integral over time which we recognize as a Fresnel transform~\cite{} \SM{[C]} of the function $F$. For the Fresnel transform, we introduce the notation}
\be\label{eq:defFresnel}
	\calF_{\tau}[F](t_{0}) \equiv \frac{e^{i\frac{\pi}{4}}}{\sqrt{2\pi} \tau} \int \ud t \, \exp\left[ - \frac{i}{2} \left( \frac{t-t_{0}}{\tau} \right)^{2}\right] F(t) \,,
\ee
together with the additional notation
\be\label{eq:Fresnelsign}
	\calF_{\tau, \epsilon}[F](t_{0}) \equiv
\begin{cases}
	 \calF_{\tau}[F](t_{0}) &\text{ if } \epsilon=1 \\
	 \calF_{\tau}[F^{*}](t_{0})^{*} &\text{ if } \epsilon=-1
\end{cases}
\ee
to accomodate for the possible sign change represented by $\epsilon$. The integral~\eqref{eq:integralquadphase} then gives for the transfer function
\be\label{eq:resultFresnel}
	\calT_{\rm phase}(f) = \calF_{\Tf, \epsilon}[F](\tf) \,.
\ee

The Fresnel transform~\cite{} \SM{[C]} is an integral transform that shares some properties with the Fourier transform. Contrarily to the latter, it is localized in the sense that the part of the integral that is centered around $t_{0}$ contributes predominantly, due to the cancelling oscillations far from $t_{0}$.  The parameter $\tau$ determines how local the transform is. In the limit $\tau\rightarrow 0$, fast oscillations away from the central value $t_{0}$ will cancel out, leading to the integral taking the value $F(t_{0})$. For large values of $\tau$, by contrast, the integral~\eqref{eq:defFresnel} has an extended support. Note also that only the part of the function $F$ that is symmetric about $t_{0}$ contributes to the integral in~\eqref{eq:defFresnel}.

In our result~\eqref{eq:resultFresnel}, both the scale $\Tf$ and the central time $\tf$ are functions of the frequency $f$. Since we have seen that $\Tf$ can be interpreted as the radiation reaction timescale in the SPA regime, this means that a faster-chirping signal ($\Tf$ small) will have a Fresnel transform that is more focused, whereas a slower-chirping signal ($\Tf$ large) will have a Fresnel transform that is more extended. \jgb{The Fresnel width must then be compared} with how fast the function $F(t)$ in the integrand is varying. In Sec.~\ref{subsec:executivesummary}, we will build an estimate for the magnitude of these phase corrections by comparing of the radiation-reaction timescale to the timescale of variation of the modulation.

Thus, the previous result~\eqref{eq:stencilresult}-\eqref{eq:stencilfresnel} can be rephrased as a quadrature rule, and the stencil $\calF_{T, \epsilon}^{N}$ is an quadrature approximation of the Fresnel transfrom $\calF_{T, \epsilon}$. If one allows for polynomial integrands\footnote{Note that such integrals with a polynomial integrand are formally divergent. One can regularize them for instance by introducing a small imaginary part in $\tau$ that is sent it to $0$ at the end of the computation.} in~\eqref{eq:defFresnel}, using the stencil~\eqref{eq:stencilfresnel} amounts to building a quadrature rule for the particular choice of nodes $\tf \pm k \Tf$, which is exact (with a regularization) if $F$ is a symmetric polynomial of degree $\leq 2N$. As a verification, performing a formal Taylor expansion in time of $F(t)$ around $\tf$ in the integral~\eqref{eq:defFresnel} and integrating term by term yields back~\eqref{eq:resultdffPsiTf}.

Note however that this quadrature rule does not fall into the class of Gaussian quadratures, due to the complex character of the kernel in~\eqref{eq:defFresnel}. The induced inner product is not positive definite and some key properties are lost, such as the construction of the optimal Gaussian nodes from the zeros of the orthonormalized polynomials. For more details on quadrature rules when applied to integrals with oscillatory kernels, we refer to~\cite{} \SM{[C]}. Note that the choice of a stencil with quadrature nodes $\tf \pm k\Tf$, even if natural, is by no means unique, and other choices would have led to different stencils. The formulation of the result~\eqref{eq:resultdffPsiTf} as a Fresnel transform~\eqref{eq:resultFresnel} opens the way for future investigations of different numerical approaches to the problem.

\jgb{Going forward, the Fresnel transform provides a convenient way to non-perturbatively express the effect of the quadratic phase term on the transfer function, but practical compuations will rely on the approximations $\calF_{T, \epsilon}^{N}[F] (t)$ in \eqref{eq:stencilfresnel}.[Check that I have this right. ]}

%%%%%%%%%%%%%%%%%%%%%%%%%%%%%%%%%%%%

\subsection{Response treatment including delays}
\label{subsec:delays}

In the case of a LISA-type detector response, the presence of the delay $d(t)$ is responsible for the frequency-dependence of the kernel $G$ introduced in~\eqref{eq:defG}. \jgb{In practice we will not use the Taylor expansion of the kernel function in~\eqref{eq:resulttaylordelay}, and will instead expand the delay itself as a slowly varying constant.}

\jgb{As a first step, we consider the transfer function when we keep only the leading order waveform~\eqref{eq:leadingorderwf}, leaving off additional amplitude and phase terms in~\eqref{eq:resulttaylorA} and~\eqref{eq:resulttaylorPsi}, but treat the kernel function nonperturbatively.} The response then depends on the signal only by the time-to-frequency correspondence and reads
\begin{equation}
	\tilde{s}(f) = \tilde{h}(f) \int \ud t \, F(t) e^{-2i\pi f d(t)} \int \ud f' \, e^{2i\pi f' (t+d(t) - t_{f})} \,,
\end{equation}
with $\tf$ defined in~\eqref{eq:deftf}. This motivates a change of variable to the delayed time function $t_{d}:t \mapsto t+d(t)$. \jgb{Assuming the delay does not vary too quickly, we can} also define the reciprocal function $t_{d}^{-1}$, and a modified time-to-frequency correspondence $\tfd = t_{d}^{-1}(\tf)$, defined implicitly by
\be
	\left. (t + d(t) - t_{f})\right|_{t=t_{f}^{d}} = 0 \,.
\ee
The above integral gives then for the transfer function
\begin{align}\label{eq:delaycorrleading}
	\calT(f) &= \int \ud t \, F(t) e^{-2i\pi f d(t)} \delta(t + d(t) - t_{f}) \nn \\
	&= F(t_{f}^{d}) \frac{e^{-2i\pi f d(t_{f}^{d})}}{1+\dot{d}(t_{f}^{d})} \,.
\end{align}
\jgb{The difference with~\eqref{eq:transferlocal}, the extra factor comprising the denominator, shows the effect of using here the non-perturbative kernel function rather than just the leading order term in~\eqref{eq:expandG}.}

For the proposed LISA configuration~\cite{LISA17}, we have for the orbital delays the scaling $d_{0}\sim R/c \simeq 500s$, and for the constellation delays the scaling $d_{c}\sim L/c \simeq 8s$ (ignoring the dependence on angular factors). Since the motion of the constellation is anually periodic, with a frequency $\Omega_{0} \simeq 2 \times 10^{-7}\mathrm{rad}.s^{-1}$, we have $\dot{d}_{0} \sim \Omega_{0} R/c \simeq 10^{-4}$ and $\dot{d}_{c} \sim \Omega_{0} L/c \simeq 1.7 \times 10^{-6}$. The smallness of the dimensionless quantity $\dot{d} \ll 1$ (and of its subsequent derivatives) will allow us to treat it perturbatively with a very good approximation, and shows also that the function $t_{d}$ is univalued and that there is no ambiguity in defining the reciprocal $t_{d}^{-1}$.

By treating $\dot{d}$ as a perturbation and keeping only first-order terms, we obtain for the delayed time reciprocal function
\begin{align}
	t_{d}^{-1}(t) &\simeq t-d(t) (1-\dot{d}(t)) \,,\nn\\
	d(t_{f}^{d}) &\simeq d(t_{f}) ( 1 - \dot{d}(t_{f})) \,.
\end{align}
Now, the most relevant correction in~\eqref{eq:delaycorrleading} comes from the phase factor at high frequencies, where the factors $2\pi f d_{0}$ and $2\pi f d_{c}$ give a magnification factors reaching respectively $3.10^{3}$ and $10^{2}$ at $1\Hz$. Ignoring the other corrections, we thus arrive at the following form for the dominant delay correction in the transfer function:
\be
	\calT(f) \simeq F(t_{f})\exp\left[ -2i\pi f d(t_{f}) (1-\dot{d}(t_{f})) \right] \,.
\ee
This first correction beyond the leading order is signal-independent and affects purely the phase of the output signal.

Next, we consider the case where the quadratic phase correction is kept as well, as in Sec.~\ref{subsec:fresneltransform}. Keeping as before only the first-order terms in $\dot{d}$ (and neglecting its higher derivatives), we can write
\begin{widetext}
\begin{align}
	\tilde{s}(f) &\simeq \tilde{h}(f) \int \ud t \, F(t) e^{-2i\pi f d(t)} \int \ud f' \, \exp\left[ 2i\pi \epsilon \Tf^{2} f'^{2} + 2i\pi f' (t+d(t) - \tf) \right] \nn\\
	&\simeq \tilde{h}(f) \frac{e^{i\epsilon\frac{\pi}{4}}}{\sqrt{2\pi}\Tf} \int \ud \tau \, \frac{F(\tau - d(\tau))}{1+\dot{d}(\tau)} e^{-2i\pi f d(\tau)(1-\dot{d}(\tau))}\exp\left[ -\frac{i\epsilon}{2} \frac{(\tau - \tf)^{2}}{\Tf^{2}} \right] \,,
\end{align}
where we used a change of variable $\tau = t_{d}(t)$. We see that the result can again be expressed as a Fresnel transform.

Finally, when considering amplitude corrections as well, as in~\eqref{eq:resulttaylorA}, additional powers of $f'$ can be translated as time derivatives with respect to the variable $\tau$ after performing the change of variables. This produces the final result:
\be\label{eq:transferfinal}
	\calT(f) = \sum\limits_{k \geq 0} \frac{(-i)^{k}}{k!} (T_{Ak})^{k} \calF_{\Tf, \epsilon} \left[ \frac{\ud^{k}}{\ud \tau^{k}} \left( \frac{F(\tau - d(\tau))}{1+\dot{d}(\tau)} e^{-2i\pi f d(\tau)(1-\dot{d}(\tau))} \right) \right] (\tf) \,.
\ee
\jgb{This expression provides a summary of the perturbative treatment we will apply in later sections, with the additional understanding that the Frensel transform is treated approximately using \eqref{eq:fresnelstencil}.[Check this]}
\end{widetext}

\subsection{Summary of the formalism}\label{subsec:executivesummary}

In this Section, we gather our previous results for the convenience of the reader, and explain how we will use them in practice. For a modulation $F$ and delay $d$, so that $s(t) = F(t) h(t+d(t))$, we obtained the transfer function $\calT (f) = \tilde{s}(f)/\tilde{h}(f)$ given in~\eqref{eq:transferfinal}.

In this result, the Fresnel transform $\calF^{\epsilon}_{\Tf}$ can in turn be approximated by the stencil $\calF^{N}_{\Tf, \epsilon}$ using the formula~\eqref{eq:stencilfresnel}. The timescales $\Tf$ and $T_{Ak}$ were defined in~\eqref{eq:defTf} and~\eqref{eq:defTA}, and the generalized time-to-frequency function $t_{f}$ was given in~\eqref{eq:deftf}. The delays $d$ are present only in the LISA context, while in the context of precessing binaries we only have to consider the modulation function $F$. In practice, only the first few of the terms in the series expansion are relevant. We will investigate several orders of approximation, combining corrections from the phase, amplitude and delays, and explore which ones are relevant for a given level of accuracy. We will use symbols of the form $\{N,A,d\}$ to indicate the order of the stencil in~\eqref{eq:stencilfresnel}, the maximal order of the amplitude correction included, and the inclusion or not of the delay corrections. \jgb{[The meaning of this notation is unclear. Some more work may be needed in this subsection.]}

In the LISA context, due to the smallness of the corrections we will only go up to $k=1$ in~\eqref{eq:transferfinal}, and in computing the remaining time derivatives in~\eqref{eq:transferfinal} we will neglect the second and higher derivatives. We will also use $F(\tau - d(\tau)) \simeq (F - d \dot{F})(\tau)$. This gives concretely:
\begin{widetext}
\begin{subequations}\label{eq:summaryNAd}
\begin{align}
	\{N,A:0,d:0\}&: \; \calT(f) = \calF^{N}_{\Tf, \epsilon} \left[ F e^{-2i\pi f d} \right] (t_{f}) \,, \\
	\{N,A:1,d:0\}&: \; \calT(f) = \calF^{N}_{\Tf, \epsilon} \left[ \left(F - i T_{A1} \left( \dot{F} - 2i\pi f \dot{d}\right) \right) e^{-2i\pi f d} \right] (t_{f}) \,, \\
	\{N,A:0,d:1\}&: \; \calT(f) = \calF^{N}_{\Tf, \epsilon} \left[ \frac{F - d\dot{F}}{1+\dot{d}} e^{-2i\pi f d (1-\dot{d})} \right] (t_{f})\,, \\
	\{N,A:1,d:1\}&: \; \calT(f) = \calF^{N}_{\Tf, \epsilon} \left[ \frac{1}{1+\dot{d}}\left(  F - d\dot{F} - i T_{A1} \left( \dot{F} - 2i\pi f \dot{d} \right) \right) e^{-2i\pi f d (1 - \dot{d})} \right] (t_{f})\,.
\end{align}
\end{subequations}
\end{widetext}

In the context of precessing binaries, the delays are absent and we will go up to $k=2$ in~\eqref{eq:transferfinal}. The tranfer function at different orders of approximation will be:
\begin{subequations}\label{eq:summaryNA}
\begin{align}
	\{N,A:0\}&: \; \calT(f) = \calF^{N}_{\Tf, \epsilon} \left[ F \right] (t_{f}) \,, \\
	\{N,A:1\}&: \; \calT(f) = \calF^{N}_{\Tf, \epsilon} \left[ F - i T_{A1} \dot{F} \right] (t_{f}) \,, \\
	\{N,A:2\}&: \; \calT(f) = \calF^{N}_{\Tf, \epsilon} \left[ F - i T_{A1} \dot{F} - \frac{1}{2} (T_{A2})^{2} \ddot{F} \right] (t_{f}) \,.
\end{align}
\end{subequations}

In the following, it will be convenient to introduce error estimates built from the Taylor-like series~\eqref{eq:resulttaylorPsi},~\eqref{eq:resulttaylorA} and~\eqref{eq:resulttaylordelay}. We simply define these error estimates at a certain level of approximation as the magnitude of the first term ignored in the original Taylor series. To give these quantities a relative meaning, we divide by the leading term. Thus we define, with $G(f,t) = F(t) e^{-2i\pi f d(t)}$,
\begin{subequations}\label{eq:deffom}
\begin{align}
	\epsilon_{\Psi 2} &\equiv \frac{1}{2} \Tf^{2} \left| \frac{1}{G}\partial_{tt}G \right| \,, \\
	\epsilon_{A 1} &\equiv T_{A1} \left| \frac{1}{G} \partial_{t} G \right| \,, \\
	\epsilon_{A 2} &\equiv \frac{1}{2} T_{A2}^{2} \left| \frac{1}{G} \partial_{tt}G \right| \,, \\
	\epsilon_{d} &\equiv \frac{1}{2\pi} \left| \frac{1}{G} \partial_{tf} G \right| \,,
\end{align}
\end{subequations}
where the function $G$ and its derivatives are evaluated at $(f, t_{f})$. For $\epsilon \ll 1$, the perturbative approach applies and $\epsilon$ can be used as an estimate for the magnitude of the effect. Reaching $\epsilon \sim 1$ will indicate a breakdown of the perturbative approach.

The physical interpretation of these error measures is clear: when the modulation and delay obey the simple scaling $\partial_{t}^{n} \rightarrow \Omega^{n}$, with $\Omega_{0}$ a characteristic frequency, the phase and amplitude error estimates are simply ratios of timescales. For instance, with this scaling $\epsilon_{\Psi 2} \sim T_{f}^{2}\Omega^{2}/2$, so that the approximation will work well when the radiation-reaction timescale is shorter than the characteristic timescale of the modulation. However, as we will show in Sec.~\ref{subsec:lisafom} this simple picture will need to be refined in the presence of delays, due to the presence of additional dimensionless factors of the form $2\pi f d$, which can be larger than 1.

On top of this perturbative formalism, in the LISA case we will also develop another approach exploiting the periodicity of the modulation and delays (see Sec.~\ref{subsec:comblisa}), while in the case of precessing binaries we will use a trigonometric polynomial approach to represent the merger-ringdown part of the signal (see Sec.~\ref{subsec:trigopoly})

%%%%%%%%%%%%%%%%%%%%%%%%%%%%%%%%%%%%
%%%%%%%%%%%%%%%%%%%%%%%%%%%%%%%%%%%%

\section{Application to the response of LISA-type detectors}
\label{sec:LISA}

In this Section, we apply the formalism of Sec.~\ref{sec:formalism} to the Fourier-domain response of a LISA-like detector, and assess the accuracy of our approach at various levels of approximation. We also propose an alternative approach for slowly-chipring signals. 

%%%%%%%%%%%%%%%%%%%%%%%%%%%%%%%%%%%%

\subsection{The response model}
\label{subsec:modelLISA}

We begin by detailing the model that we use for the response of a LISA-like detector, together with the assumptions used and their limitations. Since our aim is to assess the accuracy of our direct Fourier-domain treatment of the response, we can focus on the gravitational-wave contribution to the basic single-linkobservables. We therefore use a somewhat simplified model for the time-domain response, ignoring corrections that are crucial from the point of view of noise cancellations, keeping in mind that the response model can be enriched later without affecting the conclusions of the present analysis.

The frequency-shift response for a single link~\eqref{eq:yslr} was derived in Ref.~\cite{EW75} (see also~\cite{RCP04, Finn08, Cornish09} for more modern derivations). Several assumptions enter the result as written in~\eqref{eq:yslr}: (i) effects of the order $v/c$ are neglected, including for instance the special relativistic Doppler effect created by the relative speeds of the spacecraft on their orbits (ii) the propagation is assumed to take place in a flat spacetime, perturbed only by the gravitational wave; thus the gravitational redshift as well as the deflection of light created by the gravitational potential of the Sun is ignored (iii) all geometric factors are evaluated at a single time, whereas one should consider the beam as propagating from the position of the first spacecraft at the time of emission to the position of the second scapecraft at the time of reception, leading to a pointing-ahead effect.

Additionally, we limit ourselves to a rigid model for the orbits of the constellation, namely we assume that the constellation remains in an equilateral configuration with fixed armlengths. These simplified orbits neglect (iv) effects of order $e^{2}$ from the eccentricity of the individual Keplerian orbits, (v) the effect of gravitational perturbations coming from other celestial bodies, such as the Earth, the quadrupole of the Sun, and the other planets. 

Note that although we can neglect all the effects (i)-(v) for our present study, keeping track of these corrections is crucial for the purpose of laser noise cancellations, and led to the development of new generations of TDI observables~\cite{} \SM{[C]}. This simplified response has been abundantly used in previous studies (see e.g.~\cite{} \SM{[C]}).

It is natural to split the response~\eqref{eq:yslr} into two steps: first the orbital delay related to the orbit around the Sun of the whole constellation, and then the constellation response. The baselines for the delays are indeed very different in the two cases. Geometrical projection factors aside, we have for the orbit around the Sun $R=1\text{au}=1.5\times 10^{8} \text{km}$, while the detector armlength is $L=2.5\times 10^{6}\text{km}$ (in the configuration proposed in~\cite{LISA17}). It is natural to define two transfer frequencies for the two relevant length scales for the delays, defined such that a wavelength fits within this length scale, i.e. $2\pi f d = 1$. This gives
\begin{subequations}\label{eq:transferfrequencies}
\begin{align}
	f_{R} &= 3.2\times10^{-4}\Hz \,,\\
	f_{L} &= 1.9\times 10^{-2}\Hz \,.
\end{align}
\end{subequations}
The LISA response behaves qualitatively differently on the frequency bands separated by these transfer frequencies. \jgb{[\bf I don't understand the meaning or intent of the previous sentence]}

\jgb{The first stage of the response, the orbital delay, consists simply in applying the varying time delay to bring the wavefront sampling point from the SSB reference to the center} of the LISA triangular constellation, common to all $y_{slr}$ observables. For $h^{\rm TT}$ the transverse-traceless gravitational waveform in matrix form, we write this orbital time delay as
\be\label{eq:defresponse0}
	h_{0}^{\rm TT} (t) = h^{\rm TT}(t-\hatk\cdot p_{0}) \,,
\ee
with $p_{0}$ the position of the constellation center, which follows the Earth orbit around the Sun. The second \jgb{stage of the response calculation comprises the remaining, constellation-centered response. For the single-link contribution to the response, for the laser link from spacecraft $s$ to spacecraft $r$ along a path in direction $n_l=$ [{\bf Check that I got the notation right}], we write}
\begin{align}\label{eq:defresponseL}
	y_{slr} &= \frac{1}{2} \frac{1}{1 - \hatk\cdot n_{l}} \nn\\
	& \cdot n_{l}\cdot \left[ h_{0}^{\rm TT}(t - L - \hatk\cdot p^{L}_{s}) - h_{0}^{\rm TT}(t - \hatk\cdot p^{L}_{r}) \right] \cdot n_{l}\,,
\end{align}
where we reference the positions of the spacecraft relative to the center of the constellation, $p^{L}_{A} \equiv p_{A} - p_{0}$.

As described in App.~\ref{app:notation} We will decompose the full signal in the contributions of the individual spin-weighted spherical modes $h_{\ell m}$, whose Fourier transforms are assumed to have a smooth amplitude and phase. First, we define the matrices $P_{+},P_{\times}$ such that, in the sense of matrices,
\be
	h^{\rm TT} = h_{+}P_{+} + h_{\times}P_{\times} \,.
\ee
We focus only on positive frequencies. Assuming that the approximation~\eqref{eq:zeronegativef} applies, we consider a single mode contribution, either $h=h_{\ell m}$ for $m>0$ or $h=h_{\ell m}^{*}$ for $m<0$ (following~\eqref{eq:notationhsignm}). For each mode we define a complex matrix $P_{\ell m}$ incorporating the spin-weighted spherical harmonic constant factor as
\be
	P_{\ell m} = 
	\begin{cases} 
	\frac{1}{2} {}_{-2}Y_{\ell m} \left( P_{+} + i P_{\times} \right) \text{ for } m>0\,,\\
	\frac{1}{2} {}_{-2}Y_{\ell m}^{*} \left( P_{+} - i P_{\times} \right) \text{ for } m<0\,.
	\end{cases}
\ee

We now turn to the transformation of~\eqref{eq:defresponse0} and~\eqref{eq:defresponseL} to the Fourier domain. Applying a pure delay as in~\eqref{eq:defresponse0} translates into
\be\label{eq:G0}
	G_{0}(f, t) = e^{-2i\pi f d_{0}(t)} \,,
\ee
with $d_{0} = -\hatk \cdot p_{0}$ the delay associated to the orbit around the Sun. For the leading-order response~\eqref{eq:transferlocal}, this gives a Fourier-domain transfer function common to all modes, that is a pure phase factor, proportional to the frequency but also $t_{f}$-dependent:
\be\label{eq:transfer0local}
	\calT_{0}^{\rm local}(f) = G_{0}(f, \tf)\,.
\ee
If $(\lambda, \beta)$ are the ecliptic longitude and latitude of the source in the sky, and if the orbital phase is set by convention to $0$ at $t=0$, the orbital delay has the simple expression
\be\label{eq:delay0}
	d_{0}(t) = -R \cos\beta \cos\left(\Omega_{0}t - \lambda\right)\,.
\ee

For the constellation response~\eqref{eq:defresponseL}, we can \jgb{separately treat the orbital delay $d_r$ and the constellation delay at spacecraft $s$, $d_{s,L}$}, by writing
\begin{align}\label{eq:decomposeGslr}
	F_{slr}^{L}(t) &= \frac{1}{2} \frac{1}{1 - \hatk\cdot n_{l}(t)} n_{l}(t) \cdot P_{\ell m} \cdot n_{l} (t) \,,\nn\\
	d_{s,L}(t) &= -L - k\cdot p_{s}^{L}(t) \,, \quad d_{r}(t) = - k\cdot p_{r}^{L}(t) \,,\nn\\
	G_{slr}^{L}(f,t) &=  F_{slr}^{L}(t) \left( e^{-2i\pi f d_{s,L}(t)} - e^{-2i\pi f d_{r}(t)} \right) \,.
\end{align}
\jgb{Since we also assume the {\bf[right]} } rigid approximation for the constellation, where the armlengths are fixed, a particular simplification occurs when combining these individual delays, thanks to the relation $p^{L}_{r} - p^{L}_{s} =  L n_{l}$:
\begin{align}\label{eq:GslrL}
	G_{slr}^{L}(f,t) &= \frac{i \pi f L}{2} \sinc \left[ \pi f L\left(1-\hatk\cdot n_{l} \right) \right] \nn\\
	& \quad \cdot \exp\left[ i \pi f \left( L + \hatk\cdot \left( p_{1}^{L} + p_{2}^{L} \right) \right) \right]  n_{l} \cdot P_{\ell m} \cdot n_{l} \,,
\end{align}
with all time-dependent vectors evaluated at $t$. In the local approximation~\eqref{eq:transferlocal}, the Fourier-domain transfer function then reads
\begin{align}\label{eq:transferLlocal}
	\calT_{slr}^{L, \mathrm{local}}(f) &= G_{slr}^{L}(f, \tf) \,.
\end{align}
This leading-order result~\eqref{eq:transfer0local}-\eqref{eq:transferLlocal} is familiar, as it was derived in the past taking the point of view of monochromatic signals (see e.g.~\cite{} \SM{[C]}) \SM{[rewrite that]}.

Note that, if the corrections of Sec.~\ref{subsec:delays}, \jgb{for non-negligible $\dot d$}, are included for the constellation delays, the transfer function will not have this simple form anymore, as $\dot{d}$ will have a different velocity-dependent expression for the sending and receiving spacecraft. One must \jgb{separately handle $d_{s,L}$ and $d_{r}$ in~\eqref{eq:decomposeGslr} to compute the corrections.}

The \jgb{orbital} response~\eqref{eq:transfer0local}-\eqref{eq:delay0} takes a simple analytic form, but the phase contribution of this delay is significant across most of the frequency band and can be large for $f \gg f_{R}$.

The \jgb{constellation} response~\eqref{eq:GslrL}-\eqref{eq:transferLlocal} can be interpreted as the Fourier-domain translation of a discrete derivative taken on the waveform. The \jgb{leading factor in~\eqref{eq:GslrL}} shows that the amplitude of the response is proportional to $f$ in the low-frequency limit $f\ll f_{L}$, where the other factors are essentially unity. For $f\gtrsim f_{L}$, the $\sinc$ and the phase of the exponential generate additional structure in the response, including zero-crossings when the projected armlength \jgb{is an integer number of wavelengths}. From~\eqref{eq:transferLlocal}, an expansion for small $f\ll f_{L}$ yields back a Fourier-domain analog of the low-frequency approximation of the response~\cite{Cutler97, RCP04}, which \jgb{is equivalent to} having two LIGO-type interferometers turned by $\pi/4$ and set in motion. See Fig.~\ref{} and ~\ref{} \SM{[R]} for examples of LISA transfer functions.

\jgb{For analysis of the response we need a concrete set of gravitational waveforms. We will} use the PhenomD model~\cite{Khan+15,Husa+15}, which provides Fourier-domain inspiral-merger-ringdown waveforms for aligned spins.

%%%%%%%%%%%%%%%%%%%%%%%%%%%%%%%%%%%%

\subsection{Estimates for the magnitude of higher-order corrections}
\label{subsec:lisafom}

\begin{figure}
  \centering
  \includegraphics[width=.9\linewidth]{plots/lisafom_Psi_McDeltat.pdf}
  \caption{\SM{[To be checked and redone]} Contour levels for the analytical estimate of the error measure $\epsilon_{\Psi}$ at the starting frequency, as a function of chirp mass $\Mchirp$ and time to coalescence $\Delta t$. Blue corresponds to the orbital response and red to the constellation response. The colored shaded areas indicate regions where $\epsilon_{\Psi} \geq 1$, where the perturbative formalism is expected to break down. In the gray area, $f_{\rm start}$ given in~\eqref{eq:fstartN} is lower than the lowest in-band frequency $f_{\rm min} = 10^{-5}\Hz$, so that the signal starts at $f_{\rm min}$ and $\epsilon_{\Psi}$ becomes independent of $\Delta t$.\jgb{\bf[Maybe not shade the ``grey area'' since it looks excluded, instead can call it ``the region to the right of the black line.'']} }
  \label{fig:lisafomPsiMcDeltat}
\end{figure}

\begin{figure*}
  \centering
%  \includegraphics[width=.32\linewidth]{plots/lisafom_Psiorb.pdf}
%  \includegraphics[width=.32\linewidth]{plots/lisafom_A1orb.pdf}
%  \includegraphics[width=.32\linewidth]{plots/lisafom_dorb.pdf}
%  %
%  \includegraphics[width=.32\linewidth]{plots/lisafom_Psiconst.pdf}
%  \includegraphics[width=.32\linewidth]{plots/lisafom_A1const.pdf}
%  \includegraphics[width=.32\linewidth]{plots/lisafom_dconst.pdf}
  \includegraphics[width=.98\linewidth]{plots/lisafom_py.pdf}
  \caption{Error estimates as defined in~\eqref{eq:deffom}, for an equal-mass, non-spinning system and for total masses $M=10^{7} \Msol$, $10^{4} \Msol$ and $10^{2} \Msol$. The top row corresponds to the orbital delay part of the response~\eqref{eq:defresponse0}, and the lower row shows the LISA-centered constellation response~\eqref{eq:defresponseL}. The error measures $\epsilon_{\Psi}$ for the phase corrections, $\epsilon_{A1}$ for the amplitude, $\epsilon_{d}$ for the delay are shown from left to right. The central line and interval are the mean and $1\sigma$ standard deviation of the logarithm of $\epsilon$ computed with numerical derivatives over 400 random values for the position in the sky, inclination and polarization. The starting frequency is set by an observation time of $\Delta t = 10 \text{yrs}$ before merger. We overlay in dashed the analytical estimates obtained from from~\eqref{eq:timescalesNfstart} and~\eqref{eq:estimatederivorb}-\eqref{eq:estimatederivconst}.}
  \label{fig:fomLISA}
\end{figure*}

Using the approximate error measures $\epsilon$ introduced in~\eqref{eq:deffom}, we will now estimate, for each type of correction (phase, amplitude, delay), the size of errors in the transfer function.

To obtain an order-of-magnitude estimate for the error measures $\epsilon$~\eqref{eq:deffom}, we will use the Newtonian-order expressions~\eqref{eq:timescalesN} for the signal-dependent timescales $\Tf$ and $T_{A1}$. Higher-order amplitude terms beyond the first one in~\eqref{eq:resulttaylorA}, as well as phase terms beyond the second derivative in~\eqref{eq:expandPsi}, will turn out to be completely negligible in the LISA case and we will ignore them in the following. It is useful to separate the orbital response~\eqref{eq:transfer0local} and the constellation response~\eqref{eq:transferLlocal}, as the different baseline of the delays (orbital radius $R$ or armlength $L$) as well as the presence of a time-varying prefactor $F(t)$ both affect the result. When estimating the magnitude of the relevant derivatives of $G$, we must also take into account dimensionless delay factors of the type $2\pi f d$.

We start with the orbital response, which takes the form of a pure delay $G_{0}(f, t) = e^{-2 i \pi f d_{0}(t)}$ \jgb{\bf[Why subscripted with zero not O or R?]}, and obtain
\begin{subequations}
\begin{align}
	\frac{1}{G_{0}} \partial_{t} G_{0} &= -2i\pi f \dot{d}_{0}\,,\\
	\frac{1}{G_{0}} \partial_{tt} G_{0} &= -2i\pi f \ddot{d}_{0} - 4\pi^{2} f^{2} \dot{d}_{0}^{2} \,,\\
	\frac{1}{G_{0}} \partial_{tf} G_{0} &= -2 i \pi \dot{d}_{0} - 4\pi^{2} f d_{0} \dot{d}_{0} \,.
\end{align}
\end{subequations}

Since the one-arm constellation response~\eqref{eq:decomposeGslr}-\eqref{eq:transferLlocal} is analogous to a discrete time derivative of the signal, it is appropriate to keep explicit an overall factor $f$ reflecting this structure. Hence we write symbolically $G_{L}(f,t) \sim f F(t) e^{-2i \pi f d_{L}(t)}$, where $d_{L}$ represents a delay term and $F(t)$ represents the rest of the geometric factors in~\eqref{eq:transferLlocal}, \jgb{for which we momentarily} ignore the $f$-dependence. This gives
\begin{subequations}
\begin{align}
	\frac{1}{G_{L}} \partial_{t} G_{L} &\sim -2i\pi f \dot{d}_{c} + \frac{\dot{F}}{F}\,,\\
	\frac{1}{G_{L}} \partial_{tt} G_{L} &\sim -2i\pi f \ddot{d}_{c} - 4\pi^{2} f^{2} \dot{d}_{c}^{2} - 2i\pi f \dot{d}_{c} \frac{\dot{F}}{F} - \frac{\dot{F}^{2}}{F^{2}} + \frac{\ddot{F}}{F} \,,\\
	\frac{1}{G_{L}} \partial_{tf} G_{L} &\sim -4 i \pi \dot{d}_{c} - 2i\pi d_{c} \frac{\dot{F}}{F} - 4\pi^{2} f d_{c} \dot{d}_{c} + \frac{1}{f}\frac{\dot{F}}{F}\,.
\end{align}
\end{subequations}

In order to obtain simple scalings for the error estimates, we will make the replacements $\partial_{t}^{n} d \sim \Omega_{0}^{n} d$ as well as $\partial_{t}^{n} F \sim \Omega_{0}^{n} F$. These scalings are only approximate as different orientation angles can lead to significant variations. To represent the average of the geometric projection factor of the gravitational wave propagation vector on the plane of the orbit, we will assume $d_{0} \sim R_{\rm eff} \equiv 2R/\pi $. For the constellation delays, we simply take $d_{L} \sim L$ as there projection effects can lead to variations in both ways. The resulting estimates for the magnitude of these derivatives are
\begin{subequations}\label{eq:estimatederivorb}
\begin{align}
	\left| \frac{1}{G_{0}} \partial_{t} G_{0} \right| &\sim 2 \pi f \Omega_{0} R_{\rm eff}\,,\\
	\left| \frac{1}{G_{0}} \partial_{tt} G_{0} \right| &\sim \text{max} \left[ 2 \pi f \Omega_{0}^{2} R_{\rm eff} , 4\pi^{2} f^{2} \Omega_{0}^{2} R_{\rm eff}^{2}\right] \,,\\
	\left| \frac{1}{G_{0}} \partial_{tf} G_{0} \right| &\sim \text{max} \left[ 2 \pi  \Omega_{0} R_{\rm eff}, 4 \pi^{2} f \Omega_{0} R_{\rm eff}^{2} \right] \,,
\end{align}
\end{subequations}
for the orbital response and
\begin{subequations}\label{eq:estimatederivconst}
\begin{align}
	\left| \frac{1}{G_{L}} \partial_{t} G_{L} \right| &\sim \text{max} \left[ \Omega_{0}, 2 \pi f \Omega_{0} L \right] \,,\\
	\left| \frac{1}{G_{L}} \partial_{tt} G_{L} \right| &\sim \text{max} \left[ \Omega_{0}^{2}, 4 \pi f \Omega_{0}^{2} L, 4\pi^{2} f^{2} \Omega_{0}^{2} L^{2} \right] \,,\\
	\left| \frac{1}{G_{L}} \partial_{tf} G_{L} \right| &\sim \text{max} \left[ \Omega_{0}/f, 6 \pi \Omega_{0} L, 4\pi^{2} f \Omega_{0}^{2} L^{2} \right] \,,
\end{align}
\end{subequations}
for the constellation response. In the presence of different terms, we simply take the maximum of their norms.

\jgb{\bf[This paragraph is confusing.  Can we point concretely to these factors somewhere, (e.g. in the expansion of the exponential.xxx)? Also it seems to say that we can't do what we just did. Should discuss.]} An important point in the above is the presence of delay factors in the forms of powers of $2\pi f d$. Thus, the magnitude of the corrections cannot be evaluated with the simple replacement $\partial_{t}^{n} \rightarrow \Omega_{0}^{n}$, and the frequency dependency of the error measures $\epsilon$ will depend on the frequency being above or below the transfer frequencies $f_{R}$ and $f_{L}$ defined in~\eqref{eq:transferfrequencies}. The error measure $\epsilon_{d}$ does not depend on signal-dependent timescales and can be directly read off the estimates above, with $\epsilon_{d} = |\partial_{tf}G/(2\pi G)|$. For the error measures $\epsilon_{\Psi}$ and $\epsilon_{A1}$, we can combine the above derivatives with the Newtonian timescales given in~\eqref{eq:timescalesN} for the inspiral phase of the signal.

The starting frequency of the signal will play an important role. As a function of the time remaining before merger $\Delta t$, from the Newtonian relation~\eqref{eq:omegaphiN} we have (using the chirp mass $\Mchirp = M \nu^{3/5}$):
\be\label{eq:fstartN}
	f_{\rm start} = 1.75\times 10^{-5} \Hz \left( \frac{\Mchirp}{10^{6}M_{\odot}} \right)^{-5/8} \left( \frac{\Delta t}{10 \yr} \right)^{-3/8} \,.
\ee
This gives in turn for the Newtonian timescales~\eqref{eq:timescalesN}
\begin{subequations}\label{eq:timescalesNfstart}
\begin{align}
	\Tf^{\rm N} &= 8.78\times10^{-2}\yr \left( \frac{\Mchirp}{10^{6}M_{\odot}} \right)^{5/16} \left( \frac{\Delta t}{10 \yr} \right)^{11/16} \,, \\
	T_{A1}^{\rm N} &= 3.37\times10^{-4}\yr \left( \frac{\Mchirp}{10^{6}M_{\odot}} \right)^{5/8} \left( \frac{\Delta t}{10 \yr} \right)^{3/8} \,.
\end{align}
\end{subequations}
Note that if we \jgb{think of LISA as effectively insensitive below some minimal frequency $f_{\rm min}$, then for sufficiently high $\Mchirp$ this point of entry in the sensitive band will mark the beginning of the signal, obviating the relevance of $f_{\rm start}$}. For $f_{\rm min} = 10^{-5}\Hz$ and $\Delta t \leq 10\yr$, this is the case for $\Mchirp \geq 2.45\times 10^{6}\Msol $. Thus, our final analytical estimates for the $\epsilon$ error measures are built by inserting~\eqref{eq:timescalesNfstart} and~\eqref{eq:estimatederivorb}-\eqref{eq:estimatederivconst} in~\eqref{eq:deffom}, while ensuring the frequency cut $10^{-5}\Hz \leq f \leq 1\Hz$.


\jgb{It is useful to distinguish} between what we will call merging binaries, systems close to coalescence that will merge during the LISA mission lifetime or a few years later, and slowly-chirping binaries still in the deep inspiral phase, of which we observe only a small snapshot in Fourier domain as they do not sweep to the end of the frequency band. The massive black hole binaries (MBH) that will be observed by LISA~\cite{LISA17} fall within the first category, with their merger in band, while the proposed population of stellar-origin black hole binaries (SOBH)~\cite{Sesana16}, with masses comparable to the LIGO/Virgo detections, will comprise both merging binaries, i.e. exiting the LISA band towards larger frequencies during observations, and slowly-chirping binaries hundereds or thousands of years away from merger. Galactic binaries (GB) fall entirely within the second category, and are very close to monochromatic over the lifetime of LISA~\cite{LISA17} \SM{[C: complete]}.

We turn first to the slowly-chirping binaries. For these systems, we will focus on $\epsilon_{\Psi}$, which will be the most important error measure. Fig.~\ref{fig:lisafomPsiMcDeltat} shows contour levels for the value of our simple analytical estimate for $\epsilon_{\Psi}$ at the beginning of the observations, as a function of the chirp mass and time to merger, for both the orbital and the constellation response. The limit $\epsilon_{\Psi} = 1$ is used to single out areas where the perturbative treatment of Sec.~\ref{sec:formalism} is expected to break down, although the precise location of this boundary will vary, depending on the accuracy level required and on the orientation angles. We also single out the high-mass region where the start of the signal is set by its entry \jgb{into the sensitive band, at $f_{\rm min}$ making the error measure independent of time to merger}. We find that, for binaries in the LIGO/Virgo mass range $10^{1}-10^{2} \Msol$, we reach the limit $\epsilon_{\Psi}=1$ first for the orbital response, for time-to-merger values within the expected observed range~\cite{Sesana16}. We will introduce in Sec~\ref{subsec:comblisa} an alternative approach to deal with those signals. For completeness, we also show in Fig.~\ref{fig:lisafomPsiMcDeltat} the result for intermediate and massive black hole systems, although we expect to observe mainly merging systems for this mass range. 

We now turn to the case of merging binaries. For these we compute the three error measures $\epsilon_{\Psi}$, $\epsilon_{A1}$ and $\epsilon_{d}$. To go beyond the crude analytical estimates built from~\eqref{eq:timescalesNfstart} and~\eqref{eq:estimatederivorb}-\eqref{eq:estimatederivconst}, we use numerical derivatives for the timescales~\eqref{eq:defTf}-\eqref{eq:defTA} and for the derivatives in~\eqref{eq:deffom}. We summarize the results in Fig.~\ref{fig:fomLISA}, both for the orbital delay and the constellation response, considering equal-mass systems with total masses of $M=10^{7} \Msol$, $10^{4} \Msol$ and $10^{2} \Msol$ with a starting frequency corresponding to $\Delta t =10 \yr$ of observation. Since individual signals can show significant variation depending on the angular parameters, for the full computation with numerical derivatives we show a geometric average of $\epsilon$ over the sky position, inclination and polarization, together with $\pm 1\sigma$ geometric standard deviation. Here and in the following, for completeness we show results on the frequency band from $10^{-5}\Hz$ to $1\Hz$, while the signals are expected to have little signal-to-noise ratio outside of the band from $10^{-4}\Hz$ to $10^{-1}\Hz$, which is sometimes taken as a reference~\cite{LISA17}.

For $\epsilon_{\Psi}$, as expected from~\eqref{eq:TfN} we find a steep rise towards lower frequencies, which shows that considering merging binaries with a limited time to coalescence is crucial here. \jgb{We note differences in} behaviour between the orbital response, where for a fixed mass ratio the initial \jgb{(10yr before merger)} value of $\epsilon_{\Psi}$ grows towards lower masses, and the constellation response, where \jgb{initial} $\epsilon_{\Psi}$ grows towards higher masses. \jgb{Combining both parts of the response, for equal-mass mergers with $10^2<M<10^7$,} the error measure remains below $\epsilon_{\Psi} \sim 0.1$, where the corrections are still well manageable by the perturbative formalism (as we will show in Sec.~\ref{subsec:errorsLISA}).

For $\epsilon_{A1}$, we find that the structure in the waveform close to merger does have a noticable but limited impact on the error measure. For the orbital response, the error measure remains small for all masses. For the constellation response, we find that $\epsilon_{A1}$ can grow up to $\sim 0.01$ at the lower end of the frequency band for high masses.

The $\epsilon_{d}$ is signal-independent, and follows a quite different behaviour for the orbital and constellation responses. For the orbital response, $\epsilon_{d_{0}}$ grows for higher frequencies, reaching the $\sim 0.1$ at around $1\Hz$. For the constellation response, $\epsilon_{d_{c}}$ is mainly important at lower frequencies, reaching $\sim 0.01$ at around $10^{-5}\Hz$.

Overall, we find that for merging binaries the perturbative approach presented in Sec.~\ref{sec:formalism} will be applicable, as the smallness of the annual frequency $f_{0} \simeq 3.1\times10^{-8}\mathrm{Hz}$, compared to the timescales relevant for the signal, ensures a good separation of timescales on the whole LISA frequency band. The most relevant higher-order corrections are expected to be the phase corrections close to the starting frequency of the signal, and the delay correction at high frequencies for the orbital response. For slowly chirping binaries, we have identified regions in the parameter space where the perturbative formalism of Sec.~\ref{sec:formalism} will break down, requiring an alternative approach presented below in Sec.~\ref{subsec:comblisa}.

%%%%%%%%%%%%%%%%%%%%%%%%%%%%%%%%%%%%

\subsection{Errors in the Fourier-domain response for merging binaries}
\label{subsec:errorsLISA}

\SM{[We have to say why we take a non-spinning, equal-mass system - say why the method should also work for precessing signals]}

\SM{[Write explicitly what the orders A1, d1 mean]}

\SM{[argument why we do not compute mismatches]}

\SM{[stress that binaries very far from merger could have arbitrarily large Psi2 corrections]}

\SM{[say that we take a large number of samples to eliminate spline interpolation errors]}

Having at hand estimates for the relevance of higher-order corrections in the LISA response, we now demonstrate that including these higher-order corrections does indeed reduce the reconstruction error of the Fourier-domain transfer function. We will consider an equal-mass and non-spinning system, with only two values for the total mass, $M=10^{7} \Msol$ and $M=10^{2} \Msol$. For this example, we will also focus on the orbital response and on a single-arm constellation response $y_{132}$, as TDI combinations will appear (within the approximations listed in Sec.~\ref{subsec:modelLISA}) as linear combinations of these basic observables. The orientation angles chosen for these examples are $[\lambda,\beta,\iota,\psi] = [\pi/4,\pi/3,\pi/3,0]$.

On one hand, we perform a numerical inverse Fourier transform (IFFT) of the Fourier-domain PhenomD waveform, process the signal through the time-domain response~\eqref{eq:defresponse0} or~\eqref{eq:defresponseL}, and perform a numerical Fourier transform (FFT). This gives us the target Fourier-domain waveform, considered to be exact but for possible numerical artifacts (most notably, oscillations induced by the necessary tapering of the signal). On the other hand, we process the Fourier-domain signal through the response summarized in Sec.~\ref{subsec:executivesummary}, including various higher-order corrections. We then compare the output of the two procedures.

We show the results for the orbital response in Figs.~\ref{fig:LISAerrorM1e7orb} and~\ref{fig:LISAerrorM1e2orb} in amplitude and phase form. In both cases, in accordance with~\eqref{eq:transfer0local}, the transfer function is essentially a phase, and contains many more cycles in the low-mass case in the higher frequency band. We find that including higher-order corrections does reduce the reconstruction errors, but with the distinctive feature that including only one of the amplitude and delay corrections can make the error actually worse, while including both does improve the accuracy. In the high-mass case, we also find for all orders of approximation an increase in the errors at the very end. This is likely to be due to the fact that the errors in the transfer function are to be interpreted in the relative sense, and the amplitude of the signal is decaying rapidly \SM{[ref to amplitude plot somewhere]} in this post-merger region. Although we did not identify the precise cause of this rise in errors, one possible explanation would be tapering-induced oscillations in the numerical Fourier transforms.

For the constellation response, we show the results in Figs.~\ref{fig:LISAerrorM1e7const} and~\ref{fig:LISAerrorM1e2const}, using directly the real and imaginary part of the transfer function. We find similar features as for the orbital response: including higher-corrections reduces the errors, but in the high-mass case the amplitude and delay corrections should be added together. We also find the same growth of errors at the very end of the signal in the high-mass case, in the region where the amplitude of the signal decays.

\begin{figure}
  \centering
%  \includegraphics[width=.8\linewidth]{plots/LISAtransferM1e7dOamp.pdf}
%  \hspace{0.1cm}
%  \includegraphics[width=.8\linewidth]{plots/LISAtransferM1e7dOphase.pdf}
%  \hspace{0.1cm}
%  \includegraphics[width=.8\linewidth]{plots/LISAerrorM1e7dO.pdf}
  \includegraphics[width=.98\linewidth]{plots/LISAtransfererrorM1e7dO_py.pdf}
  \caption{Error in the transfer function for the orbital delay $d_{0}$, for $M=10^{7} \Msol$.}
  \label{fig:LISAerrorM1e7orb}
\end{figure}

\begin{figure}
  \centering
%  \includegraphics[width=.8\linewidth]{plots/LISAtransferM1e2dOamp.pdf}
%  \hspace{0.1cm}
%  \includegraphics[width=.8\linewidth]{plots/LISAtransferM1e2dOphase.pdf}
%  \hspace{0.1cm}
%  \includegraphics[width=.8\linewidth]{plots/LISAerrorM1e2dO.pdf}
  \includegraphics[width=.98\linewidth]{plots/LISAtransfererrorM1e2dO_py.pdf}
  \caption{Transfer function and reconstruction error for the orbital delay $d_{0}$, for $M=10^{2} \Msol$.}
  \label{fig:LISAerrorM1e2orb}
\end{figure}

\begin{figure}
  \centering
  %\includegraphics[width=.98\linewidth]{plots/LISAtransferM1e7y12LReIm.pdf}
  %\includegraphics[width=.98\linewidth]{plots/LISAerrorM1e7y12L.pdf}
  \includegraphics[width=.98\linewidth]{plots/LISAtransfererrorM1e7y12L_py.pdf}
  \caption{Error in the transfer function for the basic observable $y_{12}$, for $M=10^{7} \Msol$.}
  \label{fig:LISAerrorM1e7const}
\end{figure}

\begin{figure}
  \centering
  %\includegraphics[width=.98\linewidth]{plots/LISAtransferM1e2y12LReIm.pdf}
  %
  %\includegraphics[width=.98\linewidth]{plots/LISAerrorM1e2y12L.pdf}
  \includegraphics[width=.98\linewidth]{plots/LISAtransfererrorM1e2y12L_py.pdf}
  \caption{Transfer function and reconstruction error for the basic observable $y_{12}$, for $M=10^{2} \Msol$. \SM{[check $\overline{\calT}$ is defined]}}
  \label{fig:LISAerrorM1e2const}
\end{figure}

%%%%%%%%%%%%%%%%%%%%%%%%%%%%%%%%%%%%

\subsection{Slowly chirping binaries and direct approach for periodic modulations and delays}\label{subsec:comblisa}

\begin{figure}
  \centering
  \includegraphics[width=.98\linewidth]{plots/LISACombExtent_py.pdf}
  \caption{Extent of the Fourier-domain comb entering the convolution~\eqref{eq:transferdiscreteconvolution}, determined from the criterion~\eqref{eq:criteriontruncationcomb} with truncation levels $\eta=10^{-12}$, $10^{-6}$, $10^{-3}$. Blue corresponds to the orbital delay comb coefficients $c_{n}^{0}$ given by~\eqref{eq:cn0}, and red to the constellation comb $c_{n}^{L}$ for the response~\eqref{eq:GslrL}. The result is shown after averaging over 100 random orientation angles.}
  \label{fig:lisacombextent}
\end{figure}

\begin{figure}
  \centering
%  \includegraphics[width=.8\linewidth]{plots/LISAtransferSOBHepsPsi1d0ampcorr.pdf}
%  \hspace{0.1cm}
%  \includegraphics[width=.8\linewidth]{plots/LISAtransferSOBHepsPsi1d0phasecorr.pdf}
%  \hspace{0.1cm}
%  \includegraphics[width=.8\linewidth]{plots/LISAerrorSOBHepsPsi1d0.pdf}
  \includegraphics[width=.98\linewidth]{plots/LISAtransfererrorSOBHepsPsi1d0corr_py.pdf}
  \caption{Error in the transfer function for the orbital delay $d_{0}$, for a SOBH with $\epsilon_{\Psi}\sim 1$.}
  \label{fig:}
\end{figure}

\begin{figure}
  \centering
%  \includegraphics[width=.8\linewidth]{plots/LISAtransferSOBHepsPsi10d0ampcorr.pdf}
%  \hspace{0.1cm}
%  \includegraphics[width=.8\linewidth]{plots/LISAtransferSOBHepsPsi10d0phasecorr.pdf}
%  \hspace{0.1cm}
%  \includegraphics[width=.8\linewidth]{plots/LISAerrorSOBHepsPsi10d0.pdf}
  \includegraphics[width=.98\linewidth]{plots/LISAtransfererrorSOBHepsPsi10d0corr_py.pdf}
  \caption{Error in the transfer function for the orbital delay $d_{0}$, for a SOBH with $\epsilon_{\Psi}\sim 10$.}
  \label{fig:}
\end{figure}

\begin{figure}
  \centering
%  \includegraphics[width=.8\linewidth]{plots/LISAtransferSOBHepsPsi1y12LReIm.pdf}
%  \hspace{0.1cm}
%  \includegraphics[width=.8\linewidth]{plots/LISAerrorSOBHepsPsi1y12L.pdf}
  \includegraphics[width=.98\linewidth]{plots/LISAtransfererrorSOBHepsPsi1y12L_py.pdf}
  \caption{Error in the transfer function for the constellation response $y_{132}$, for a SOBH with $\epsilon_{\Psi}\sim 1$.}
  \label{fig:}
\end{figure}

\begin{figure}
  \centering
%  \includegraphics[width=.8\linewidth]{plots/LISAtransferSOBHepsPsi10y12LReIm.pdf}
%  \hspace{0.1cm}
%  \includegraphics[width=.8\linewidth]{plots/LISAerrorSOBHepsPsi10y12L.pdf}
  \includegraphics[width=.98\linewidth]{plots/LISAtransfererrorSOBHepsPsi10y12L_py.pdf}
  \caption{Error in the transfer function for the constellation response $y_{132}$, for a SOBH with $\epsilon_{\Psi}\sim 10$.}
  \label{fig:}
\end{figure}

As explained in Sec.~\ref{subsec:lisafom}, slowly-chirping binaries can be problematic for the perturbative formalism of Sec.~\ref{sec:formalism}. Indeed, when far enough from merger, SOBH systems~\cite{Sesana16} can have error estimates $\epsilon_{\Psi}$ reaching and exceeding $1$, as shown in Fig.~\ref{fig:lisafomPsiMcDeltat}. Here we investigate these sources in more details, and propose an alternative treatment for their instrument response.

First, we should mention that these signals share similarities with the galactic binaries that will provide numerous quasi-monochromatic signals in the LISA band~\cite{LISA17}. They are far from merger, slowly chirping, and span only a narrow frequency band over the course of the LISA mission lifetime. For the galactic binaries, an accurate and very efficient numerical treatment of the response has been proposed in~\cite{} \SM{[C]} and widely used in applications (see e.g.~\cite{} \SM{[C]}). This treatment is referred to as the fast-slow decomposition, or heterodyning approach. If the gravitational wave signal extends only on the narrow frequency band $f \in [f_{*}, (1+\eta) f_{*}]$ with $\eta \ll 1$, then scaling out a carrier frequency of the signal by multiplying by $e^{-2 i \pi f_{*} t}$ will eliminate most of the time variability, allowing to process the signal through the time-domain response and to take a FFT with a Nyquist frequency shifted from $(1+\eta) f_{*}$ to $\eta f_{*}$, i.e. with a much smaller number of samples. This multiplication is simply equivalent to a shift in frequency domain, which can be restored after the numerical FFT has been computed. The efficiency of this approach is contingent to the smallness of $\eta$. For galactic binaries, this quantity is typically \SM{[complete]}. For SOBH systems, however, $\eta$ will take a continuous set of values from roughly $10^{-4}$ to $1$ \SM{[check values]}. \SM{[add sentence: cover the intermediate ground]} Although the systems for which $\eta$ is the largest are also the easiest to treat with the perturbative formalism of Sec~\ref{sec:formalism}, we propose here yet a third method for dealing with potentially troublesome systems. We leave for the future a more detailed study of SOBH systems as a population, and the investigation of the precise boundaries and overlap areas of the three methods as well as their respective computational costs.

The third approach we propose here exploits the fact that, in the LISA case, the modulations and delays entering~\eqref{eq:yslr} are periodic, with a period of one year and a frequency $f_{0} = \Omega_{0}/2\pi = 1/\mathrm{yr} \simeq 3.2\times 10^{-8}\mathrm{Hz}$. For any given frequency $f$, $G(f,t)$ is periodic in time, so that~\eqref{eq:defG} becomes a discrete Fourier series
\be\label{eq:Gdiscretefourier}
	G(f,t) = \sum_{n \in \mathbb{Z}} c_{n}(f) e^{-in\Omega_{0}t} \,,
\ee
with frequency-dependent discrete Fourier coefficients (that we will also call comb coefficients) given by the integrals (we recall our Fourier convention~\eqref{eq:defFT})
\be\label{eq:defcn}
	c_{n}(f) = \frac{\Omega_{0}}{2\pi} \int_{0}^{\frac{2\pi}{\Omega_{0}}} \ud t \, e^{i n \Omega_{0} t} G(f,t) \,.
\ee

In the case of the orbital delay, with $G_{0}(f,t)$ given by~\eqref{eq:G0} and~\eqref{eq:delay0}, the particularly simple expression of the delay gives an analytic expression for the coefficients $c_{n}^{0}$ in terms of Bessel functions of the first kind, as
\be\label{eq:cn0}
	c_{n}^{0} = i^{n} e^{i n \lambda} J_{n} \left[ -2 \pi f R \cos \beta \right]\,.
\ee
By contrast, for the constellation response, to our knowledge the coefficients $c_{n}^{L}$ do not admit such a simple close-form expression, and they must be computed numerically using the integrand given by~\eqref{eq:GslrL}. Truncating~\eqref{eq:Gdiscretefourier} to a finite order $N$, this computation reduces to a DFT \SM{[acronym]}, and the $c_{n}$ coefficients are given by~\eqref{eq:ykDFT} and~\eqref{eq:ckyk}. 

Inserting~\eqref{eq:Gdiscretefourier} into the convolution~\eqref{eq:FDkernel} leads to the following generalized discrete convolution:
\be\label{eq:transferdiscreteconvolution}
	\tilde{s}(f) = \sum_{n \in \mathbb{Z}} c_{n}(f - n f_{0}) \tilde{h} (f - n f_{0}) \,.
\ee
Thus, computing the Fourier-domain response now requires to convolve the signal with a discrete comb with frequency-dependent coefficients $c_{n}$. In practice, this sum is to be truncated at $|n|\leq N$, for some finite order $N$ determining the accuracy of the approximation. To assess the expected truncation error, we use a simple criterion based on the $L^{1}$-norm of the $\{c_{n}\}$ sequence. For a given target truncation error $\eta$, we define the truncation \SM{change notation to avoid confusion with $N$ from Fresnel stencil} $N(f, \eta)$ as the smallest integer such that 
\be\label{eq:criteriontruncationcomb}
	\sum_{|n| > N(f, \eta)} |c_{n}(f)| < \eta \,.
\ee
The truncation order $N(f,\eta)$ is frequency-dependent, and also depends on the orientation angles. For $c_{n}^{L}$, it also determines the number of function evaluations and the cost of the DFT. For $c_{n}^{0}$, the expression~\eqref{eq:cn0} can provide an asymptotic bound on the width of the comb. For large values of $n$ we have indeed the equivalent $| J_{n}[z] | \sim (e z/2n)^{n}/\sqrt{2\pi n} \; \left[ n\rightarrow + \infty \right]$~\cite{} \SM{[C: dlmf?]}, so that a conservative estimate for the truncation order (almost independent of $\eta$) is given by $N \geq  |e \pi f R \cos \beta|$. Fig.~\ref{fig:lisacombextent} shows numerical computations for $N(f,\eta)$, averaged over orientation angles, for both the constellation and the orbital response and for the truncation levels  $\eta=10^{-12}$, $10^{-6}$, $10^{-3}$. We see that, especially at high frequencies, the orbital response requires more coefficients than the constellation response due to the longer baseline of the delay. At the frequency $f=0.1\Hz$, we have for the orbital response $N \simeq 200$.

An important point is that, since the signals from slowly-chirping binaries extend on a narrow frequency band, for both responses the frequency-dependence of $c_{n}^{0}(f)$, $c_{n}^{L}(f)$ will be very mild, so that the coefficients can be computed at two or three frequencies and interpolated in-between. The computational cost of this approach is set by the truncation order $N$, and the number of intermediate frequencies where the transfer function resulting from the sum~\eqref{eq:transferdiscreteconvolution} must be computed for later interpolation, with the necessary waveform evaluations \SM{[possibly refer to appendix on interpolation errors here]}. We leave for future work a proper assessment of the computational performance of this method compared to the others \SM{[say a bit more ?]}. 

Although straightforward and generic, this writing of the Fourier-domain response function will be in general more expensive to compute than the perturbative transfer function approach of Sec.~\ref{sec:formalism}.

Note that for slowly-chirping binaries, the large features in the orbital transfer functions (see Fig.~\ref{} \SM{[R]} and Fig.~\ref{} \SM{[R]}) start to dominate signal-dependent derivatives. For the clarity of the presentation, we still do two separate comparisons for the orbital response~\eqref{} \SM{[R]} and the constellation response~\eqref{} \SM{[R]}, while the actual treatment would require to process the signal through the two steps of the response at once, instead of sequentially. This would result in replacing the discrete comb~\eqref{eq:defcn} by a double sum, however the extent of the constellation $c_{n}$ is rather limited in $n$.

%%%%%%%%%%%%%%%%%%%%%%%%%%%%%%%%%%%%
%%%%%%%%%%%%%%%%%%%%%%%%%%%%%%%%%%%%

\section{Application to waveforms from precessing binaries}
\label{sec:precession}

\SM{[explain somewhere the fact that the modulation has support for negative f, convolution has support to larger f - pretty important for direct convolution]}

\SM{[cite GWFrames]}

\SM{[replace $pp$ everywhere with $\perp\perp$]}

\SM{[check error estimates with single-spin configurations $+$, $\perp$, $-$]}

In this Section, we illustrate the application of the formalism described in~\ref{sec:formalism} to signals from precessing binaries, with modulations induced by the precession. We investigate how the separation of the relevant timescales evolves in the late inspiral and after the merger occurs. We will also give in App.~\ref{app:precpreviousapproaches} a short overview of previous approaches to this problem of the Fourier-domain precession, and how they relate to our formalism.

%%%%%%%%%%%%%%%%%%%%%%%%%%%%%%%%%%%%

\subsection{Precession and frame decomposition}
\label{subsec:precdef}

We will now turn to the application of our formalism to the decomposition of signals from precessing binaries. As we already discussed in Sec.~\ref{subsec:modulationPrec}, in the presence of spin components that are not aligned with the orbital angular momentum, an inspiraling binary system will undergo precession of its orbital plane~\cite{Apostolatos+94, Kidder95}. This is a crucial effect to be taken into account in the modelling of such signals.

As proposed by several authors~\cite{BCV03b, BCPTV05, Schmidt+10, OShaughnessy+11, Boyle+11}, if one performs the mode decomposition of the waveform~\eqref{eq:defmodes} not in a fixed inertial frame but rather in a time-dependent, rotating frame that follows the plane of the orbit, it is possible to restore much of the structure of a non-precessing waveform. In particular, one recovers qualitatively the hierarchy of mode amplitudes that prevails for non-precessing systems, with the modes $h_{22}$ and $h_{2,-2}$ being dominant. In the following, we will identify the $z$-axis of the co-precessing frame as the dominant eigenvector of the matrix representing the action of the angular momentum operator acting on the waveform modes, as proposed in~\cite{OShaughnessy+11} (see App.~\ref{app:wigner} for more details).

We define $(\alpha, \beta, \gamma)$ are the Euler angles of the active rotation from the inertial frame to the precessing frame, in the $(z,y,z)$ convention. The first two angles, $\alpha$ and $\beta$, are the two spherical angles tracking the direction of the radiation axis, which during the inspiral follows essentially the normal to the orbital plane. The last angle $\gamma$ parametrizes the remaining freedom of rotation around this radiation axis. To fix this third degree of freedom, we use the minimal rotation condition~\cite{Boyle+11}, enforcing the absence of rotation of the precessing frame around the radiation axis. In terms of Euler angles, this condition translates to
\be\label{eq:gammadot}
	\dot{\gamma} = -\dot{\alpha}\cos \beta \,.
\ee
A natural choice for the $z$ axis of the inertial frame is the direction of $\bm{J}$, the total angular momentum, which is almost constant\footnote{Except in the cases of transitional precession~\cite{Apostolatos+94}, where for high mass ratios and large antialigned spins the orbital angular momentum and the spin can almost cancel each other.}. Quite generically, and in particular in the case of simple precession~\cite{Apostolatos+94, Kidder95} but also in the model we will take for the post-merger precession, the radiation axis precesses on a cone around the direction of $\bm{J}$, with $\alpha$ increasing, while $\beta$, the opening angle of the precession cone, is slowly varying.

The modes in the inertial frame $h_{\ell m}^{\rm I}$ and the modes in the precessing frame $h_{\ell m}^{\rm P}$ are then related by~\cite{Goldberg+67}
\begin{subequations}
\label{eq:wignerrot}
\begin{align}
	h_{\ell m}^{\rm I} = \sum\limits_{m=-\ell}^{\ell} \calD^{\ell *}_{mm'} (\alpha,\beta,\gamma) h_{\ell m'}^{\rm P} \,, \\
	h_{\ell m}^{\rm P} = \sum\limits_{m=-\ell}^{\ell} \calD^{\ell }_{m'm} (\alpha,\beta,\gamma) h_{\ell m'}^{\rm I} \,,
\end{align}
\end{subequations}
where the coefficients $\calD^{\ell}_{mm'}$ are given by Wigner D-matrices~\cite{Wigner59} as
\be\label{eq:defWignerD}
	\calD^{\ell}_{mm'} (\alpha, \beta, \gamma) = e^{im \alpha} d^{\ell}_{mm'}(\beta) e^{im' \gamma}\,.
\ee
Here, the Wigner d-matrix $d^{\ell}_{mm'}(\beta)$ takes the form of a polynomial in $\cos (\beta/2)$, $\sin (\beta/2)$, and acts as an amplitude for the modulation function. We refer to App.~\ref{app:wigner} for explicit expressions.

In the following, our objective will be to compute mode-by-mode transfer functions $\calT^{\ell}_{mm'}$, defined as
\be\label{eq:defprectransfer}
	\mathrm{FT} \left[ \calD^{\ell *}_{mm'} (\alpha,\beta,\gamma) h_{\ell m'}^{\rm P} \right] (f) \equiv \calT^{\ell}_{mm'}(f) \tilde{h}_{\ell m'}^{\rm P} (f)
\ee
such that the complete Fourier-domain inertial-frame waveform will be given as the sum of these individual mode contributions,
\be\label{eq:defprechIsum}
	\tilde{h}_{\ell m}^{\rm I} (f) = \sum\limits_{m'=-\ell}^{\ell} \calT^{\ell}_{mm'}(f) \tilde{h}_{\ell m'}^{\rm P} (f) \,.
\ee
In the notations that we used in Sec.~\ref{sec:formalism}, in the absence delays, we wish to compute the convolutions
\be\label{eq:precconvolution}
	\tilde{s} (f) = \int df' \; \tilde{F}(f') \tilde{h} (f-f') \,,
\ee
with $\tilde{s}(f)$ being one mode contribution to $\tilde{h}_{\ell m}^{\rm I} (f)$ in~\eqref{eq:defprechIsum}, $\tilde{h}$ being one of the P-frame modes $\tilde{h}_{\ell m'}^{\rm P} (f)$, and the modulation $F(t)$ being one of the time-dependent Wigner matrices $\calD^{\ell *}_{mm'}$.

An important qualitative observation is that, for small to moderate opening angles, the modulation functions show a clear hierarchy in amplitude. This situation is quite generic, since large misalignments between $\bm{J}$ and $\bm{\ell}$ can only be reached with large spins and large mass ratios. From the explicit expression~\eqref{eq:defWignerdapp}, one can see that $d^{\ell}_{mm'}(\beta)$ is greatly suppressed in the limit $\beta \rightarrow 0$ when increasing $|m-m'|$. Intuitively, this means that, in the limit of a small misalignment between frames, the rotation produces mainly mode contributions with the same mode number. Additionally, when $\beta$ is constant or slowly varying, \eqref{eq:gammadot} gives $\gamma \simeq - \alpha \cos\beta$, and we have in that case
\be\label{eq:wignerphasesimpleprec}
	\calD^{\ell *}_{mm'} (\alpha, \beta, \gamma) \propto e^{i(m' \cos\beta - m) \alpha} \,,
\ee
with $\cos\beta \simeq 1$ for small $\beta$. This shows that the modulation for $m=m'$ has a suppressed phase, while increasing $|m-m'|$ increases the magnitude of the modulation phase. Thus, modulation functions for distant mode contributions (for instance from the $h^{\rm P}_{22}$ mode all the way to the $h^{\rm I}_{2,-2}$ mode) have larger phase evolutions, but smaller amplitudes.

The relations~\eqref{eq:wignerrot} do not mix different values of $\ell$. Furthermore, rotations leave invariant the combined square amplitude $\calA_{\ell}$ for each $\ell$, as well as the total square amplitude:
\be\label{eq:defsumamplitude}
	\calA^{2} = \sum\limits_{\ell \geq 2}\calA_{\ell}^{2} = \sum\limits_{\ell \geq 2}\sum\limits_{m=-\ell}^{\ell} |h_{\ell m}|^{2} \,.
\ee
One can therefore use these amplitudes (in our case, limited to $\ell = 2$) to define a frame-independent peak amplitude of the precessing waveform.

When approximating the precessing-frame waveform by a non-precessing one, one can use the assumption (exactly valid for non-precessing systems, see~\eqref{eq:symmetryhlminusm})
\be\label{eq:approxsymmetryhlminusm}
	h^{\rm P}_{\ell,-m} \simeq (-1)^{\ell} h^{\rm P*}_{\ell,m} \,.
\ee
Similarly, in the Fourier domain, one can neglect either the negative or positive frequency band in the Fourier transform of precessing-frame modes (as in~\eqref{eq:zeronegativef})
\begin{align}\label{eq:approxzeronegativef}
	\tilde{h}_{\ell m}^{\rm P} (f) &\simeq 0 \text{ for } f<0, \; m>0 \nn\,,\\
	\tilde{h}_{\ell m}^{\rm P} (f) &\simeq 0 \text{ for } f>0, \; m<0 \,,
\end{align}
and neglect altogether the $m=0$ modes, $\tilde{h}_{\ell 0}^{\rm P} (f) \simeq 0$.

When using the approximations~\eqref{eq:approxsymmetryhlminusm}-\eqref{eq:approxzeronegativef}, one can derive a symmetry relation in the transfer functions themselves. From the explicit expression of the Wigner matrices, we have indeed (see App.~\ref{app:wigner})
\be
	\calD^{\ell *}_{-m,-m'} = (-1)^{m+m'}\calD^{\ell}_{mm'} \,.
\ee
Since for a function $g$ we have in general for its conjugate
\be
	\mathrm{FT}[g^{*}](-f) = \tilde{g}(f)^{*} \,,
\ee
we can write, using~\eqref{eq:approxsymmetryhlminusm},
\be
	\mathrm{FT}[\calD^{\ell *}_{mm'} h_{\ell m'}^{\rm P}](-f)^{*} = (-1)^{\ell+m+m'} \mathrm{FT}[\calD^{\ell *}_{-m,-m'} h_{\ell, -m'}^{\rm P}](f) \,,
\ee
or, for transfer functions,
\be
	\calT^{\ell}_{mm'}(f) = (-1)^{m+m'} \calT^{\ell *}_{-m,-m'}(-f) \,.
\ee
This means that such a model is required to cover only the positive frequency band and the values $m'>0$, since
\begin{align}\label{eq:hIlmposnegfreq}
	\tilde{h}^{\rm I}_{\ell m}(f) &= \sum_{m'>0} \calT^{\ell}_{mm'}(f) \tilde{h}^{\rm P}_{\ell, m'}(f) \text{ for } f>0 \,, \nn\\
	\tilde{h}^{\rm I}_{\ell m}(f) &= \sum_{m'>0} (-1)^{\ell+m+m'} \calT^{\ell *}_{-m,m'}(-f) \tilde{h}^{\rm P *}_{\ell, m'}(-f) \text{ for } f<0 \,.
\end{align}

Further simplifications occur when including only the dominant harmonics $h^{\rm P}_{22}$, $h^{\rm P}_{2,-2}$ in the precessing-frame waveform (as is done for instance in PhenomP~\cite{Hannam+13}, as well as in our toy model~\ref{subsec:precmodel}), we have in this case for $f<0$
\be
	\calT^{2}_{m,-2}(f) = (-1)^{m} \calT^{2 *}_{-m,2}(-f) \,,
\ee
which for both $f>0$ and $f<0$ translates into
\be\label{eq:symmetryhIfor22only}
	\tilde{h}^{\rm I}_{2 m}(f) = (-1)^{m} \tilde{h}^{\rm I *}_{2,-m}(-f) \,.
\ee
When reconstructing the polarizations $h_{+},h_{\times}$ according to~\eqref{eq:hpcfrommodes}, we have in this case
\be
	\tilde{h}_{+,\times} (f) = \sum_{m = -2}^{2} L^{+,\times}_{2 m} \tilde{h}^{\rm I}_{2 m}(f) \,,
\ee
where we defined
\begin{align}
	L^{+}_{\ell m} &\equiv \frac{1}{2} \left( {}_{-2}Y_{\ell m} + (-1)^{m} {}_{-2}Y_{\ell, -m}^{*} \right) \,, \nn\\
	L^{\times}_{\ell m} &\equiv \frac{i}{2} \left( {}_{-2}Y_{\ell m} - (-1)^{m} {}_{-2}Y_{\ell, -m}^{*} \right) \,.
\end{align}

Finally, we note that with the restriction to $\ell = 2$ and $m' = 2$, in the approximate phase given in Eq.~\eqref{eq:wignerphasesimpleprec}, the coefficient of $\alpha$ is close to $0$ for small $\beta$ for $m=2$, and increasingly positive when going down from $m=1$ to $m=-2$. In our Fourier convention~\eqref{eq:defFT}, the Fourier transforms $\tilde{F}$ thus have most of their support on negative frequencies. This means that the support of the convolution integral~\eqref{eq:precconvolution} extends to the right side for $\tilde{h}$, i.e. $f-f' > f$. This point will become important when considering the high-frequency part of the signal in Sec.~\ref{subsec:trigopoly}. Note, however, that this statement might not be true anymore when including more modes in our model.

\subsection{Simplified model for precessing IMR waveforms}
\label{subsec:precmodel}

\SM{[qualify how bad the model is with respect to full NR ?]}

\SM{[explain shortcomings of pure SEOB in post-merger, problem of $\beta$ oscillating ?]}

We wish to apply the formalism presented in Sec.~\ref{sec:formalism} to IMR waveforms of precessing binaries, exploring the separation of the timescales involved through the inspiral and merger, and assessing the relevance of the higher-order corrections summarized in~\ref{subsec:executivesummary}. To this end, we will use a simplified model for the precession, allowing a number of simplifications and idealizations.

%To this end, we will use a simplified model for IMR waveforms of precessing binaries, which we will use in the following to . Since our purpose in this article is not to propose a full-fledged waveform model, but rather to investigate wether the formalism of Sec.~\ref{sec:formalism} is relevant to tackle the merger part of the signals, and wether its application reduces the errors in the reconstruction of transfer functions, we will allow for a number of idealizations and simplifications in this toy model.

We want to be able to investigate signals with a long inspiral phase, beyond the range covered by numerical relativity simulations, and we want to include the merger and ringdown phase. We will use the following three ingredients for this simplified model:
\begin{itemize}
	\item PhenomD~\cite{Husa+15, Khan+15} for the IMR Fourier-domain waveform in the precessing frame;
	\item SEOBNRv3~\cite{Pan+13, BTB16} for the Euler angles during the inspiral;
	\item an effective extension of the Euler angles post-merger based on~\cite{OShaughnessy+12} (see~\eqref{eq:OmegaframeQNM} below).
\end{itemize}

For the precessing-frame waveform, we make all the simplifying assumptions described above in Sec.~\ref{subsec:precdef}: we approximate it as a non-precessing waveform, further enforcing the approximations~\eqref{eq:approxsymmetryhlminusm} and~\eqref{eq:approxzeronegativef} and limiting our analysis to the dominant harmonics\footnote{See~\cite{London+17} for a recent extension of the model to include higher harmonics.} $h^{\rm P}_{22}$ and $h^{\rm P}_{2,-2}$. We will use the PhenomD waveform model~\cite{Husa+15, Khan+15}, an aligned-spin Fourier-domain model publicly available in the LIGO Algorithm Library (LAL) that covers the inpiral, merger and ringdown for binaries with generic spin magnitude. The amplitude and phase of the waveform are produced as piecewise analytical functions of the frequency. Using a Fourier-domain approximant that is smooth by construction avoids the Gibbs oscillations induced by the tapering of of finite-length time-domain waveforms, and will allow us to easily take Fourier-domain derivatives of the waveform\footnote{Note that the PhenomD amplitude and phase are smooth on three separate frequency bands (inspiral, intermediate, and ringdown) with junction conditions that are only of class $C^{1}$. To avoid spurious discontinuities, we introduce decaying corrective terms of the form $(f-f_{\rm join})^{2} e^{-\lambda(f-f_{\rm join})^{2}}$ on each side of $f_{\rm join}$, the junction frequency between any two given bands, resulting in functions of class $C^{2}$.}.

For the frame trajectory in the inspiral phase, we use an SEOBNRv3 waveform~\cite{Pan+13, BTB16}, that incorporates all degrees of freedom of both spins. As the inspiral of the SEOBNRv3 waveform is constructed by using the instantaneous orbital plane of the dynamics as the precessing frame, the frame extraction procedure of~\cite{OShaughnessy+11} simply returns the normal to the orbital plane as the radiation axis. The frame thus presents oscillatory features of small amplitude at the orbital timescale, due to the nutation of the orbital plane, which we smooth out using a Gaussian filtering with width based on the orbital phase as in~\cite{Blackman+17a}.

For the post-merger precession, the intuitive interpretation of the precessing frame as roughly following the orbital plane of the binary is lost. However, the prescription of~\cite{OShaughnessy+11} to extract the radiation axis is based entirely on the waveform itself, and can be applied to the ringdown as well. For $\ell=2$, analyzing numerical relativity waveforms, Ref.~\cite{OShaughnessy+12} described a qualitative model where the radiation axis essentially keeps precessing around the final angular momentum $\bm{J}$, but transitions to a faster precession rate, with an angular velocity determined by the quasinormal mode (QNM) frequencies of the remnant black hole as
\be\label{eq:OmegaframeQNM}
	\Omega_{\rm frame} = \omega_{220}^{\rm QNM} - \omega_{210}^{\rm QNM} \,.
\ee
This translates in general in a significant acceleration of the precession post-merger with respect to the inspiral (see Fig.~\ref{fig:precmodel}). For the opening angle of the precessing cone $\beta$, Ref.~\cite{OShaughnessy+12} proposes an exponential decay driven similarly by the difference between damping frequencies. We found by a non-exhaustive \SM{[allowed to say that ? the catalog is vast now]} inspection of precessing waveforms currently available in the SXS catalog (see e.g.~\cite{SXScatalog, Mroue+12, Mroue+13}) that this picture is at least qualitatively correct for $\ell=2$, with $\beta$ appearing to transition to a lower value rather than decaying all the way to zero. By contrast, we found that the current ringdown prescription in the SEOBNRv3 model can cause a qualitatively different behaviour, with oscillations in $\beta$. \SM{[motivate this with final spin estimate as well, if we update this ?]}
%and its estimate for the spin of the final remnant~\cite{} would require to be updated (see e.g.~\cite{}) as shown in Fig.~\ref{fig:precmodel}.

In our model, after reaching the time of peak amplitude as defined in~\eqref{eq:defsumamplitude}, we will use the prescription~\eqref{eq:OmegaframeQNM} for the post-merger frame precession and simply keep $\beta$ constant,
\begin{subequations}
\begin{align}\label{eq:eulerQNM}
	\alpha_{\rm post-merger}(t) &= \alpha(t_{\rm peak}) + (\omega_{220}^{\rm QNM} - \omega_{210}^{\rm QNM})(t-t_{\rm peak}) \,, \\
	\beta_{\rm post-merger}(t) &= \beta(t_{\rm peak}) \,, \\
	\gamma_{\rm post-merger}(t) &= \gamma(t_{\rm peak}) - (\alpha(t) - \alpha(t_{\rm peak}) )\cos \beta \,,
\end{align}
\end{subequations}
where we use the minimal-rotation condition~\eqref{eq:gammadot} for a constant $\beta$. To compute the QNM frequencies as a function of the final spin $\chi_{f}$, we use fits constructed in Ref.~\cite{Berti+05}, illustrated in Fig.~\ref{fig:QNM}. The final spin is taken to be the same as the one computed internally to the SEOBNRv3 code~\cite{Pan+13, BTB16}, where the final-spin fit formula of Ref.~\cite{BR09} \SM{[see if we can update the final-spin formula]}, built for spin-aligned systems, is applied to the spin components projected on the orbital angular momentum $\bm{L}$ at merger.

Fig.~\ref{fig:precmodel} presents a comparison of the frame trajectory at merger for the SEOBNRv3, NR waveforms and for our model~\eqref{eq:eulerQNM}. The example catalog waveform is SXS:BBH:0058, generated with SpEC~\cite{SXScatalog, SpEC, Mroue+12, Mroue+13}, with a mass ratio of $q=5$, and a single in-plane spin of $\chi_{1} = 0.5$. The NR and SEOB post-merger frames disagree mainly due to a different value of the final spin $\chi_{f}$\footnote{To compute $\chi_{f}$, SEOBNRv3 uses fitting formulas from~\cite{BR09}. These formulas will be updated to more recent fits incorporating additional NR data (see e.g.~\cite{HBR16}) in the next version of the SEOB code~\cite{Ossokine+18}, which is expected to reduce or remove this disagreement. \SM{[update this reference in prep.]}}, which leads to a different $\Omega_{\rm frame}$. Given their different value of $\chi_{f}$, however, both agree with the qualitative description~\eqref{eq:OmegaframeQNM}. The oscillations developing after $t\sim 50 M$ occur in a regime where the overall waveform amplitude has decayed to small values. Our simple model~\eqref{eq:eulerQNM} follows well the SEOBNRv3 behaviour for $\alpha$ and $\gamma$, but departs more for $\beta$ as we do not model its variation at merger.

We would now like to underline a number of caveats and limitations of our simplified model. First, it is clear from Fig.~\ref{fig:precmodel} that the acceleration of the precession post-merger is going to be the most challenging feature for the separation of timescales at the basis of Sec.~\ref{sec:formalism}. While we checked the qualitative soundness of the decomposition outlined in Sec.~\ref{subsec:precdef} and of the post-merger precession~\eqref{eq:OmegaframeQNM} in some of the available numerical relativity waveforms~\cite{SXScatalog}, it is important to stress that there is no guarantee that another prescription for the precessing frame would not yield a better separation of timescales. Secondly, approximating the precessing-frame waveform by a spin-aligned one comes with known limitations, ignoring for instance mode asymmetries departing from~\eqref{eq:approxsymmetryhlminusm} as shown in~\cite{Boyle+14}. Thirdly, more exploration of the parameter space would be needed, as the QNM frequencies entering~\eqref{eq:OmegaframeQNM} vary rapidly for large $\chi_{f}$. Finally, we ensure the smoothness both of the Fourier-domain precessing-frame waveform and of the time-domain modulation, allowing us to take the derivatives required by the formalism of Sec.~\ref{sec:formalism}, which is an idealization.

Leaving aside the question of a more in-depth investigation of the best representation of precessing waveforms in their post-merger phase, we will proceed to investigating the separation of timescales in three chosen examples. The parameters of these example cases are summarized in Table~\ref{tab:precparams}. We consider a mass ratio of $q=4$, close-to-maximal spins of $\chi_{1} = \chi_{2} = 0.95$, and vary both spins misalignmnent angles to be $\pi/6$ (case labeled $++$, close to aligned spins), $\pi/2$ ($\perp\perp$, spins in the plane) and $5\pi/6$ ($--$, close to anti-aligned spins). Table~\ref{tab:precparams} also shows the remnant spins that are internally computed in both the SEOBNRv3 and the PhenomD/PhenomP codes, as well as the QNM frequencies and the resulting frame precession frequency $\Omega_{\rm frame}$. Note that the case $--$ shows a larger variation in the direction of $\bm{J}$, and the spin of the remnant is almost 0, due to the fact that the spin is large and antialigned with a large mass ratio. This example therefore serves the purpose of lying at the edge of validity of the picture of standard precession along a cone around an almost-fixed direction, and the standard hierarchy between mode contributions is not valid in this case. \SM{[soundness of trajectories, particularly in this case, are to be checked again]}

\begin{figure}
  \centering
%  \includegraphics[width=.45\linewidth]{plots/eulerSXS0058alphagamma.pdf}
%  \hspace{0.2cm}
%  \includegraphics[width=.45\linewidth]{plots/eulerSXS0058beta.pdf}
  \includegraphics[width=.98\linewidth]{plots/eulerSXS0058_py.pdf}
  \caption{Evolution of Euler angles $(\alpha, \beta, \gamma)$ near merger for the example SpEC waveform SXS:BBH:0058~\cite{SpEC, SXScatalog, Mroue+12, Mroue+13} \SM{[C]}. Full, dashed and dotted lines represent the NR, SEOB and extended SEOB waveforms respectively. The vertical line at $t=0$ indicates the time of merger. In the left panel the full and dashed black lines indicate the asymptotic behaviour~\eqref{eq:OmegaframeQNM} for the rotation of the frame around the direction of the final $\bm{J}$. The difference between NR and SEOB there is due to the value of the final spin, with $\chi_{f}^{\rm NR} = 0.54$ and $\chi_{f}^{\rm SEOB} = 0.42$, which leads to a different $\Omega_{\rm frame}$. In the right panel, $\beta$ is asymptotically constant in our toy model. \SM{[Add amplitude/$\omega$ plots comparing NR/SEOB, and possibly PhenomD for the 22 mode ? Add Euler angles produced by IFFT of PhenomP ?]}}
  \label{fig:precmodel}
\end{figure}

%\begin{figure*}
%  \centering
%  \includegraphics[width=.48\linewidth]{plots/prectoymodelalphagamma.pdf}
%  \hspace{0.2cm}
%  \includegraphics[width=.48\linewidth]{plots/prectoymodelbeta.pdf}
%  \caption{Evolution of Euler angles $(\alpha, \beta, \gamma)$ for our two toy models. Dashed lines represent Case I and solid lines Case II. The vertical line indicates the time of merger. In the left panel the dashed black lines indicate the enforced asymptotic behaviour in Case II, where the frame rotates with constant angular velocity $\omega_{220} -\omega_{210}$. In the right panel, $\beta$ is asymptotically constant.}
%  \label{fig:prectoymodel}
%\end{figure*}

\SM{[Check the mismatch for $--$ between Jf from dynamics and chif used in SEOB internals to do the ringdown attachment - check that the frame trajectory makes sense by comparing to orbital plane.]}

%In Eq.~\eqref{eq:defmodulationprec} above, the modulation function $F$ evolves on the precessional timescale, which evolves throughout the inspiral. The separation between the precessional timescale and the orbital timescale will be crucial for our analysis. In the limit of low frequencies, we will see in Sec.~\ref{} below that, although these timescales become more and more separated, the decrease in the chirping rate gives raise to a corrective contribution that does not vanish in this limit. In the other limit, as the system gets close to merger, the separation of timescales becomes weaker, and we will explore in Sec.~\ref{} the application of our formalism to the merger and ringdown phase.

\begin{table}[t]
\begin{ruledtabular}\caption{Parameters of of the three example cases that we use to illustrate our formalism. The three cases differ by their spin alignment angles $\theta_{A} = \bm{\chi}_{A} \cdot \hat{\bm{L}}_{i}$, the spins being in the orbital plane, almost aligned and almost anti-aligned. The table gives also the final spins magnitudes (for SEOB as well as PhenomD for comparison), angles between the initial and final angular momenta $\theta(\bm{J}_{i}, \bm{J}_{f})$, as well as QNM frequencies and frame rotation velocity used for the post-merger extension. \SM{[To be updated with longer waveforms, 10Hz 10Msol. ?]} \SM{[add misalignment angles $(\ell, \bm{J}_{i})$?]}}\label{tab:precparams}
%\begin{tabular}{C{.2\columnwidth}C{.24\columnwidth}C{.15\columnwidth}}
\begin{tabular}{ccccccc}\label{tab:precexamples}
	$f_{\rm min}$ & $ M_{\rm min} $ & $q$ & $\chi_{1}$ & $\chi_{2}$ & $ \phi_{1} $ & $ \phi_{2} $ \\
	\hline
	$20\mathrm{Hz}$ & $20\Msol$ & $ 4.0 $ & $ 0.95 $ & $ 0.95 $ & $0$ & $\pi/2$ \\
	\hline\hline
	Case && ++ && $\perp\perp$ && $--$ \\
	\hline
	$\theta_{1}$ && $\pi/6$ && $\pi/2$ && $5\pi/6$ \\
	$\theta_{2}$ && $\pi/6$ && $\pi/2$ && $5\pi/6$ \\
	\hline
	$\chi_{f}$ && $0.89$ && $0.50$ && $0.02$ \\
	$\chi_{f}^{\rm Ph}$ && $0.90$ && $0.47$ && $0.002$ \\
	$\theta(\bm{J}_{i}, \bm{J}_{f})$ && $0.02$ && $0.04$ && $0.27$ \\
	$M \omega_{220}^{\rm QNM}$ && $0.66$ && $0.47$ && $0.377$ \\
	$M \omega_{210}^{\rm QNM}$ && $0.51$ && $0.42$ && $0.375$ \\
	$M \Omega_{\rm frame}$ && $0.15$ && $0.04$ && $0.002$ \\
\end{tabular}
\end{ruledtabular}
\end{table}

%\begin{table}[t]
%\begin{ruledtabular}\caption{Parameters of the three example cases that we use to illustrate our formalism.}\label{tab:precparams}
%%\begin{tabular}{C{.2\columnwidth}C{.24\columnwidth}C{.15\columnwidth}}
%\begin{tabular}{cccc}
%	Case & pp & $++$ & $--$ \\
%	\hline
%	$q$ & $4.0$ & $4.0$ & $4.0$ \\
%	$\chi_{1}$ & $0.95$ & $0.95$ & $0.95$ \\
%	$\chi_{2}$ & $0.95$ & $0.95$ & $0.95$ \\
%	$\theta_{1}$ & $\pi/2$ & $\pi/6$ & $5\pi/6$ \\
%	$\theta_{2}$ & $\pi/2$ & $\pi/6$ & $5\pi/6$ \\
%	$\phi_{2}$ & $0.0$ & $0.0$ & $0.0$ \\
%	$\phi_{2}$ & $\pi/2$ & $\pi/2$ & $\pi/2$
%\end{tabular}
%\end{ruledtabular}
%\end{table}

\SM{[We could add plots of the 3 frame trajectories, and of the FD transfer functions in Amp/Phase form. But the trajectories are not very informative except for - - (where our model might not be very solid) and for showing oscillations in $\beta$. The reconstruction errors plots already show the amplitude of the modes themselves, giving an idea of the hierarchy; what is the missing is the phase in Tlmmp, which is not very informative.]}

%\begin{figure*}
%  \centering
%  \includegraphics[width=.48\linewidth]{plots/prectransferAcaseI.pdf}
%  \hspace{0.2cm}
%  \includegraphics[width=.48\linewidth]{plots/prectransferPsicaseI.pdf}
%  %
%  \includegraphics[width=.48\linewidth]{plots/prectransferAcaseII.pdf}
%  \hspace{0.2cm}
%  \includegraphics[width=.48\linewidth]{plots/prectransferPsicaseII.pdf}  
%  \caption{Amplitude and phase of the Fourier-domain transfer fonctions $\calT_{22,2m'}$ as defined in~\eqref{eq:defTlmlmp} for individual mode contributions. The left panels show the amplitude and the right panels the phase.}
%  \label{fig:prectransfer}
%\end{figure*}

%%%%%%%%%%%%%%%%%%%%%%%%%%%%%%%%%%%%

\subsection{Estimates for the separation of timescales}
\label{subsec:sizecorrPrec}

\begin{figure}
  \centering
  \includegraphics[width=.98\linewidth]{plots/postmergeromega_py.pdf}
  \caption{\SM{[label y axis]} Quasi-normal mode frequencies for the 0th overtone of the modes $22$ and $21$, as a function of the dimensionless spin of the final black hole $\chi_{f}$, as well as their difference $\Omega_{\rm frame}$ (see~\eqref{eq:OmegaframeQNM}). The lower panel shows the  ratio $\Omega_{\rm frame} / \omega^{\rm QNM}_{220}$ characteristic of the separation of timescales between the phase of the modulation~\eqref{eq:wignerphasesimpleprec} and the P-frame waveform phase.}
  \label{fig:QNM}
\end{figure}

%We show in Fig~\ref{} the frequencies $\omega_{220}^{\rm QNM}$, $\omega_{210}^{\rm QNM}$ and their difference $\Omega_{\rm frame}$. The behaviour for near-to-extremal spin of the remnant would require more exploration, as the QNM frequencies vary a lot when approaching extremality.

\SM{[add plot of QNM frequency and frequency diff according to final spin ?]}.

\SM{[justify somewhere smallness of cubic-in-phase term]}

\SM{[write somewhere that we require mainly FD smoothness]}

We now turn to the separation of timescales and to the magnitude of the higher-order corrections derived in Sec.~\ref{sec:formalism} in the case of precessing binaries, as can be estimated by the quantities $\epsilon_{\Psi 2}$, $\epsilon_{A1}$ and $\epsilon_{A 2}$ introduced in~\eqref{eq:deffom}.

In the inspiral phase, we can obtain a qualitative picture of the timescales involved by using well-known leading-order post-Newtonian results. To simplify things further, we will consider a single-spin system, and use orbit-averaged precession equations. In this configuration, both the orbital angular momentum $\bm{L}$ and the spin vector $\bm{S}$ undergo simple precession on a cone around the total angular momentum $\bm{J}$ (see e.g.~\cite{Apostolatos+94, Kidder95}), with an opening angle of the cone and a precession velocity $\Omega_{\rm prec}$ that vary only on the radiation reaction timescale. In the general case, the presence of two spins complicates the evolution of the system, but the picture of a precession cone for $\bm{L}$ remains approximately valid (see~\cite{Kesden+14} for a classification of generic precession trajectories).

We will use the following notations: for $m_{1}$, $m_{2}$ the masses of the two bodies, we set $M=m_{1}+m_{2}$, $\nu=m_{1}m_{2}/M^{2}$, $\delta = (m_{1}-m_{2})/M$, and $|\bm{S}_{A}|=Gm_{A}^{2} \chi_{A}$ for $A=1,2$, with $\chi$ the dimensionless spin between $0$ and $1$ for Kerr black holes. We define $\bm{\ell}$ as the unit vector normal to the orbital plane. We take the convention $m_{1} \geq m_{2}$ and assume that only the more massive object has a spin, $\bm{S}_{2} = 0$. As in Sec.~\ref{subsec:SPA}, we will use the notation $v = (G M \omega/c^{3})^{1/3}$ with $\omega = \dot{\varphi}$ the orbital frequency, which translates to $v=(G \pi M f/c^{3})^{1/3}$ for the 22 mode when the SPA applies. At the Newtonian order, the orbital angular momentum is $\bm{L} = L_{N} \bm{\ell}$, with
\be\label{eq:defLN}
	L_{N} = \frac{G M^{2} \nu}{c v} \,.
\ee
Since we neglect radiation reaction, $\bm{J} = \bm{L} + \bm{S}_{1}/c$ is treated a constant that we use to set the z-axis so that $\bm{J} = J \bm{e}_{z}$, and we decompose the spin in its aligned and perpendicular components as $\bm{S}_{1} = S_{1}^{z} \bm{e}_{z} + \bm{S}_{1}^{\perp}$. At leading order, the precession equations read
\begin{align}
	\dot{\bm{S}}_{1} &= \bm{\Omega}_{1} \times \bm{S}_{1} \,, \nn\\
	\dot{\bm{\ell}} &= - \frac{1}{c L_{N}}  \dot{\bm{S}}_{1}\,,
\end{align}
with the spin precession velocity~\cite{Kidder95}
\be
	\bm{\Omega}_{1} = \Omega_{1} \bm{\ell} = \frac{c^{3}}{G M} \left( \frac{3}{4} + \frac{\nu}{2} - \frac{3\delta}{4} \right) v^{5} \bm{\ell} \,.
\ee
which is formally a 1PN quantity. Considering only the leading PN order, we ignore effects quadratic in the spin that would enter here at 1.5PN. Decomposing the vectors in their in-plane component and projection on $z$, and using $\delta^{2} = 1-4\nu$ to make explicit the overall scaling in $\nu$, we obtain for the frame precession velocity~\cite{Kidder95}
\be
	\dot{\alpha} \equiv \Omega_{\rm prec} = \frac{c^{4}}{G^{2} M^{3}} \frac{7+\delta}{2(1+\delta)} v^{6} J\,.
\ee
Separating the factors as
\be\label{eq:defLambdaxi}
	\Lambda \equiv \frac{7+\delta}{4(1+\delta)} \,, \quad	\xi \equiv 1 + \frac{v S_{1}^{z}}{G M^{2} \nu} \,,
\ee
we see that $\Lambda$ is a mass ratio-dependent factor chosen to be always of order 1, varying from $7/4$ for equal masses to $1$ in the test-mass limit, while the factor $\xi$ contains the contribution of the aligned component of the spin to the precession rate. With this notation,
\be\label{eq:Omegaprec}
	\Omega_{\rm prec} = \frac{2c^{3} \nu}{G M} \Lambda \xi v^{5} \,.
\ee
If precession effects are in general larger for larger mass ratios, as the precession cones widens, the overall scaling of $\nu$ in~\eqref{eq:Omegaprec} above shows that, as long as the orbital angular momentum still dominates the spin in $\xi$, increasing the mass ratio yields a slower precession rate. For high mass ratios and spins, the correction to $\xi$ in~\eqref{eq:defLambdaxi} starts to become important.

We now turn to the Euler angles and modulation functions $\calD^{\ell *}_{mm'}(\alpha, \beta, \gamma)$, given explicitly in~\eqref{eq:defWignerD}. As explained in~\ref{subsec:precdef}, in the case of simple precession, the opening angle of the precession cone $\beta$ is essentially constant, so that the minimal rotation condition~\eqref{eq:gammadot} gives $\gamma  = -\alpha \cos \beta$ (up to a constant), and the only variable part of the precession modulation functions in~\eqref{eq:wignerrot} are the phases, according to~\eqref{eq:wignerphasesimpleprec}. The constant rate of rotation around $\bm{J}$ translates into $\dot{\alpha} = \Omega_{\rm prec}$. From the closure relation $\bm{J} = \bm{L} + \bm{S}_{1}/c$, we have
\be\label{eq:betaconst}
	\cos \beta = \sqrt{1 - \left( \frac{vS_{1}^{\perp}}{G M^{2}\nu} \right)^{2}}\,.
\ee
For a given mode contribution $\calD^{\ell *}_{mm'}$, \eqref{eq:wignerphasesimpleprec} shows that we will have factors of $(m' \cos\beta - m)$ when taking derivatives.

We have now everything we need to compute the error estimates~\eqref{eq:deffom} for the transfer function $\calT^{\ell}_{mm'}$~\eqref{eq:defprectransfer} with $m' \neq 0$. Given our restrictive  assumptions of orbit-averaged leading-order PN and single-spin simple precession, the result will only be a crude order-of-magnitude estimate. Using the Fourier-domain leading-order timescales generalized for modes $h^{\rm P}_{\ell m'}$ in the text below~\eqref{eq:timescalesN} yields: \SM{[check]}
\begin{subequations}\label{eq:precfomPN}
\begin{align}
	\epsilon_{\Psi 2} &= \frac{m'}{2} \frac{5\nu}{96} \Lambda^{2} \xi^{2} (m' \cos\beta - m)^{2} v^{-1} \,, \\
	\epsilon_{A 1} &=  \frac{(7 - 2\kappa_{\ell m'})\nu}{6} \Lambda \xi |m' \cos\beta - m| v^{2} \,, \\
	\epsilon_{A 2} &= \frac{(7 - 2\kappa_{\ell m'}) (13 - 2\kappa_{\ell m'}) \nu^{2}}{72} \Lambda^{2} \xi^{2} (m' \cos\beta - m)^{2} v^{4} \,,
\end{align}
\end{subequations}
where $v = (G M \omega/c^{3})^{1/3}$ is related to the Fourier frequency $f$ by $v = (2/m')^{1/3} (GM \pi f/c^{3})^{1/3}$ for a mode $h^{\rm P}_{\ell m'}$. It is worth noting that $\epsilon_{\Psi 2}$ has an overall frequency scaling of $v^{-1}$, formally at $-0.5$PN order, which means that the relative size of this correction grows towards smaller frequencies, away from merger. Remember however that the quantities $\epsilon$ are meant to measure errors relatively to the leading order, and the opening angle of the precession cone, giving the overall normalization for precession effects in the waveform, also goes to 0 as $v$ in that limit. The amplitude error estimates $\epsilon_{A1}$, $\epsilon_{A2}$, by contrast, have the more usual behaviour of PN corrections growing towards merger. The geometric factors $(m' \cos\beta - m)$ show that mode contributions with a larger $|m-m'|$ are harder to model, again in a relative sense, but those contributions are however suppressed in amplitude (see discussion below~\eqref{eq:defprechIsum}).

The overall $\nu$ scaling indicates a better separation of timescales for larger mass ratio systems. These expressions also show that, since the total angular momentum appears as a factor in $\Omega_{\rm prec}$, higher-order corrections will be larger for spin-aligned systems than for anti-aligned spins, an effect that becomes significant for large mass ratios. Note that the regime where $\xi$ gets close to 0 corresponds to the transitional precession range~\cite{Apostolatos+94}, with the spin of the primary compensating the orbital angular momentum, and our analysis based on simple precession is not valid anymore.

\SM{[Describe the difference in morphology of the precession in the three cases ?]}
We can somewhat complement this picture for the post-merger precession by considering the accelerated frame rotation rate described by the model~\eqref{eq:OmegaframeQNM} and illustrated in Fig.~\ref{fig:precmodel}. We show in Fig.~\ref{fig:QNM} the dependency of the QNM frequencies $\omega_{220}^{\rm QNM}$, $\omega_{210}^{\rm QNM}$ with the final spin of the remnant black hole $\chi_{f}$, together with the ratio $\Omega_{\rm frame} / \omega^{\rm QNM}_{220}$. This ratio is characteristic of the separation of timescales, in the ringdown regime, between the phase of the modulation~\eqref{eq:wignerphasesimpleprec} and the P-frame waveform phase, and increases monotonically with $\chi_{f}$, reaching $\sim 0.4$ for $\chi_{f} \rightarrow 1$. We cannot translate readily the time-domain separation of timescales in the ringdown regime to the Fourier-domain error measures defined in~\eqref{eq:deffom}, as the time-to-frequency correspondence~\eqref{eq:deftf} does not reach times beyond the merger, as shown by Fig.~\ref{fig:tf}. However, a faster frame rotation will yield a more extended Fourier transform of the modulation~\eqref{eq:defG}, and will be more challenging to accomodate with the formalism of Sec.~\ref{sec:formalism}.

To go beyond the above order-of-magnitude picture, we now present a numerical computation of the error estimates~\eqref{eq:deffom} for our post-merger extended precession model presented in Sec.~\ref{subsec:precmodel}, for the three examples summarized in Table.~\ref{tab:precexamples}. Here and in the following we consider only the $h^{\rm I}_{2 m}$ mode contributions induced by $h^{\rm P}_{22}$ for positive frequencies, knowing that the ones induced by $h^{\rm P}_{2,-2}$ for negative frequencies can be deduced using~\eqref{eq:hIlmposnegfreq}. We use analytic derivatives of the PhenomD phase and amplitude for the Fourier-domain based timescales $\Tf$, $T_{A1}$ and $T_{A2}$, and numerical derivatives for the time-domain modulation.

%The first two cases, $++$ and $\perp\perp$, have \SM{[come back to this after having checked trajectories]} the usual behaviour of precession on a cone, with a larger opening angle for $\perp\perp$ due to the larger in-plane spin components. After merger, the case $++$, with a lerger remnant spin, displays a much faster frame rotation $\Omega_{\rm frame}$ than the other cases, as shown by Table~\ref{tab:precexamples}. The case $--$ is qualitatively different, close to the transitional precession regime, with a large and varying $\beta$ and a significant change in the direction of $\bm{J}$.

The results are shown in Fig.~\ref{fig:fomprec}, using the merger frequency and the ringdown frequency, shown by the vertical lines, to give an idea of the separation between the inspiral and post-merger phases. We can take $\epsilon \sim 1$ as an order-of-magnitude indication of the breakdown of a perturbative treatment. An important point about Fig.~\ref{fig:fomprec} is that the error estimates are relative for each mode, thus higher $|m-m'|$ modes, which are found to be the hardest to model precisely, can be very subdominant in the final waveform. 

The magnitude of the error estimates for $++$ and $\perp\perp$ is roughly in agreement with the PN-inspired computation~\eqref{eq:precfomPN} above for the inspiral part of the signal \SM{[check if the PN is quantitatively ok -- overlay it on the plot ?]} \SM{[Pb: scaling of FOM might be dominated here by SS oscillations, notably for 2nd der. ? To check - check also geometric scaling from mode to mode]}, with $\epsilon_{\Psi 2}$ showing a negative slope $v^{-1}=(Mf)^{-1/3}$, and with a hierarchy between modes due to the factors of $|m' \cos \beta - m|$. The case $--$ departs from the simple precession picture, and~\eqref{eq:precfomPN} does not apply. Before merger, we see that amplitude corrections are within the preturbative regime, contrarily to $\epsilon_{\Psi 2}$ that can exceed $1$ for subdominant modes in the $++$ and $\perp\perp$ cases. The amplitude corrections are found to be in the perturbative regime for all cases during the inspiral, however both the $++$ and $\perp\perp$ cases show a sharp increase of $\epsilon_{A1}$, $\epsilon_{A2}$ post-merger. The case $--$, with its small remnant spin and mild post-merger frame rotation, shows no such increase.

Overall, the conclusion to be drawn of Fig.~\ref{fig:fomprec} is that for the $++$ and $\perp\perp$ cases we expect the perturbative approach to be applicable only during the inspiral, with a possible breakdown for the post-merger phase, especially for the case $++$ with its fast post-merger frame rotation. The higher $|m-m'|$ modes are expected to be more challenging for the perturbative formalism, including during the inspiral. To make this picture quantitative, we need a full comparison of the signals processed at different orders of approximation against a numerical Fourier transform, which will be presented in Sec.~\ref{subsec:precerror} below.

\begin{figure*}
  \centering
%  \includegraphics[width=.32\linewidth]{plots/fom_pp_Psi2.pdf}
%  \hspace{0cm}
%  \includegraphics[width=.32\linewidth]{plots/fom_pp_A1.pdf}
%  \hspace{0cm}
%  \includegraphics[width=.32\linewidth]{plots/fom_pp_A2.pdf}
%  %
%  \includegraphics[width=.32\linewidth]{plots/fom_++_Psi2.pdf}
%  \hspace{0cm}
%  \includegraphics[width=.32\linewidth]{plots/fom_++_A1.pdf}
%  \hspace{0cm}
%  \includegraphics[width=.32\linewidth]{plots/fom_++_A2.pdf}
%  %
%  \includegraphics[width=.32\linewidth]{plots/fom_--_Psi2.pdf}
%  \hspace{0cm}
%  \includegraphics[width=.32\linewidth]{plots/fom_--_A1.pdf}
%  \hspace{0cm}
%  \includegraphics[width=.32\linewidth]{plots/fom_--_A2.pdf}
  \includegraphics[width=.98\linewidth]{plots/precfom_py.pdf}
  \caption{Error estimates of the approximation as defined in~\eqref{eq:deffom} for the precession modulations, for the three cases $++$, $\perp\perp$ and $--$ listed in Table~\ref{tab:precexamples}. The thick and thin vertical lines show, respectively, the merger frequency and the asymptotic ringdown frequency. The colors correspond to different values of $m$ in $\calT^{2}_{m 2}$ as defined in~\eqref{eq:defprectransfer}. We show the three error estimates $\epsilon_{\Psi 2}$, $\epsilon_{A1}$, $\epsilon_{A2}$ defined in~\eqref{eq:deffom}, and the range $\epsilon \gtrsim 1$ indicates a breakdown of the formalism of Sec.~\ref{sec:formalism}. \SM{[overlay PN simple precession ?]}}
  \label{fig:fomprec}
\end{figure*}

%\begin{figure*}
%  \centering
%  \includegraphics[width=.48\linewidth]{plots/precfom22caseI.pdf}
%  \hspace{0.2cm}
%  \includegraphics[width=.48\linewidth]{plots/precfom21caseI.pdf}
%  %
%  \includegraphics[width=.48\linewidth]{plots/precfom22caseII.pdf}
%  \hspace{0.2cm}
%  \includegraphics[width=.48\linewidth]{plots/precfom21caseII.pdf}
%  \caption{Figures of merit of the approximation as defined in~\eqref{eq:deffom} for the precession modulations. The precessing-frame waveform is taken to be the $22$ mode of an equal-mass, non-spinning system. The two upper panels show the case where the frame rotation is freezed at merger (Case I), while the two lower panels show the case where it rotates after merger (Case II) (see Sec.~\ref{subsec:precmodel} for their definitions). The left panels show the figures of merit for $F_{22, 22}$, and the right panels for $F_{22, 21}$. The vertical line represents the frequency at merger. The amplitude corrections become non-negligible after merger, and become large enough to challenge a perturbative treatment in Case II for $F_{22,21}$ (notice the change of scale for the lower right panel).}
%  \label{fig:TfTA}
%\end{figure*}

%%%%%%%%%%%%%%%%%%%%%%%%%%%%%%%%%%%%

\subsection{Direct convolution approach for the merger-ringdown phase}
\label{subsec:trigopoly}

\begin{figure*}
  \centering
%  \includegraphics[width=.48\linewidth]{plots/trig_amp.pdf}
%  \hspace{0.2cm}
%  \includegraphics[width=.48\linewidth]{plots/trig_phase.pdf}
  \includegraphics[width=.98\linewidth]{plots/trigopoly_py.pdf}
  \caption{Amplitude (left panel) and phase (right panel), compared to its trigonometric polynomial representation for the three cases listed in Table~\ref{tab:precexamples}. The thick and thin vertical lines indicate the frequency of the merger and ringdown (the QNM frequency) respectively. The continuous line show the target signal $\tilde{h}(f)$, while the dashed line shows the artificially symmetrized trigonometric polynomial~\eqref{eq:hsymtrigo}. The dots show the discrete samples entering~\eqref{eq:ykDFTprec}. The lower panels show the amplitude and phase residuals of~\eqref{eq:hsymtrigo} compared to the oringial $\tilde{h}(f)$.}
  \label{fig:trigopoly}
\end{figure*}

\SM{[possibly move this to the formalism section ? but plots for our three examples]}

\SM{[say that another possibility would be to go back to time-domain and FFT for the limited-length near-merger part of the signal]}

\SM{[expand motivation: sharp feature TD $\rightarrow$ smooth in FD $\rightarrow$ hope to have a cheap trigo representation of FD merger range]}

As shown by Fig.~\ref{fig:fomprec}, the faster evolution of the modulation functions in the post-merger phase and the resulting weaker separation of timescales can be expected to be challenging for the perturbative formalism layed out in Sec.~\ref{sec:formalism}. Motivated by this forecasted shortcoming of the Taylor-like expansion approach, here we investigate an alternative way of handling the merger-ringdown part of the signal.

In the correspondence between the time and Fourier domain, sharp features in the time domain map to extended features in the Fourier domain, and vice versa. The merger and ringdown part of the Fourier-domain signal extends over a wide range of frequencies, while it corresponds to a short interval of times, much shorter than the inspiral phase. One approach would be to separate the waveform between the inspiral and merger phase, build a Fourier-domain model for the inspiral part only, to which the formalism of Sec.~\ref{sec:formalism} could be applied, while the merger and ringdown part could be handled by a direct DFT \SM{[acronym]}, which would be a cheap operation on a limited range in time. Although this approach should be the most robust, here we will use an alternative, simpler to implement and keeping the setting of a Fourier-domain precessing-frame waveform combined with a time-domain precession modulation.

As argued in Sec.~\ref{subsec:precdef}, the support of the convolution~\eqref{eq:precconvolution} will be mainly one-sided towards the high-frequency part of $\tilde{h}(f)$, which is featureless and slowly varying as a function of $f$. Taking advantage of this, we will adopt a one-sided trigonometric polynomial representation for $\tilde{h}(f)$. For frequencies high enough that the support of the convolution~\eqref{eq:precconvolution} does not extend beyond the range covered by this trigonometric representation, the result will be obtained directly as a DFT \SM{[acronym]} with a limited number of samples. The limitation to the high-frequency range is crucial here: the Fourier-domain amplitude and phase diverge as $f^{-7/6}$ and $f^{-5/3}$ respectively for $f \rightarrow 0$, and our procedure would require a much finer sampling to cover part of the inspiral, going back to being equivalent to an inverse DFT of the full signal if we were to cover all frequencies.

We consider the high-frequency part of the signal above some frequency $f_{0}$, above which the signal has limited amplitude and phase evolution, up to some maximal frequency $f_{\rm max}$ where the Fourier-domain amplitude of the signal has decayed to a negligible level. In practice, we define $f_{\rm knee}$ as the peak of $f^{2}A(f)$, representing the onset of the decay in amplitude, and roughly corresponding to $\omega_{22}^{\rm QNM}/\pi$. We set $f_{0}\equiv 2/3f_{\rm knee}$, which is close in practice to the merger frequency, and $f_{\rm max}$ is chosen so that the amplitude is $10^{-4}$ of the amplitude at $f_{\rm knee}$. We also eliminate a constant and a linear term in the phase by choosing another frequency central to the high-frequency range we want to represent, which we take to be $f_{p} = f_{\rm knee}^{2/3} f_{\rm max}^{1/3}$ with corresponding time $t_{p}\equiv \tf(f_{p})$. Note that the method should only be weakly sensitive to the precise choice of $f_{p}$ and $t_{p}$.

Instead of tapering the signal to $0$ at $f_{0}$, to limit the deviation from the original signal we take advantage of the one-sidedness and only flatten the amplitude\footnote{In practice, this is implemented as the discrete integral of a cosine window function on just the first two samples. We also taper to 0 the last three samples before $f_{\rm max}$, and $0$-pad by a factor of 2.}, and artificially symmetrize the signal. To ensure continuity, this artificial symmetrization to a fictitious range $f\in [f_{0} - (f_{\rm max} - f_{0}), f_{0}]$ is done by imposing symmetric amplitudes and phases about $f_{0}$. Defining $\Delta f \equiv 2 (f_{\rm max} - f_{0})$, we write
\begin{widetext}
\be\label{eq:defhsym}
	\tilde{h}_{\rm sym}(f) = 
	\begin{cases} 
		\exp\left[i \Psi(f_{0}) - 2i\pi (f-f_{0}) t_{p} \right] \tilde{h}(f) \,,  &\text{ for } f \in [f_{0}, f_{0} + \Delta f /2] \\
	\tilde{h}_{\rm sym}(2 f_{0} - f)^{*} \,,  &\text{ for } f \in [f_{0} - \Delta f/2, f_{0}]
	\end{cases}
\ee
\end{widetext}

Next, we build a a trigonometric polynomial representation of $\tilde{h}_{\rm sym}$, a construction intimately related to the DFT, that we recall in App.~\ref{app:notation}. Over the frequency range $f\in [f_{0}, 2f_{\rm max} - f_{0}]$, we can approximate
\be\label{eq:hsymtrigo}
	\tilde{h}(f) \simeq \tilde{h}_{\rm sym } (f) \simeq \sum\limits_{k=-M}^{+M} (-1)^{k} c_{k} e^{2i\pi k \frac{f-f_{0}}{\Delta f}} \,,
\ee
where the factor $(-1)^{k}$ comes from the fact that $f_{0}$ is here at the center of the interval. The coefficients $c_{k}$ are built following the rules~\eqref{eq:ckyk} from the inverse DFT coefficients
\be\label{eq:ykDFTprec}
	y_{k} = \frac{1}{N} \sum\limits_{j=0}^{N-1} \tilde{h}_{\rm sym}\left( f_{0} + \frac{2j - N}{N} \Delta f \right) \omega^{jk} \,.
\ee
Taking the point of view of an interpolation problem, Fig~\ref{fig:trigopoly} shows the accuracy of this representation of the high-frequency part of the signal, by comparing the orginal $\tilde{h}(f)$ to its  trigonometric-polynomial representation~\eqref{eq:hsymtrigo}, for the three cases listed in Table~\ref{tab:precexamples}. We see that we can achieve a good agreement already for 32 samples (128 counting the symmetrization and 0-padding) \SM{[check this number]}. Errors in the phase $\Psi$ grow towards high frequencies because they are essentially errors in a relative sense, and amplitudes are decaying in this region.

The crucial point in this approach is that the convolution integral in~\eqref{eq:precconvolution} has support mainly on $f'<0$, as discussed in Sec.~\ref{subsec:precdef}, which means that when we try to compute the transfer function $\calT(f)$ at a given frequency $f$, we only need the trigonometric-polynomial representation~\eqref{eq:hsymtrigo} to be accurate for frequencies $>f$ in Fig.~\ref{fig:trigopoly}. Thus, the trigonometric representation of $\tilde{h}(f)$ can be used to compute the convolution almost all the way down to $f_{0}$, effects of the tapering aside. However, this statement is tied to our restriction to the $22$-mode, and extending the method to a precessing-frame waveform that includes more modes $h^{\rm P}_{\ell m}$ will require care.

When inserting this representation~\eqref{eq:defhsym} and~\eqref{eq:hsymtrigo} into~\eqref{eq:precconvolution}, we obtain
\begin{align}\label{eq:resultdirectconvol}
	\tilde{s}(f) &= e^{-i \Psi(f_{0})} e^{2i\pi (f-f_{0}) t_{p}} \nn\\
	& \qquad \cdot\sum\limits_{k=-M}^{+M} (-1)^{k} c_{k} e^{2i\pi k \frac{f-f_{0}}{\Delta f}} F(t_{p} + k\delta t) \,, 
\end{align}
where we defined the time sampling $\delta t \equiv 1/\Delta f$. We see that we are left with a DFT-like expression to compute from $N+1$ time samples of the modulation function $F$, centered around $t_{p}$. We see that, apart from the conditioning described above with the artificial symmetrization, this is analogous to a DFT of the product of the modulation with the time-domain signal obtained through an inverse DFT. 

In terms of computational performance, the implementation of this approach is expected to have a reasonable cost. In the following, we will use $M=64$ samples ($32$ useful samples before 0-padding, $N=128$ samples in total for the artificially symmetrized signal). The computation of~\eqref{eq:ykDFTprec} and~\eqref{eq:resultdirectconvol} amounts to two DFT operations, implemented through an FFT/IFFT, and is done only once for a given waveform and modulation function. The number of samples is of the same order of magnitude as the one required to represent the Fourier-domain waveform with an interpolating cubic spline for its amplitude and phase (a few hundreds for the full frequency band, see e.g~\cite{Puerrer14}), thus the cost should be comparable to the Taylor-like approach presented in Sec.~\ref{subsec:executivesummary}.

%%%%%%%%%%%%%%%%%%%%%%%%%%%%%%%%%%%%

\subsection{Error control for the Fourier-domain modulation}
\label{subsec:precerror}

\SM{[update to $10 M_{\odot}$ and $10\mathrm{Hz}$ ?]}

We now assess the accuracy of the transfer function computation, applying the formalism of Secs.~\ref{subsec:executivesummary} and~\ref{subsec:trigopoly} to the three examples listed in Table~\ref{tab:precexamples}, and comparing the result to a reference numerical computation.

To numerically compute the reference transfer functions, we first have to perform an IFFT of our Fourier-domain P-frame waveform, to obtain $h^{\rm P}_{22}$ as a function of time. To mitigate the effect of the necessary tapering of the waveform when computing this numerical inverse Fourier transform, we apply a Planck-window tapering on the range $f\in 16-20 \Hz$ for a total mass of $M=20 \Msol$. In parallel, we generate an SEOB waveform with the appropriate length in time. Both are aligned to peak at $t=0$, we build the post-merger extended modulation functions as explained in Sec.~\ref{subsec:precmodel}, compute the inertial-frame modes $h^{\rm I}_{2m}$ following~\eqref{eq:wignerrot} before computing their Fourier-domain counterparts $\tilde{h}^{\rm I}_{2m}$ with an FFT. The Fourier-domain transfer functions are then computed using~\eqref{eq:defprectransfer}.

The figures Fig.~\ref{fig:precerrors++}, Fig.~\ref{fig:precerrorspp} and Fig.~\ref{fig:precerrors--} show the result for the cases $++$, $\perp\perp$ and $--$ of Table~\ref{tab:precexamples} at three successive approximations, following the notation of~\eqref{eq:summaryNA}: $\{N:0,A:0, \mathrm{No \; Conv.}\}$, which is simply the leading-order transfer function~\eqref{eq:transferlocal}, ignoring all the corrections; $\{N:3,A:2, \mathrm{No \; Conv.}\}$, which incorporates both the phase corrections~\eqref{eq:stencilfresnel}, using a stencil size $N=3$, and the amplitude corrections up to second order in~\eqref{eq:summaryNA}; and $\{\Psi:2,A:2,\text{conv}\}$, which is the same as the previous for the inspiral but uses the convolution formalism of Sec.~\ref{subsec:trigopoly} to cover the high frequency range, with a smooth transition in the shaded range. The panels show the Fourier-domain amplitude for each of the modes, both exact and from the reconstruction, the relative errors in amplitude, and the errors in phase. The errors here are relative to each mode, not to the dominant mode. Thus, the amplitude plots importantly allow to visualize the mode hierarchy, giving an idea of the impact of relative errors in the subdominant modes on the full waveform.

The $++$ case, shown in Fig.~\ref{fig:precerrors++}, is the most challenging. As expected from the analysis of Sec.~\ref{subsec:sizecorrPrec}, the presence of strong aligned spins and a large spin of the remnant degrades the separation of timescales, and the reconstruction shows large relative errors, at least in subdominant mode contributions. Applying the higher-order corrections of Sec.~\ref{subsec:executivesummary} does improve the accuracy in the inspiral, but the most difficult modes $m=-1$ and $m=-2$ still show large errors. We also find that for this case higher-order corrections do not reduce the errors in the merger-ringdown region, consistently with the breakdown of the perturbative formalism indicated by Fig.~\ref{fig:fomprec}. Using the convolution treatment improves the main modes $m=2$, $m=1$, but not the most challenging modes $m=-1$ and $m=-2$, which can be seen to depart from the perturbative treatment before the range covered by the convolution and for which the convolution~\eqref{eq:precconvolution} extends to very high frequencies where our trigonometric polynomial representation of the signal is not accurate (see Fig.~\ref{fig:trigopoly}). Note however that, as will be shown below by unfaithfulness computations, these large relative errors for the subdominant modes do not affect much the full waveform.

The $\perp\perp$ case is shown in Fig.~\ref{fig:precerrorspp}. Errors are increasing to larger $|m-m'|$, higher-order corrections improve the reconstruction but are unsufficient for the merger-ringdown, and the convolution works for all modes on the frequency range where it is applied. The errors at the very high end of the frequency band occur when the overall amplitudes are low, and are therefore unimportant. 

In the case $--$, shown in Fig.~\ref{fig:precerrors--}, the mode hierarchy is not respected, as the frame trajectory does not quite follow the picture of simple precession, and the estimates~\eqref{eq:precfomPN} do not apply. The precession velocity is much milder, both in the inspiral and in the merger-ringdown range, thanks to a low remnant spin, and the separation of timescales is better, as shown in Fig.~\ref{fig:fomprec}. This leads to smaller errors, and, apart from some amplitude errors in the merger-ringdown region, we find that in this case even the leading-order treatment gives good results.

In order to illustrate what these errors really mean for the analysis of GW signals, we also compute the unfaithfulness (or mismatch) between various orders of approximation and the exact, numerical result. This unfaithfulness figure, although giving a simplified view of waveform inaccuracies, is commonly used to quantify disagreements between template families and to compare them to numerical relativity waveforms. For real signals $h_{1}$ and $h_{2}$, one introduces the usual noise-weighted scalar product~\cite{CF94}
\be\label{eq:defoverlap}
	\left( h_{1} | h_{2} \right) = 4\text{Re} \int_{0}^{+\infty} \ud f \frac{\tilde{h}_{1}(f) \tilde{h}_{2}^{*}(f)}{S_{n}(f)} \,,
\ee
with $S_{n}(f)$ the noise power spectral density, for which we use the aLIGO ZDHP noise curve~\cite{LIGOProspects13}.

To define the unfaithfulness, different prescriptions are possible, taking into account or not th edetector response and optimizing over different sets of parameters. Here, we will use a simple prescription~\cite{Blackman+17a} applied directly at the level of $h_{+}$, $h_{\times}$, optimizing over time, phase, and polarization.

For spin-aligned waveforms, the unfaithfulness is customarily defined as the normalized overlap~\eqref{eq:defoverlap} above optimized over the time at coalescence and global phase of the signal. In the precessing case, different prescriptions coexist, differing by the extrinsic parameters that are optimized over. Here we adopt the same prescription that was used in the construction of a numerical relativity surrogate model for precessing waveforms~\cite{} \SM{[R]}. We refer to the Appendix.~\SM{[R]} of~\cite{} for more details \SM{[complete a bit here]}.

Here we take our illustrative model of Sec.~\ref{} \SM{[R]} as the reference waveform, thus ignoring the inaccuracy introduced by our approximate treatment of post-merger precession and focusing only on the passage from time-domain to Fourier-domain. We show the unfaithfulness obtained for 3 inclination angles between the line-of-sight and the direction of the final $\bm{J}$, $0$ (face-on), $\pi/3$ and $\pi/2$ (edge-on), using different approximations (note that we consider more options than for the preceding error plots). The total mass ranges from $20\Msol$ (where the inspiral dominates) \SM{[to be updated to $10 \Msol$]} to $400\Msol$ (where the merger-ringdown phase dominates). The symbol $\Psi:n$ with $n=0$ or $n=2$ means that the phase corrections~\eqref{eq:Fresnelstencil} are either ignored or included with a stencil size $N=3$. The symbol $A:n$ with $n=0,1,2$ indicates the order at which the amplitude corrections in~\eqref{eq:transferfinal} are included. Additionally, for all these approximation orders we either apply the perturbative formalism of Sec.~\ref{sec:formalism} to the whole frequency band (symbol ``no conv''), or treat the high-frequency part with the convolution formalism of Sec.~\ref{subsec:trigopoly} (symbol ``conv''). 

Note that the order of approximation $\{\Psi:0,A:0\}$ corresponds to the leading-order formula~\eqref{eq:transferlocal}, which is close (but not identical; see the discussion in Sec.~\ref{} \SM{[R]}) to the PhenomP~\cite{} \SM{[C]} treatment. The order of approximation $\{\Psi:2,A:0\}$ is equivalent to formalism of Ref.~\cite{KCY14} with a stencil order of $N=3$, except that the latter formalism is based on an SPA representation of the P-frame waveform, which we bypass by keeping to the Fourier-domain, as discussed above in Sec.~\ref{} \SM{[R]}.

For the $++$ case, the face-on computation (where only the $h^{\rm I}_{22}$ mode contributes) shows good agreement even at leading order. By contrast, with $\pi/3$ or $\pi/2$ inclination the mismatch reaches $10^{-2}$ for high masses, even with the perturbative corrections. Using the convolution limits the mismatch, and the inclusion of amplitude corrections makes a clear improvement.

For the $\perp\perp$ case, both the perturbative corrections and the convolution make a clear improvement. One can see clearly that the convolution becomes unimportant at low masses, as the inspiral dominates here.

The $--$ case does not show the same behaviour with inclination as the other ones, as we measure the inclination angle with respect to the final direction of $\bm{J}$, which moves during the inspiral. The mismatches are small in this case.

Overall, we find that all the corrections we investigated, perturbative in phase, perturbative in amplitude, and the convolution treatment, are relevant in lowering the unfaithfulness. The only exception are second-order amplitude corrections that make little difference, and can even give worse results than first-order in some cases. Using all the tools at our hands, we are able to keep the unfaithfulness of our three example waveforms below $2.10^{-3}$ for all masses and inclinations.

\begin{figure*}
  \centering
%  \includegraphics[width=.32\linewidth]{plots/precerrors_++_00_noconv_col.pdf}
%  \hspace{0.cm}
%  \includegraphics[width=.32\linewidth]{plots/precerrors_++_32_noconv_col.pdf}
%  \hspace{0.cm}
%  \includegraphics[width=.32\linewidth]{plots/precerrors_++_32_conv_col.pdf}
  \includegraphics[width=.98\linewidth]{plots/precerror_++_py.pdf}
  \caption{Amplitude of the inertial-frame modes $h^{I}_{2m}$ and errors in the Fourier-domain transfer fonctions $\calT^{2}_{m2}$ for the configuration $++$. The columns show the orders of approximation $\{\Psi:0,A:0\}$, $\{\Psi:2,A:2\}$, $\{\Psi:2,A:2,\text{conv}\}$. \SM{[Compact the columns to share the x axis, add vertical line for merger frequency and shaded area for stitching pert/conv]} \SM{[check: where is the phase of the orange, mode 20 ?]} \SM{[add vertical lines for merger and RD freqs]} \SM{[scaling of the amplitude ?]}}
  \label{fig:precerrors++}
\end{figure*}

\begin{figure*}
  \centering
%  \includegraphics[width=.32\linewidth]{plots/precerrors_pp_00_noconv_col.pdf}
%  \hspace{0.cm}
%  \includegraphics[width=.32\linewidth]{plots/precerrors_pp_32_noconv_col.pdf}
%  \hspace{0.cm}
%  \includegraphics[width=.32\linewidth]{plots/precerrors_pp_32_conv_col.pdf}
  \includegraphics[width=.98\linewidth]{plots/precerror_pp_py.pdf}
  \caption{Amplitude of the inertial-frame modes $h^{I}_{2m}$ and errors in the Fourier-domain transfer fonctions $\calT^{2}_{m2}$ for the configuration $\perp\perp$. The columns show the orders of approximation $\{\Psi:0,A:0\}$, $\{\Psi:2,A:2\}$, $\{\Psi:2,A:2,\text{conv}\}$. \SM{[Compact the columns to share the x axis, add vertical line for merger frequency and shaded area for stitching pert/conv]} \SM{[add vertical lines for merger and RD freqs]}}
  \label{fig:precerrorspp}
\end{figure*}

\begin{figure*}
  \centering
%  \includegraphics[width=.32\linewidth]{plots/precerrors_--_00_noconv_col.pdf}
%  \hspace{0.cm}
%  \includegraphics[width=.32\linewidth]{plots/precerrors_--_32_noconv_col.pdf}
%  \hspace{0.cm}
%  \includegraphics[width=.32\linewidth]{plots/precerrors_--_32_conv_col.pdf}
  \includegraphics[width=.98\linewidth]{plots/precerror_--_py.pdf}
  \caption{Amplitude of the inertial-frame modes $h^{I}_{2m}$ and errors in the Fourier-domain transfer fonctions $\calT^{2}_{m2}$ for the configuration $--$. The columns show the orders of approximation $\{\Psi:0,A:0\}$, $\{\Psi:2,A:2\}$, $\{\Psi:2,A:2,\text{conv}\}$. \SM{[Compact the columns to share the x axis, add vertical line for merger frequency and shaded area for stitching pert/conv. Check the post-merger frame trajectory for - -.]} \SM{[add vertical lines for merger and RD freqs]}}
  \label{fig:precerrors--}
\end{figure*}

%\begin{figure*}
%  \centering
%  \includegraphics[width=.48\linewidth]{plots/precerrorA22caseI.pdf}
%  \hspace{0.2cm}
%  \includegraphics[width=.48\linewidth]{plots/precerrorPsi22caseI.pdf}
%  %
%  \includegraphics[width=.48\linewidth]{plots/precerrorA21caseI.pdf}
%  \hspace{0.2cm}
%  \includegraphics[width=.48\linewidth]{plots/precerrorPsi21caseI.pdf}  
%  \caption{Errors in the Fourier-domain transfer fonctions $\calT_{22,22}$ and $\calT_{22,21}$ as defined in~\eqref{eq:defTlmlmp}, for Case I and for various approximations. The black curve is the exact result provided by the FFT. The blue curve gives the leading-order, local approximation. The red curve includes quadratic corrections in phase and in amplitude. The yellow curve shows the result of the direct conolution approach of Sec.~\ref{subsec:directconvolution}. The left panels show the amplitude and the right panels the phase difference with the FFT. Note that the vertical scales are not uniform. [is it ok to show only 21 and 22 ? include the other modes which show problems ?]}
%  \label{fig:precerrorcaseI}
%\end{figure*}

%\begin{figure*}
%  \centering
%  \includegraphics[width=.48\linewidth]{plots/precerrorA22caseII.pdf}
%  \hspace{0.2cm}
%  \includegraphics[width=.48\linewidth]{plots/precerrorPsi22caseII.pdf}
%  %
%  \includegraphics[width=.48\linewidth]{plots/precerrorA21caseII.pdf}
%  \hspace{0.2cm}
%  \includegraphics[width=.48\linewidth]{plots/precerrorPsi21caseII.pdf}  
%  \caption{Errors in the Fourier-domain transfer fonctions $\calT_{22,22}$ and $\calT_{22,21}$. Conventions are the same as in Fig.~\ref{fig:precerrorcaseI} but for Case II, with a more rapid post-merger frame precession. Note that the vertical scales are not uniform. [is it ok to show only 21 and 22 ? include the other modes which show problems ?]}
%  \label{fig:precerrorcaseII}
%\end{figure*}

\SM{[say how the problems with reconstruction notably for ++ are probably caused by the convolution extending to the right and not to the left]}

\SM{[be honest: to get 1\% mismatch we need to combine high spin, high mass and high inclination]}

%\begin{figure*}
%  \centering
%  \includegraphics[width=.48\linewidth]{plots/precmmsimple.pdf}
%  \hspace{0.2cm}
%  \includegraphics[width=.48\linewidth]{plots/precmmasympt.pdf}
%  \caption{Mismatches computed between the exact FFT and the various approximations respresented in Figs.~\ref{fig:precerrorcaseI} and~\ref{fig:precerrorcaseII}. [log scale for masses ? discuss the fact that in case II, leading order is accidentally better than Psi 2]}
%  \label{fig:precmm}
%\end{figure*}

\begin{figure*}
  \centering
%  %
%  \includegraphics[width=.32\linewidth]{plots/mismatches_++_inc0.pdf}
%  \hspace{0.1cm}
%  \includegraphics[width=.32\linewidth]{plots/mismatches_++_incpi3.pdf}
%  \hspace{0.1cm}
%  \includegraphics[width=.32\linewidth]{plots/mismatches_++_incpi2.pdf}
%  %
%  \includegraphics[width=.32\linewidth]{plots/mismatches_pp_inc0.pdf}
%  \hspace{0.1cm}
%  \includegraphics[width=.32\linewidth]{plots/mismatches_pp_incpi3.pdf}
%  \hspace{0.1cm}
%  \includegraphics[width=.32\linewidth]{plots/mismatches_pp_incpi2.pdf}
%  %
%  \includegraphics[width=.32\linewidth]{plots/mismatches_--_inc0.pdf}
%  \hspace{0.1cm}
%  \includegraphics[width=.32\linewidth]{plots/mismatches_--_incpi3.pdf}
%  \hspace{0.1cm}
%  \includegraphics[width=.32\linewidth]{plots/mismatches_--_incpi2.pdf}
  \includegraphics[width=.98\linewidth]{plots/precMM_py.pdf}
  \caption{Unfaithfulness. Computed here with $f_{\rm min} = 20\mathrm{Hz}$, for masses between 20 and 400 $\Msol$, using the standard aLIGO noise curve \SM{[reference for that curve]}. \SM{[Extend computation to fmin=10 and Mmin=10]} \SM{[relate unfaithfulness levels to future detectors in the text]} \SM{[check definition of inclination: from $J$ or initial $L$ ?]} \SM{[legend is missing A:2]} \SM{[do we want to show A:2 at all ?]}}
  \label{fig:precunfaithfulness}
\end{figure*}

%%%%%%%%%%%%%%%%%%%%%%%%%%%%%%%%%%%%
%%%%%%%%%%%%%%%%%%%%%%%%%%%%%%%%%%%%

\section{Summary and Conclusions}
\label{sec:discussion}

[]

\SM{[discuss the extension of our LISA treatment for precessing signals]}

\SM{[discuss other detectors, e.g. ET]}

\SM{[we should investigate broader parameter space]}

%%%%%%%%%%%%%%%%%%%%%%%%%%%%%%%%%%%%
%%%%%%%%%%%%%%%%%%%%%%%%%%%%%%%%%%%%

\vspace{4.5mm}

\hspace{0.85in}
{\bf Acknowledgments}

\vspace{3.5mm}

[Acknowledgments]

%%%%%%%%%%%%%%%%%%%%%%%%%%%%%%%%%%%%
%%%%%%%%%%%%%%%%%%%%%%%%%%%%%%%%%%%%

\appendix

\section{Notation and conventions}
\label{app:notation}

\SM{[say a word of resampling needed at low and high f for LISA, and in general interpolation errors, appendix or refer to paper PE ?]}

\SM{[clean up part on discrete Fourier coeffs]}

The convention we will be using for the Fourier transform of a signal $h(t)$ and its inverse is
\begin{subequations}
\label{eq:defFT}
\begin{align}
	\tilde{h}(f) &= \mathrm{FT}[h](f) =  \int \ud t \, e^{+2i\pi f t} h(t) \,, \\
	h(t) &= \mathrm{IFT}[\tilde{h}](t) =  \int \ud f \, e^{-2i\pi f t} \tilde{h}(f) \,.
\end{align}
\end{subequations}
Notice that this sign convention is not the most frequently used in the literature. We chose it to ensure that, with the conventions of~\cite{BlanchetLiving}, spin-weighted spherical modes $h_{\ell m}$ with $m>0$ will have support mostly for positive frequencies. One can revert to the more usual convention by taking $f\rightarrow -f$.

The effect of a shift in time of the time-domain signal translates into a linear phase contribution added to the Fourier-domain signal. For $h_{\Delta t}(t) \equiv h(t+\Delta t)$ with $\Delta t$ a constant, in our convention we have
\be\label{eq:shifttime}
	\tilde{h}_{\Delta t} (f) = e^{-2i\pi f \Delta t} \tilde{h}(f) \,.
\ee

A useful representation of the gravitational waveform is given by its decomposition in spin-weighted spherical harmonics. The waveform emitted in the direction $(\Theta, \Phi)$, with its two polarizations $h_{+},h_{\times}$, can be decomposed as a superposition of modes as~\cite{Thorne80}
\be\label{eq:defmodes}
	h_{+} - i h_{\times} = \sum\limits_{\ell \geq 2} \sum\limits_{m=-\ell}^{\ell} {}_{-2}Y_{\ell m}(\Theta,\Phi) h_{\ell m} \,,
\ee
where the ${}_{-2}Y_{\ell m}(\Theta,\Phi)$ are spin-weighted spherical harmonics~\cite{Goldberg+67}. Conversely, the individual polarizations are obtained from the individual modes as
\begin{subequations}
\begin{align}
	h_{+} = \frac{1}{2} \sum\limits_{\ell, m} \left[ {}_{-2}Y_{\ell m}h_{\ell m} + {}_{-2}Y_{\ell m}^{*} h_{\ell m}^{*} \right] \,,\\
	h_{\times} = \frac{i}{2} \sum\limits_{\ell, m} \left[ {}_{-2}Y_{\ell m}h_{\ell m} - {}_{-2}Y_{\ell m}^{*} h_{\ell m}^{*} \right] \,.
\end{align}
\end{subequations}

For a non-precessing system, with a fixed orbital plane, the individual modes have the additional symmetry property
\be\label{eq:symmetryhlminusm}
	h_{\ell, -m} = (-1)^{\ell} h_{\ell m}^{*} \,.
\ee
Since we will work in the Fourier domain, it will be useful to write the contributions of the individual modes to the Fourier transforms of the polarizations as
\begin{subequations}\label{eq:hpcfrommodes}
\begin{align}
	\tilde{h}_{+}(f) &= \frac{1}{2} \sum\limits_{\ell \geq 2} \sum\limits_{m=-\ell}^{\ell} \left[ {}_{-2}Y_{\ell m} \tilde{h}_{\ell m}(f) + {}_{-2}Y_{\ell m}^{*} \tilde{h}_{\ell m}(-f)^{*} \right] \,, \\
	\tilde{h}_{\times}(f) &= \frac{i}{2} \sum\limits_{\ell \geq 2} \sum\limits_{m=-\ell}^{\ell} \left[ {}_{-2}Y_{\ell m} \tilde{h}_{\ell m}(f) - {}_{-2}Y_{\ell m}^{*} \tilde{h}_{\ell m}(-f)^{*} \right] \,,
\end{align}
\end{subequations}
where we used $\widetilde{h_{\ell m}^{*}}(f) = \tilde{h}_{\ell m}(-f)^{*}$. An additional approximation often used in waveform models consists in considering that the Fourier transforms $\tilde{h}_{\ell m}$ have support only on one side of the spectrum, either for positive of for negative frequencies depending on the sign of $m$. This holds in particular within the stationary phase approximation, as explained below \SM{[change here]}. With our sign convention~\eqref{eq:defFT}, this approximation reads
\begin{align}\label{eq:zeronegativef}
	\tilde{h}_{\ell m} (f) &\simeq 0 \text{ for } f<0, \; m>0 \nn\,,\\
	\tilde{h}_{\ell m} (f) &\simeq 0 \text{ for } f>0, \; m<0 \,.
\end{align}
When both and~\eqref{eq:zeronegativef} apply, \eqref{eq:hpcfrommodes} becomes simpler. For $f>0$, we have then
\be
	\tilde{h}_{+,\times} (f) = \sum_{\ell \geq 2} \sum_{m = 1}^{\ell} K^{+,\times}_{\ell m} \tilde{h}_{\ell m}(f) \,,
\ee
where we set
\begin{align}
	K^{+}_{\ell m} &\equiv \frac{1}{2} \left( {}_{-2}Y_{\ell m} + (-1)^{\ell} {}_{-2}Y_{\ell, -m}^{*} \right) \,, \nn\\
	K^{\times}_{\ell m} &\equiv \frac{i}{2} \left( {}_{-2}Y_{\ell m} - (-1)^{\ell} {}_{-2}Y_{\ell, -m}^{*} \right) \,.
\end{align}
the range $f<0$ being obtained readily, since $h_{+},h_{\times}$ are real quantities, from $\tilde{h}_{+,\times} (-f) = \tilde{h}_{+,\times}(f)^{*}$.

In the following, we will drop the mode indices and the sum over modes, and we will denote generically by $h$ an individual mode of the waveform\footnote{This notation for a generic individual mode is not to be confused with the often encountered notation $h\equiv h_{+} - i h_{\times}$.}. Note that, if we are considering real signals $s(t)$ (such as $h_{+}$, $h_{\times}$, or any detector output), we can focus  on the positive-frequencies range and reconstruct the negative frequencies as $\tilde{s}(-f) = \tilde{s}(f)^{*}$. For positive frequencies $f>0$, we will use the generic notation $h$ \SM{[not using the 'generic notation' anymore]} as
\begin{align}\label{eq:notationhsignm}
	h(t) &\equiv h_{\ell m}(t) \,, \quad \tilde{h}(f) \equiv \tilde{h}_{\ell m}(f) \text{ for } m>0 \,, \nn\\
	h(t) &\equiv h_{\ell m}^{*}(t) \,, \quad \tilde{h}(f) \equiv \tilde{h}_{\ell m}(-f)^{*} \text{ for } m<0 \,.
\end{align}
We will further decompose $\tilde{h}$ into a Fourier-domain amplitude $A$ and a phase $\Psi$ according to
\be\label{eq:defAPsi}
	\tilde{h}(f) \equiv A(f) e^{-i\Psi(f)} \,.
\ee

Since we intend to work in the Fourier domain, we will use PhenomD waveforms~\cite{Khan+15}. These phenomenological waveforms provide analytic fits for both the Fourier-domain amplitude and phase of the 22 mode, with fit coefficients that were calibrated on EOB/NR hybrids \SM{[describe smoothing across intervals ?]}. Fig.~\ref{fig:ampphase} shows an example of amplitude and phase in geometric units for an equal-mass binary. \SM{[remove that and point to appendix on interpolation errors]}

%\begin{figure*}
%  \centering
%  \includegraphics[width=.48\linewidth]{plots/Af.pdf}
%  \hspace{0.2cm}
%  \includegraphics[width=.48\linewidth]{plots/Psif.pdf}
%  \caption{Fourier-domain amplitude and phase in geometric units for an equal-mass, non-spinning system. The vertical line represents $\omega_{22}^{\rm peak}/2\pi$, the frequency corresponding to the time-domain frequency of the waveform at the peak amplitude. The starting frequency $Mf_{min} = 2.6\times 10^{-4}$ is appropriate for a 2-years observation of a $M=10^{6} M_{\odot}$ system by LISA, or for an observation by LIGO/Virgo of a $M=5.4 M_{\odot}$ system entering the instrument band at $10\Hz$. [add residuals of interpolation on $~300$ points, to illustrate the compression of Amp/Phase FD waveforms ? or show different spins/mass ratios ?] \SM{[Plot perhaps not necessary ?]}}
%  \label{fig:ampphase}
%\end{figure*}

We recall that, with our sign convention, the effect of a positive shift in time $h' (t) = h(t- \Delta t)$ is
\be
	\tilde{h}'(f) = e^{2i\pi f \Delta t} \tilde{h}(f) \,.
\ee
If we also allow for a global shift in phase of the signal, which corresponds to a rotation of the unit vector pointing towards the observer in the frame of the gravitational waves source, a constant term can also be added (note that the effect of such a rotation gives different phase changes $e^{\pm i m \delta \phi}$ for the different modes of the radiation). We see that both a constant and a linear term in the Fourier-domain phase can be adjusted by changing extrinsic parameters. Thus, as is well known, apart from the relative phases of the modes for waveforms that include higher harmonics, the intrinsic information on the phasing is contained in the second derivative of the Fourier-domain phase $\Psi(f)$. \SM{[rewrite this paragraph]}

In our convention, the link between the Discrete Fourier Transform (DFT) \SM{[check acronym]} and the trigonometric polynomial representation of a function goes as follows. For a periodic function $F(x)$ defined on $x\in [x_{0}, x_{0} + \Delta x]$, and represented by $N$ samples $x_{j} = \ov{x} + j \Delta x/N$, $j=0,\dots,N-1$, with $N$ large enough to satisfy, at least approximately, the Nyquist criterion, we can build a trigonometric interpolant $P(x)$ as
\be
	P(x) = \sum\limits_{k=-M}^{+M} c_{k} e^{2i\pi k \frac{x-x_{0}}{\Delta x}} \,,
\ee
that will satisfy the system $P(x_{j}) = F(x_{j})$ for $j=0,\dots, N-1$. Here we set $M=N/2$, assuming N is even. This trigonometric polynomial representation is equivalent to truncating the formal Fourier series, representing the full signal, to a finite order $M$. The coefficients $c_{k}$, in the full series, are defined as integrals over the interval of $F$ weighted by the appropriate exponential (as in~\eqref{eq:defcn}). In both interpretations, either the truncated approximation of the Fourier series or  the trigonometric interpolation formulation, these coefficients are related to the coefficients of the inverse DFT. If we set $\omega \equiv e^{2i\pi/N}$ and define \SM{[check convention]}
\be\label{eq:ykDFT}
	y_{k} = \frac{1}{N} \sum\limits_{j=0}^{N-1} F(x_{j}) \omega^{jk} \,,
\ee
which is the expression of the inverse DFT in our sign convention~\eqref{eq:defFT}, the coefficients $c_{k}$ are given by
\begin{align}\label{eq:ckyk}
	c_{k} &= y_{k} \text{ for } k=0,\dots, M-1 \,, \nn\\
	c_{k} &= y_{k+N} \text{ for } k=-M+1,\dots, -1 \,, \nn\\
	c_{M} &= c_{-M} = \frac{y_{M}}{2} \,,
\end{align}
where the condition $c_{M} = c_{-M}$ is an arbitrary condition enforced to match the number of degrees of freedom. In practice, a good representation of the Fourier series of the signal is achieved when the truncation order $M$ is sufficient so that the coefficients $c_{|n|\geq M}$ become negligibly small.

%%%%%%%%%%%%%%%%%%%%%%%%%%%%%%%%%%%%

\section{Wigner matrices and precessing frame}
\label{app:wigner}

In this Appendix, we summarize our conventions for the Wigner matrices.

\SM{[add O'Shaughnessy frame extraction]}
\SM{[cite other prescriptions]} \SM{[C]}

Several constructions for the radiation axis of the wa precessing frame have been proposed~\cite{} \SM{[C]}. 

\SM{[This differs from G. Faye's definition by a transposition $m\leftrightarrow m'$]}
\be\label{eq:defWignerDapp}
	\calD^{\ell}_{mm'} (\alpha, \beta, \gamma) = e^{im \alpha} d^{\ell}_{mm'}(\beta) e^{im' \gamma}\,,
\ee
\SM{[check]}
\begin{widetext}
\be\label{eq:defWignerdapp}
	d^{\ell}_{mm'}(\beta) = \sum\limits_{k=k_{\rm min}}^{k_{\rm max}} \frac{(-1)^{k}}{k!} \frac{\sqrt{(l+m)! (l-m)! (l+m')! (l-m')!}}{(l+m-k)! (l-m'-k)! (k-m+m')!} \left( \cos\frac{\beta}{2} \right)^{2\ell+m-m'-2k} \left( \sin\frac{\beta}{2} \right)^{2k-m+m'}\,,
\ee
\end{widetext}
where the boundaries of the sum, $k_{\rm min} = \mathrm{max}(0, m-m')$ and $k_{\rm max} = \mathrm{min}(\ell+m, \ell-m')$, can also be read by enforcing that the arguments of the factorials must be non-negative.

\SM{[add here explicitly the dominant-eigenvector construction of the P-frame]}

\section{Explicit expression for the stencil coefficients}
\label{app:stencil}

\begin{table}[t]
\begin{ruledtabular}\caption{Stencil coefficients $a_{N,k}^{\epsilon}$ entering the formula~\eqref{eq:stencilfresnel}, given here for $\epsilon=1$. The case $\epsilon=-1$ is obtained by complex conjugation.}\label{tab:stencil}
\begin{tabular}{c|cccccc}
	$N \backslash k$ & $0$ & $1$ & $2$ & $3$ & $4$ & $5$ \\
	\hline
	$0$ & $1$ & - & - & - & - & - \\
	$1$ & $1+i$ & $-i$ & - & - & - & - \\
	$2$ & $\frac{1+5i}{4}$ & $\frac{3-4i}{3}$ & $\frac{-3+i}{12}$ & - & - & - \\
	$3$ & $\frac{17 i-3}{18}$ & $\frac{13-7 i}{8}$ & $\frac{-5-i}{10}$ & $\frac{11 i+15}{360}$ & - & - \\
	$4$ & $\frac{185 i-69}{288}$ & $\frac{209-47 i}{120}$ & $\frac{-41 i-67}{120}$ & $\frac{251 i+147}{2520}$ & $\frac{-29 i-7}{3360}$ & \\
	$5$ & $\frac{1669 i-690}{3600}$ & $\frac{2393-135 i}{1440}$ & $\frac{-645 i-646}{1260}$ & $\frac{3295 i+831}{20160}$ & $\frac{26-345 i}{15120}$ & $\frac{429 i-115}{302400}$ \\
\end{tabular}
\end{ruledtabular}
\end{table}

In this Appendix, we give explicit expressions for the stencil coefficients entering~\eqref{eq:stencilfresnel}. In this work we used only low order stencils with $N\leq 20$, and one can trivially invert of the system~\eqref{eq:stencilsystem}, as this operation has to be done only once. One can also obtain closed-form expressions for these coefficients, thanks to the particular choice of samples at $\pm kT$ that gives to the system~\eqref{eq:stencilsystem} the form of a Vandermonde system. Defining the Vandermonde matrix $V(x_{0},\dots,x_{N})$ as $V_{ij} = (x_{j})^{i}$, setting $x_{j} = j^{2}$, and $b_{p} \equiv (-i\epsilon)^{p}(2p-1)!!$, the linear system~\eqref{eq:stencilsystem} becomes
\be
	b_{p} = \sum\limits_{k=0}^{N} V_{pk} a_{N,k}^{\epsilon} \,.
\ee
%Now, since the Vandermonde system can be seen as a reformulation of the polynomial interpolation problem, we know that $({}^{t}V^{-1})_{ij}$ is the coefficient of $X^{i}$ in the Lagrange interpolation polynomial $\prod_{k\neq j} (X-x_{k})/(x_{j} - x_{k})$. If we introduce the symmetric polynomials $\sigma$ such that
%\be	
%	\sigma(m, \{x_{1}, \dots, x_{n}\}) = \sum\limits_{1\leq i_{1}<\dots<i_{m}\leq n} x_{i_{1}}\dots x_{i_{m}} \,,
%\ee
%we obtain for the inverse of the matrix $V$
%\be
%	V^{-1}_{ij} = \frac{1}{\prod_{k\neq i} (i^{2} - k^{2})} (-1)^{N-j} \sigma(N-j, \{0,\dots,N^{2}\}\backslash \{i^{2}\}) \,.
%\ee
The expression of the inverse of a Vandermonde matrix in terms of symmetric polynomials then allows us to write the $a_{N,k}^{\epsilon}$, for every finite order $N$, as:
\begin{align}
	a_{N,k}^{\epsilon} &= \frac{1}{\prod\limits_{\substack{q=0 \\ q\neq k}}^{N} (q^{2}-k^{2})} \sum\limits_{p=0}^{N} (i\epsilon)^{p}(2p-1)!! \nn\\ & \quad \cdot \sum\limits_{\substack{ 0 \leq j_{1} < \dots < j_{N-p} \leq N \\ j_{1}, \dots, j_{N-p} \neq k}} j_{1}^{2}\dots j_{N-p}^{2}
\end{align}
The two cases $\epsilon\pm 1$ correspond simply to a complex conjugation of the coefficients $a_{N,k}^{\epsilon}$. Table~\ref{tab:stencil} gives the resulting complex rational values for these coefficients for $N\leq 5$.

\section{Precession in current Fourier-domain waveform models}
\label{app:precpreviousapproaches}

In this Appendix, we give a more detailed overview of the treatment of precession in existing waveform models that generate signals directly in the Fourier domain. Our objective is not to give an exhaustive account of the various existing models, but rather to highlight how their treatment relates to ours.

To describe previous approaches to the problem of producing precessing waveforms directly in the Fourier domain, avoiding the use of a time-domain generation followed by a FFT \SM{[check acronym]}, it is useful to introduce schematically four different approximations (dropping the mode indices):
\begin{itemize}
	\item unstable SPA: $\calT(f) \tilde{h}^{\rm P}(f) = \mathrm{SPA}\left[ \calD^{*} h^{\rm P} \right](f)$,
	\item $0^{\text{th}}$-order SUA: $\calT(f) = \calD^{*}(\tfSPA) $,
	\item Fourier-domain $0^{\text{th}}$-order SUA: $\calT(f) = \calD^{*}(\omega = \pi f ) $,
	\item SUA: $\calT(f) = \frac{1}{2} \sum\limits_{k} a_{k} \calD^{*}(\tfSPA \pm k \Tf^{\rm SPA})$.
\end{itemize}

In the first option, one applies directly the SPA, as described in Sec.~\ref{subsec:SPA}, to the product of the modulation and the signal. This is known (see e.g.~\cite{KCY13}) to lead to possible pathologies, since the prefactor $1/\sqrt{\ddot{\phi}_{\rm prec}}$ can blow up due the precessing contributions to the phase of the inertial-frame waveform.

The second option corresponds to simply multiplying the Fourier-domain signal by the modulation function evaluated at the time $\tfSPA$ as defined in~\eqref{eq:deftfSPA}. In the SUA formalism of Ref.~\cite{KCY14}, this treatment could be called the 0th order as it amounts to using a stencil reduced to a single point, i.e. using only one term with $a_{0,0} = 1$ in~\eqref{eq:stencilfresnel}. It is equivalent to the leading order of our formalism ($\{0, A:0, d:0 \}$ in the terminology of Sec.~\ref{subsec:executivesummary}), with the difference that we use a more general definition of $t_{f}$ (see~\eqref{eq:deftf}) that extends through the merger and ringdown.

While keeping implictly the same level of approximation, one can also use frequency-based expressions for the modulation, as in Refs.~\cite{LOS13, Hannam+13}. The precessing-frame evolution is then modelled using post-Newtonian expressions for the Euler angles $(\alpha^{\rm PN}, \beta^{\rm PN}, \gamma^{\rm PN})$ as functions of the orbital frequency $\omega$. Using the SPA-inspired correspondence~\eqref{eq:deftfSPA} between the orbital and Fourier-domain frequency, one then writes 
\be\label{eq:precPhenomP}
	\calT^{\ell}_{mm'}(f) = \calD^{\ell *}_{mm'}(\alpha^{\rm PN}, \beta^{\rm PN}, \gamma^{\rm PN}) \left( \omega=\pi f\right) \,.
\ee
When the SPA is valid for the underlying signal $h^{\rm P}_{\ell m}$, the condition $\omega(t) = \pi f$ is equivalent by definition to $t=t^{\rm SPA}_{f}$. In~\cite{LOS13}, only the inspiral phase is modelled, and only single-spin configurations, for which the system undergoes simple precession~\cite{Apostolatos+94, Kidder95}. The PhenomP model~\cite{Hannam+13} uses Euler angles computed in the limit of a small opening angle of the precession cone and at the spin-orbit level (see also~\cite{BBF11, MBBB13}).

In the PhenomP model~\cite{Hannam+13}, the frequency-based transfer function~\eqref{eq:precPhenomP} is used as an effective prescription covering the whole Fourier-domain frequency band, including the merger and ringdown. Note that this procedure amounts to using PN results outside of their range of validity, since in the merger-ringdown phase both the PN perturbative treatment and the SPA approximation break down. In particular, when computed as a function of time from the balance equation between emitted flux and orbital energy $\calF = -dE/dt$, the orbital $\omega^{\rm PN}(t)$ blows up and cannot be extended through merger. In practice, one finds however that the extended frequency-based expressions~\eqref{eq:precPhenomP} are mildly varying when extrapolating to higher frequencies, and this approach has been validated by comparisons to numerical relativity waveforms~\cite{Hannam+13}. \SM{[refer to PhenomPv3 in prep.]}

In the SUA formalism proposed in~\cite{KCY13, KCY14}, one extends the SPA to incorporate the leading-order correction during the inspiral phase. The transfer function is then computed by evaluating the modulation on a stencil of times centered around the time-of-frequency given by the SPA, arriving at the formula~\eqref{eq:stencilfresnel} with the times $t = \tfSPA$ and $T = \TfSPA$ as defined in~\eqref{eq:deftfSPA} and~\eqref{eq:TfSPA}. As described in Sec.~\ref{sec:formalism}, our formalism reduces to the SUA when keeping only the quadratic phase correction, but the SUA is a priori limited to the inspiral phase of the signal through the definitions of the times $t^{\rm SPA}_{f}$ and $T^{\rm SPA}_{f}$. For inspiral waveforms, Ref.~\cite{KCY14} showed that this treatment improves the accuracy with respect to the 0th-order approximation.

%%%%%%%%%%%%%%%%%%%%%%%%%%%%%%%%%%%%
%%%%%%%%%%%%%%%%%%%%%%%%%%%%%%%%%%%%

%\bibliography{ListeRef.bib}
\bibliography{references.bib}


\end{document}

