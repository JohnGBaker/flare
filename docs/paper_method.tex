\documentclass[aps,showpacs,%twocolumn,
prd,superscriptaddress,nofootinbib]{revtex4}

\usepackage{amsmath}
\usepackage{amsfonts}
\usepackage{amssymb}
\usepackage{latexsym}
\usepackage{graphicx}
\usepackage{bm}
\usepackage{color}
\usepackage{enumerate}
\usepackage{ulem}

\newcommand{\be}{\begin{equation}}
\newcommand{\ee}{\end{equation}}
\newcommand\ud{{\mathrm{d}}}
\newcommand\uD{{\mathrm{D}}}
\newcommand\calO{{\mathcal{O}}}
\newcommand\calF{{\mathcal{F}}}
\newcommand\bfx{\mathbf{x}}
\newcommand{\ov}[1]{\overline{#1}}
\newcommand{\ph}[1]{\phantom{#1}}
\newcommand{\cte}{\mathrm{cte}}
\newcommand{\nn}{\nonumber}
\newcommand{\hatk}{\hat{k}}
\newcommand{\Hz}{\,\mathrm{Hz}}
\newcommand{\sinc}{\,\mathrm{sinc}}
\newcommand{\Msol}{M_{\odot}}
\newcommand{\tf}{\tilde{t}_{f}}
\newcommand{\Tf}{\tilde{T}_{f}}

\begin{document}

\title{Modulation and delays in Fourier domain: precessing binaries and LISA-type detector response}

\author{John G. Baker}
\affiliation{Gravitational Astrophysics Laboratory, NASA Goddard Space Flight Center, 8800 Greenbelt Rd., Greenbelt, MD 20771, USA}
\author{Alessandra Buonanno}
\affiliation{Department of Physics, University of Maryland, College Park, MD 20742, USA}
\affiliation{Max Planck Institute for Gravitational Physics (Albert Einstein Institute), Am M\"uhlenberg 1, Potsdam-Golm, 14476, Germany}
\author{Philip. B. Graff}
\affiliation{Department of Physics, University of Maryland, College Park, MD 20742, USA}
\affiliation{Gravitational Astrophysics Laboratory, NASA Goddard Space Flight Center, 8800 Greenbelt Rd., Greenbelt, MD 20771, USA}
\author{Sylvain Marsat}
\affiliation{Department of Physics, University of Maryland, College Park, MD 20742, USA}
\affiliation{Gravitational Astrophysics Laboratory, NASA Goddard Space Flight Center, 8800 Greenbelt Rd., Greenbelt, MD 20771, USA}
\affiliation{Max Planck Institute for Gravitational Physics (Albert Einstein Institute), Am M\"uhlenberg 1, Potsdam-Golm, 14476, Germany}


\date{\today}

\begin{abstract}

[Abstract]

\end{abstract}

\pacs{
04.25.D-, % numerical relativity
04.70.Bw, % classical black holes
04.80.Nn, % Gravitational wave detectors and experiments
95.30.Sf, % relativity and gravitation
95.55.Ym, % Gravitational radiation detectors
97.60.Lf  % black holes (astrophysics)
}

\maketitle

%%%%%%%%%%%%%%%%%%%%%%%%%%%%%%%%%%%%

\section{Introduction}
\label{sec:intro}

[Introduction]

%%%%%%%%%%%%%%%%%%%%%%%%%%%%%%%%%%%%
%%%%%%%%%%%%%%%%%%%%%%%%%%%%%%%%%%%%

\section{Methodology}
\label{sec:methodology}

%%%%%%%%%%%%%%%%%%%%%%%%%%%%%%%%%%%%

\subsection{Fourier transform of a modulated and delayed signal}
\label{subsec:FTgeneral}

Let us consider, in general, the problem of determining the Fourier transform of a signal $h(t)$ to which we apply a varying delay $d(t)$ and a multiplication by a modulation function $F(t)$. In the case of precessing binaries, the delays will ba absent, and the modulation functions will be the time-dependent Wigner coefficients applied to rotate the waveform from a precessing frame, where it verifies a restored mode hierarchy, to an inertial frame where the observations take place. In the case of a LISA-type detector, the signal $h(t)$ will simply be the waveform in a fixed heliocentric frame, and the delays will come from the motion of each detector in the wave front, while the modulation will represent the time-variation in the detector orientation. Both can be considered as periodic functions with a period of a year, which allows for an additional rewriting in terms of Fourier series instead of Fourier transforms, as will be developed in Sec.~\ref{}. 

The convention we will be using for Fourier transform will be
%
\be\label{eq:defFT}
	\tilde{h}(f) =  \int \ud t \, e^{+2i\pi f t} h(t) \,.
\ee
%
Notice that this convention uses the opposite sign with respect to the one used in general. It was chosen to ensure that, with the conventions of~\cite{}, modes $h_{\ell m}$ with $m>0$ will have support for positive frequencies. One can revert to the more usual convention by taking $f\rightarrow -f$.

Defining
%
\be
	h_{d}(t) = h(t+d(t)) \,, \quad s(t) = F(t)h_{d}(t) \,,
\ee
%
we have
%
\be
	h_{d}(t) = \int \ud f \, e^{-2i\pi f (t+d(t))}\tilde{h}(f) \,,
\ee
%
and
%
\begin{align}
	\tilde{s}(f) &= \int \ud t \, e^{2i\pi f t} F(t)  \int \ud f' \, e^{-2i\pi f' (t+d(t))}\tilde{h}(f') \nn\\
	&= \int \ud f' \, \tilde{h}(f-f') \int \ud t \, e^{2i\pi f' t} e^{-2i\pi (f-f') d(t)} F(t) \,.
\end{align}
%
We can rewrite the last equation as a kind of convolution with a frequency-dependent Kernel, according to
%
\be\label{eq:FDkernel}
	\tilde{s}(f) = \int \ud f' \, \tilde{h}(f-f') \tilde{g}(f-f',f') \,,
\ee
%
where we introduced the frequency-dependent function of time $g(f,t)$, and its Fourier transform in the auxiliary frequency $f'$, denoted by $\tilde{g}(f,f')$, as
%
\begin{subequations}
\begin{align}\label{eq:defg}
	g(f,t) &= e^{-2i\pi f d(t)} F(t) \,, \\
	\tilde{g}(f,f') &= \int \ud t \, e^{2i\pi f' t} e^{-2i\pi f d(t)} F(t) \,.
\end{align}
\end{subequations}
%
In the case of precessing binaries, where the delays $d(t)$ will be absent, the frequency-dependence of $g$ will vanish and this function will simply reduce to the modulation $F(t)$.

In our approach, two assumptions will be crucial to the approximation:
\begin{itemize}
	\item the time scales entering the signal $h(t)$ are well separated from the ones entering the modulation and the delay, so that $\tilde{g}(f-f',f')$ has support limited to $[-f_{\rm max},f_{\rm max}]$, with $f_{\rm max}$ small compared the characteristic scale for the variation of the Fourier-domain signal;
	\item the original signal has a smooth amplitude and phase in the Fourier-domain, slowly varying over the scale $f_{\rm max}$, so that a local treatment in $f'$ of the convolution in~\eqref{eq:FDkernel} will be possible.
\end{itemize}
Although well verified for the case of LISA-type detectors, the first assumption can be a limitation of the approximation in the case of preccesing binaries, and we will carefully explore its validity in Sec.~\ref{}.

Notice that these approximations are not exactly the same as the ones usually made to justify the stationary phase approximation (thereafter SPA) for the treatment of the Fourier transform of the inspiral part of the signal. Indeed, we will exploit the smoothness of the Fourier-domain $\tilde{h}(f)$ to be able to cover the merger-ringdown part of the signal $h$, where the SPA itself breaks down.

%%%%%%%%%%%%%%%%%%%%%%%%%%%%%%%%%%%%

\subsection{Leading order: the locally linear phase approximation}
\label{subsec:LLP}

Under the assumptions above, it is possible to see the convolution in the $f'$ variable in~\eqref{eq:FDkernel} as being a local operation, allowing us to expand the signal $\tilde{h}(f-f')$ around $f$.

To this end, we will separate the signal in a Fourier-domain amplitude and phase, according to
%
\be
	\tilde{h}(f) = A(f) e^{-i\Psi(f)} \,,
\ee
%

A first leading-order approximation, which could be called the locally linear phase (LLP) approximation, is to discard the $f'$ dependence in the first argument of $\tilde{g}(f-f', f')$, to keep only the leading order in the Fourier-domain amplitude, and to expand the Fourier-domain phase to the first order, according to
%
\be
	\tilde{h}_{\rm LLP}(f-f') \simeq A(f) \exp\left[ -i\left( \Psi(f) - f' \frac{\ud \Psi}{\ud f} \right) \right] \,,
\ee
%
Here and in the rest of the paper, derivatives with respect to $f$ will always be evaluated at $f$.

Plugging this relation in~\eqref{eq:FDkernel}, we obtain
%
\begin{align}
	\tilde{s}_{\rm LLP}(f) &= \tilde{h}(f) \int \ud f' \, \exp\left[ i f' \frac{\ud \Psi}{\ud f} \right] \tilde{g}(f,f') \nn\\
	&= \tilde{h}(f) g\left( f, -\frac{1}{2\pi} \frac{\ud \Psi}{\ud f} \right) \,,
\end{align}
%
which motivates the introduction of the effective time
%
\be\label{eq:deftf}
	\tf \equiv -\frac{1}{2\pi} \frac{\ud \Psi}{\ud f} \,,
\ee
%
where the tilde here does not indicate  a Fourier transform operation but simply that this quantity is an effective time defined from Fourier-domain waveform. This definition is a generalization of the time-to-frequency correspondence at the heart of the SPA, as will be shown in Sec.~\ref{}, but it is worth noting the it is defined entirely from the Fourier-domain waveform. In particular, $\tf$ will in general not be monotonously increasing with frequency in the high-frequency part of the waveform, corresponding to the ringdown.

Using this notation, we have simply
%
\be\label{eq:LLP}
	\tilde{s}_{\rm LLP}(f) = \tilde{h}(f) g(f, \tf) \,.
\ee
%
The interpretation of this approximation is straightforward: the signal is simply multiplied by the response function evaluated at the time $\tf$. Additionally, when delays are present, the frequency-dependence of the response is evaluated directly at $f$. For precessing binary, this corresponds to the approximation used for the PhenomP approximant~\cite{}, as we will discuss in Sec.~\ref{}.

It is worth noting that a shift in time of the time-domain signal will, by construction, be appropriately propagated to~\eqref{eq:LLP}. If $h'(t) = h(t -  \Delta t)$ but the modulation and delays are left unchanged (which is relevant for space-based detectors, not for precessing binaries), $\Psi'(f) = \Psi(f) - 2\pi f \Delta t$, so that
%
\be\label{eq:LLP}
	\tilde{s}'_{\rm LLP}(f) = \tilde{h}'(f) g(f, \tf + \Delta t) \,. 
\ee
%
This observation also justifies that the first derivative of the Fourier-domain phase cannot, in general, be considered a small quantity, as it can take arbitrary values for arbitrary shifts in time of the signal. The leading order approximation corresponds to neglecting the curvature in $\Psi(f)$, hence the name Locally Linear Phase approximation.

%%%%%%%%%%%%%%%%%%%%%%%%%%%%%%%%%%%%

\subsection{General Taylor expansion in the Fourier domain}
\label{subsec:TaylorFD}

We now extend our approach beyond the LLP approximation. We will give a formal derivation of the  general formula for the higher-order corrections in~\eqref{eq:FDkernel}, and then specialize it to the special cases of interest.

We will formally Taylor-expand $\tilde{h}(f-f')$ in $f'$ as follows: we write
%
\begin{subequations}
\begin{align}
	\Psi(f-f') &= \Psi(f) + 2\pi f' \tf + \sum\limits_{p\geq 2}^{N} \frac{(-1)^{p}}{p!} {f'}^{p} \frac{\ud^{p} \Psi}{\ud f^{p}} \,, \label{eq:expandPsi}\\
	A(f-f') &= A(f) \sum\limits_{q\geq 0} \frac{(-1)^{q}}{q!} {f'}^{q} \frac{1}{A}\frac{\ud^{q} A}{\ud f^{q}} \,, \label{eq:expandA}\\
	\tilde{g}(f-f', f') &= \sum\limits_{r\geq 0} \frac{(-1)^{r}}{r!} {f'}^{r} \frac{\partial^{r} }{\partial f^{r}}  \tilde{g}(f,f') \label{eq:expandg}
\end{align}
\end{subequations}
%
This expansion will only be usable in practice when we will be able to keep only a few terms in each of these expansions. We therefore limited the expansion in $\Psi$ to a finite order $N$, to keep the formula readable. Notice that we expand $\tilde{g}(f-f',f')$ in $f'$ only in its first argument. We can also interchange the $f$-derivatives of $g$ with its Fourier transform.

If we further expand the exponentials to obtain a pure $f'$-expansion, by applying the derivative rule
%
\be
	\int \ud f'\, {f'}^{n} \frac{\partial^{m} }{\partial f^{m}}  \tilde{g}(f,f') e^{-2i\pi f' \tf} = \frac{1}{(-2i\pi)^{n}} \left( \frac{\partial^{m} }{\partial f^{m}} \frac{\partial^{n} }{\partial t^{n}} g \right)(f,\tf) \,,
\ee
%
we arrive at the formal expansion
%
\begin{align}
	\tilde{s}(f) &= \tilde{h}(f) \sum\limits_{q\geq 0} \sum\limits_{r\geq 0} \sum\limits_{k_{2}\geq 0} \dots \sum\limits_{k_{N}\geq 0} \frac{(-1)^{q}}{q!} \frac{(-1)^{r}}{r!} \prod\limits_{p=2}^{N} \frac{1}{k_{p}!}\left( \frac{(-1)^{p}(-i)}{p!} \frac{\ud^{p}\Psi}{\ud f^{p}}\right)^{k_{p}} \frac{1}{A} \frac{\ud^{q} A}{\ud f ^{q}} \frac{1}{(-2i\pi)^m} \left( \frac{\partial^{r} }{\partial f^{r}} \frac{\partial^{m} }{\partial t^{m}} g \right)(f,\tf) \,, \nn\\
	\text{where } & m = p+r+2k_{2}+\dots+N k_{N} \,.
\end{align}
%

This rather cumbersome expression is easy to simplify for special cases. For instance, if we expand only the frequency-dependence of $g$ as in~\eqref{eq:expandg}, we obtain
%
\be
	\tilde{s}(f) = \tilde{h}(f) \sum\limits_{r\geq 0} \frac{1}{(2i\pi)^{r}r!} \left( \frac{\partial^{r} }{\partial f^{r}} \frac{\partial^{r} }{\partial t^{r}} g \right)(f,\tf)
\ee
%
which interestingly looks like a Taylor expansion of $g$, but with joint derivatives in frequency and time. Similarly, the expansion~\eqref{eq:expandA} gives
%
\be\label{eq:resultdfA}
	\tilde{s}(f) = \tilde{h}(f) \sum\limits_{q\geq 0} \frac{1}{(2i\pi)^{q}q!} \frac{1}{A} \frac{\ud^{q} A}{\ud f ^{q}}  \left( \frac{\partial^{q} }{\partial t^{q}} g \right)(f,\tf)
\ee
%

Of particular interest for us will be the case where we keep only the first term in~\eqref{eq:expandPsi}, and expand the resulting exponential. Indeed, this term will be  the dominant correction beyond leading order in the applications that we will consider in Secs.~\ref{} and~\ref{}, and this particular approximation will turn out to be closely related to the SPA approach of~\cite{}. The above approach yields
%
\be\label{eq:resultdffPsi}
	\tilde{s}(f) = \tilde{h}(f) \sum\limits_{p\geq 0} \frac{1}{p!} \left( \frac{i}{8\pi^{2}}\frac{\ud^{2} \Psi}{\ud f^{2}} \right)^{p} \left( \frac{\partial^{2p} }{\partial t^{2p}} g \right)(f,\tf) \,,
\ee
%
and it is natural to define a new timescale, as a function of frequency, as
%
\be\label{eq:defTf}
	\Tf^{2} = \frac{1}{4\pi^{2}}\left| \frac{\ud^{2}\Psi}{\ud f^{2}} \right| \,.
\ee
%
The interpretation of this timescale for the inspiralling part of the signal, when the SPA applies, will be discussed in the next section. Whe can use this notation to rewrite~\eqref{eq:resultdffPsi} as
%
\begin{align}\label{eq:resultdffPsiTf}
	 \tilde{s}(f) &= \tilde{h}(f) \sum\limits_{p\geq 0} \frac{(-i\epsilon)^{p}}{2^{p}p!} \Tf^{2p} \left( \frac{\partial^{2p} }{\partial t^{2p}} g \right)(f,\tf) \,, \nn \\
	 \text{with } & \epsilon = -\mathrm{sgn}(\ud^{2}\Psi/\ud f^{2} ) \,.
\end{align}
%

%%%%%%%%%%%%%%%%%%%%%%%%%%%%%%%%%%%%

\subsection{Relation to the stationary phase approximation}
\label{subsec:linkSPA}

Of particular interest is the relation between these results and the SPA. One first writes the time-domain signal in amplitude and phase form as $h(t) = a(t) e^{-2i\varphi(t)}$, where for gravitational wave signals $\varphi$ will correspond to the orbital phase, with the orbital frequency being $\omega = \dot{\varphi}$. The factor of 2 in the phase of the wave is appropriate for the dominant 22 harmonic of the signal. The approximation then applies to signals verifying the conditions
%
\be\label{eq:conditionSPA}
	\left| \frac{\dot{a}/a}{\omega} \right| \ll 1\,, \quad \left|\frac{\dot{\omega}}{\omega^{2}} \right| \ll 1\,, \quad \left| \frac{(\dot{a}/a)^{2}}{\dot{\omega}} \right| \ll 1 \,.
\ee
%
Since the integral~\eqref{eq:defFT} defining the Fourier transform is rapidly oscillatory unless the term $2\pi f t$ cancels the evolution of $-2\varphi(t)$, its support is well centered around the point of stationary phase. This defines the time of stationary phase implicitly by the relation
%
\be\label{eq:deftfSPA}
	\omega(t_{f}) = \pi  f \,.
\ee
% 
For a chirping signal of increasing phase, $\omega>0$ and $\dot{\omega}>0$, there is a unique point of stationary phase, located in the positive frequency range $f>0$. Using the conditions above, one formally expands the signal around $t_{f}$, to quadratic order in the time. The resulting integral yields
%
\begin{subequations}
\begin{align}
	\tilde{h}_{\rm SPA}(f) &= A_{\rm SPA}(f) e^{-i\Psi_{\rm SPA}(f)} \,, \\
	A_{\rm SPA}(f) &= a(t_{f}) \sqrt{\frac{\pi}{\dot{\omega}(t_{f})}} \,, \\
	\Psi_{\rm SPA}(f) &= 2\varphi(t_{f}) - 2\pi f t_{f} + \frac{\pi}{4} \,.
\end{align}
\end{subequations}
%
For non-precessing gravitational wave signals, the SPA is a very good approximation through the inspiralling phase. Ref.~\cite{} evaluated the first correction to this approximation and found that it can be considered as a term of the fifth post-Newtonian order. However, it is important to note that the approximation breaks down when reaching the merger-ringdown part of the signal, as the rapidly varying features of the merger break the locality of the time-to-frequency relation~\eqref{eq:deftfSPA}.  

Taking the first derivative of the defining relation~\eqref{eq:deftfSPA} gives
%
\be
	t_{f} = -\frac{1}{2\pi} \frac{\ud \Psi_{\rm SPA}}{\ud f} \,,
\ee
%
which means that our generalized definition~\eqref{eq:deftf} for the effective time $\tf$ coincides with $t_{f}$ where the SPA is applicable. Similarly, if we define
%
\be
	T_{f} \equiv \frac{1}{\sqrt{2\dot{\omega}(t_{f})}} \,,
\ee
%
we have
%
\be
	T_{f}^{2} = -\frac{1}{4\pi^{2}}  \frac{\ud^{2} \Psi_{\rm SPA}}{\ud f^{2}} \,,
\ee
%
where $\ud^{2}\Psi/\ud f^{2} < 0$ in the SPA with our sign conventions. Comparing with~\eqref{eq:defTf} above, we see that our generalized time scale $\Tf$ also reduces to $T_{f}$ when the SPA applies. However, its definition extends to the merger-ringdown part of the signal, and we have to include an absolute value to allow for a possible change of sign of $\ud^{2}\Psi/\ud f^{2}$ in that regime.

We can now compare our result to the approach of Ref.~\cite{}, where the authors extend the SPA, in a formalism called the shifted asymptotic expansion (SUA). The derivation includes an expansion of the modulation in Bessel functions, a study of the displacement of the stationary phase point induced by the precession, and a resummation of the result in the form of time derivatives. The main result of~\cite{}, their equation~(34), reads exactly as~\eqref{eq:resultdffPsiTf} with the identifications $\tilde{H}_{corr}(f)\rightarrow \tilde{s}(f)/\tilde{h}(f)$, $T\rightarrow \Tf$ and $e^{-i\delta\phi} \rightarrow F$. Thus, we are able to identify the Fourier-domain approximation that the treatment of~\cite{} amounts to, namely the approximation~\eqref{eq:expandPsi} where in the Fourier-domain convolution~\eqref{} the phase of the signal is expanded to quadratic order and the amplitude is not expanded.

%%%%%%%%%%%%%%%%%%%%%%%%%%%%%%%%%%%%

\subsection{Magnitude of the corrections beyond the LLP approximation}
\label{subsec:sizecorr}

\subsubsection{Inspiral of a precessing binary}
\label{subsubsec:sizecorrprecession}

We investigate now the post-Newtonian order counting of the expansion~\eqref{eq:resultdffPsiTf} during the inspiralling phase of a precessing binary, with modulation $F$ as in~\eqref{}. Using the notation $\calO(n) = \calO(c^{-n})$, we have for quasi-cirular inspirals that $\dot{\omega} = \calO(5)$, corresponding to the radiation reaction effect at the 2.5PN order. The mode mixing will be induced by the rotation of the orbital angular momentum $\bm{L}$ around the total angular momentum $\bm{J}$, with the angular velocity for this precession $\bm{\Omega} = \calO(2)$. The relevant scaling for the terms in the series~\eqref{eq:resultdffPsiTf}, for the inspiral phase where the SPA applies, is therefore
%
\be
	\Tf^{2p} \partial_{t}^{2p} F = \calO(-1) \,,
\ee
%
meaning that the expansion becomes {\it less} well behaved for {\it lower} frequencies. The presence of a $1/p!$ gives a formal sense to the expansion if the successive derivatives of $F(t)$ are appropriately bounded, but one may have to keep a large number of terms in the sum. The approximation that consists in taking the first term in the series, the LLP approximation~\eqref{} which was in particular used for the PhenomP~\cite{} waveform model, gets {\it worse} the further away from merger the signal is. [This does not seem to have been clearly stated before, to check]. We investigate further in Sec.~\ref{} the validity of the treatment of precession used for PhenomP.

Notice that, for the amplitude derivatives in~\eqref{eq:expandA} and the higher order derivatives of the phase in~\eqref{eq:expandPsi}, the same PN scaling arguments give however the opposite, more usual behaviour that the terms beyond the first one appear as corrections becoming less and less important at lower frequencies. For instance, within the SPA we obtain
%
\begin{subequations}
\begin{align}
	\frac{\ud^{3} \Psi}{\ud f^{3}} \partial_{t}^{3} F &= \calO(6) \,, \\
	\frac{1}{A}\frac{\ud A}{\ud f} \partial_{t} F &= \calO(2) \,.
\end{align}
\end{subequations}
%

\subsubsection{Delays in the LLP approximation for LISA-type detectors}
\label{subsubsec:LISAorbitaldelay}

In the case of a LISA-type detector response, it is interesting to consider the case where the . The following will be mostly relevant for the delay induced by the motion of the constellation around the sun. Indeed, the analysis of the magnitude of the corrections beyond the LLP shows that it is the only one where the figure of merit of the approximation can exceed $10^{-1}$, due to the large baseline of the delay.

If one can neglect the signal related amplitude and phase corrections as given in~\eqref{eq:resultdfA} and~\eqref{eq:resultdffPsiTf}, but one keeps the frequency-dependence in the convolution kernel in~\eqref{}, the general form of the response reads
\begin{equation}
	\tilde{s}(f) = \tilde{h}(f) \int \ud t \, F(t) e^{-2i\pi f d(t)} \int \ud f' \, e^{2i\pi f' (t+d(t) - t_{f})} \,,
\end{equation}
where we recall that we defined $t_{f} \equiv -1/(2\pi)\ud \Psi/\ud f$. If we introduce a new time, defined implicitly by
\be
	\left. (t + d(t) - t_{f})\right|_{t=t_{f}^{d}} = 0 \,,
\ee
The above integral can be formally computed as
\begin{align}
	\tilde{s}(f) &= \tilde{h}(f) \int \ud t \, F(t) e^{-2i\pi f d(t)} \delta(t + d(t) - t_{f}) \nn \\
	&= \tilde{h}(f) F(t_{f}^{d}) \frac{e^{-2i\pi f d(t_{f}^{d})}}{1+\dot{d}(t_{f}^{d})}
\end{align}
In the case of the delay associated with the motion of the constellation around the Sun, the delay takes here a simple form (restoring factors $c$),
\be
	d(t) = \frac{R}{c} \cos(\beta) \cos(\Omega_{0} t -\lambda) \,,
\ee
with $R/c \simeq 500 \mathrm{s}$ and $\Omega_{0}R/c \simeq []$. In the case of the delays between spacecrafts of the constellation, the delays have a more complicated time dependence and scale instead as $L/c \simeq [] \mathrm{s}$ and $\Omega_{0} L/c \simeq []$.

Since the propagation delays are small compared to the characteristic timescale of the orbital motion $T_{0} = 1\mathrm{yr}$, we will be able to treat $d$ as a perturbation in the equations above~\eqref{}. By doing so, we obtain for the delayed time at first order
\begin{align}
	t_{f}^{d} &= t_{f} - d(t_{f}) ( 1 - \dot{d}(t_{f})) + \calO(\Omega_{0}d) \,, \nn\\
	d(t_{f}^{d}) &= d(t_{f}) ( 1 - \dot{d}(t_{f})) + \calO(\Omega_{0}d) \,.
\end{align}
Now, since $\dot{d} \ll 1$, we can consider that the only relevant correction in~\eqref{} comes from the phase factor, where the factor $2\pi f d$ gives an additional magnification by a factor $3.10^{3}$. We thus arrive in this case at the approximate form
\be
	\tilde{s}(f) \simeq \tilde{h}(f) F(t_{f})\exp\left[ -2i\pi f d(t_{f}) (1-\dot{d}(t_{f})) \right] \,.
\ee   

\subsubsection{The case of a LISA-type detector [TO BE UPDATED]}
\label{subsubsec:sizecorrLISA}

If we put actual numbers, we have $\Omega/2\pi \simeq 3.1\times10^{-8}\mathrm{Hz} \ll f$ over all the frequency band we consider. Considering that the comb is of limited extend in $n$, we might want to Taylor expand the relevant quantities accross the comb. Since the phase is rapidly varying, and since we have to consider a constant and a linear term in the phase as arbitrary, it is natural to Taylor expand the phase at the first order (that is, approximate the phase linearly locally), neglect the $n\Omega/2\pi$ in $(f-n\Omega/2\pi)$ appearing in the argument of the amplitude and other terms, and assess the quality of this approximation.

Preliminary tests show that this approximation should be valid, unsurprisingly, for all the mildly varying functions encountered: $\sinc$, $\exp$ and the linear term in the $c_{n}$'s, and the Bessel functions $J_{n}$. However, $\tilde{h}$ has a Fourier-domain phase which becomes very rapidly varying at low frequencies, in the deep inspiral, $\Psi\propto f^{-5/3}$; here it is not obvious that we can neglect the second derivative of the phase.

For a Newtonian inspiral, applying the SPA gives $\tilde{h}(f) = A_{N}(f)e^{-i\Psi_{N}(f)}$ with
%
\begin{subequations}
\begin{align}
	A_{N}(f) &= -\sqrt{\frac{5\pi}{24}} \frac{G^{2}m^{2}}{Dc^{5}} \nu^{1/2}v^{-7/2}\,,\\
	\Psi_{N}(f) &= 2\pi f t_{0} - \phi_{0} + \frac{3}{128\nu v^{5}} \,, 
\end{align}
\end{subequations}
%
where $v=(G m \pi f/c^{3})^{1/3}$. The magnitude of the first terms in a Taylor expansion in $f_{0}$ is (taking $10^{-5}\mathrm{Hz}$ as a reference starting frequency):
%
\begin{widetext}
\begin{subequations}
\begin{align}
	|\delta A/A| &= \left| \frac{1}{A_{N}}\frac{\ud A_{N}}{\ud f} f_0 \right| \simeq 3\times10^{-3} \left(\frac{f}{10^{-5}\Hz}\right)^{-1} \,,\\
	|\delta^{1}\Psi | &= \left| f_0\frac{\ud \Psi_{N}}{\ud f} \right| \simeq 10^{3}\left( \frac{m}{10^{6}\Msol} \right)^{-5/3} \left( \frac{\nu}{1/4} \right)^{-1} \left( \frac{f}{10^{-5}\Hz} \right)^{-8/3} \,, \\
	|\delta^{2}\Psi | &= \left| \frac{1}{2} f_0^{2}\frac{\ud^{2} \Psi_{N}}{\ud f^{2}} \right| \simeq 5 \left( \frac{m}{10^{6}\Msol} \right)^{-5/3} \left( \frac{\nu}{1/4} \right)^{-1} \left( \frac{f}{10^{-5}\Hz} \right)^{-11/3} \,, \\
	|\delta^{3}\Psi | &= \left| \frac{1}{6} f_0^{3}\frac{\ud^{3} \Psi_{N}}{\ud f^{3}} \right| \simeq 2\times 10^{-2} \left( \frac{m}{10^{6}\Msol} \right)^{-5/3} \left( \frac{\nu}{1/4} \right)^{-1} \left( \frac{f}{10^{-5}\Hz} \right)^{-14/3} \,.
\end{align}
\end{subequations}
\end{widetext}
%
The approximation degrades for lower starting frequencies, and, at a fixed frequency, when the total mass decreases and/or the mass ratio increases. However, if we consider only merging binaries, that is to say binaries of which the merger is seen during the observation period of the mission, we have a lower limit for the starting frequency which can be larger than the detector's lowest frequency in band.

Namely, if we take a Newtonian estimate for the relation between the frequency and the time to coalescence, we have
%
\be
	\pi f(t) = \left[ \frac{256\nu}{5c^{5}} (Gm)^{5/3} \Delta t \right]^{-3/8} \,,
\ee
%
with $\Delta t$ the time to coalescence, and the starting frequency is then
%
\be
	f_{\rm start} = 10^{-5}\mathrm{Hz} \left( \frac{m}{8.5\times10^{6}\Msol} \right)^{-5/8} \left( \frac{\nu}{1/4} \right)^{-3/8} \left( \frac{\Delta t}{5 \mathrm{yr}} \right)^{-3/8} \,.
\ee
%
So we see that the ``worst'' total mass is $8.5\times10^6\Msol (\nu/(1/4))^{3/5}$, for 5 years of observation and $f_{\rm low} = 10^{-5}\mathrm{Hz}$ for the entry in band, which are conservative values; for higher masses, part of the signal is cut because it is out-of-band, while for lower masses, the signal is cut because of the limited time of observation before merger, and we have the scaling
%
\be
	|\delta^{2}\Psi |_{f_\mathrm{start}} \propto m^{5/8}\nu^{3/8}\Delta t^{11/8} \,,
\ee
%
which shows that the effect of the truncation of the signal for a finite observation time wins: the approximation becomes better for lower mass and higher mass ratios systems. The worse point in the parameter space should be, for these Newtonian estimates, at equal mass and $m=8.5\times 10^{6} \Msol$, with $|\delta^{2}\Psi |_{\mathrm{worst}} \simeq 0.15$.

%%%%%%%%%%%%%%%%%%%%%%%%%%%%%%%%%%%%

\subsection{Exact quadratic term in the phase and Fresnel transform}
\label{subsec:fresneltransform}

Since the second-derivative corrective term in~\eqref{} is the most interesting, since it contains the most important deviation from the LLP, we will investigate it in more details in this section. For simplicity of notation, here we will keep to the case of precessing binaries, with only a modulation $F$ and no delays.

To obtain~\eqref{eq:resultdffPsi}, we expanded the exponential. If we keep it as it is, however, we have
%
\begin{subequations}
\begin{align}
	\tilde{s}(f) &= \tilde{h}(f) \int\ud f'\, \tilde{F}(f') e^{-2i\pi f'\tf} \exp\left[ -\frac{i}{2} \frac{\ud^{2} \Psi}{\ud f^{2}} {f'}^{2} \right] \, \\
	&= \tilde{h}(f) \int \ud t\, F(t) \int\ud f'\, e^{2i\pi f' (t-\tf)} \exp\left[ 2i\pi^{2} \epsilon{f'}^{2} \Tf^{2} \right] \,,
\end{align}
\end{subequations}
%
where we recall that $\epsilon = -\mathrm{sgn}(\ud ^{2} \Psi/\ud f^{2})$. The resulting Gaussian integral yields [check how $\epsilon$ propagates]
%
\begin{subequations}
\begin{align}\label{eq:resultFresnel}
	\tilde{s}(f) &= \tilde{h}(f) \calF_{\Tf}[F](\tf) \quad \text{if } \epsilon = 1 \,,\\
	\tilde{s}(f) &= \tilde{h}(f) \calF_{\Tf}[F](\tf)^{*} \quad \text{if } \epsilon = -1 \,,
\end{align}
\end{subequations}
%
where we introduced the Fresnel transform~\cite{} of $F$, defined as follows:
%
\be\label{eq:defFresnel}
	\calF_{\tau}[F](t_{0}) \equiv \frac{e^{i\frac{\pi}{4}}}{\sqrt{2\pi} \tau} \int \ud t \, \exp\left[ - \frac{i}{2} \left( \frac{t-t_{0}}{\tau} \right)^{2}\right] F(t) \,.
\ee
%
The Fresnel transform applies to a large class of functions [be more precise] and intervenes in the theory of diffraction [give references and examples]. Although it is interesting to be able to relate our results to a known transformation used in signal processing, whether this opens avenues for efficient computation methods remains to be investigated (for instance by performing a linear decomposition of the modulation $F$ in wavelets, or chirplets, which have analytically computable Fresnel transforms). 

As a verification, upon performing a Taylor expansion in time of $F(t)$ around $\tf$ in the above integral and integrating  term by term the result, one readily recovers our previous result~\eqref{eq:resultdffPsiTf}.

%%%%%%%%%%%%%%%%%%%%%%%%%%%%%%%%%%%%

\subsection{Resummation of derivatives}
\label{subsec:resummation}

The authors of~\cite{} did not reformulate their result as in~\eqref{eq:resultFresnel}, but they proposed  a resummation scheme. Indeed, the result~\eqref{} looks like a symmetrixed Taylor expansion, except for the factors $i^{p}$ and $1/p!$ instead of $1/(2p)!$. After having truncated the sum at some finite $N$, one can write
%
\begin{align}\label{eq:resum}
	\tilde{s}(f) &\simeq \tilde{h}(f) \sum\limits_{p = 0}^{N} \frac{(-i\epsilon\Tf^{2})^{p}}{2^{p}p!} \partial_{t}^{2p}F(\tf) \nn\\
	&\simeq \tilde{h}(f) \sum\limits_{p= 0}^{N} \sum\limits_{k=0}^{N} a_{N,k}\frac{(k\Tf)^{2p}}{(2p)!}  \partial_{t}^{2p}F(\tf) \nn\\
	&\simeq \tilde{h}(f) \frac{1}{2}\sum\limits_{k=0}^{N} a_{N,k} \left( F(\tf + k\Tf) + F(\tf-k\Tf) \right)\,,
\end{align}
%
where we applied a truncation approximation at the first and last stage. The second equality holds provided that the complex coefficients $a_{N,k}$ are a solution of the $N+1$-dimensional linear system
%
\be
	(-i\epsilon)^{p} (2p-1)!! = \sum\limits_{k=0}^{N} a_{N,k} k^{2p} \quad \text{for } p=0,\dots,N \,.
\ee
%
The two solutions for the comb coefficients in the two cases $\epsilon = \pm 1$ are simply related by a complex conjugation.

An immediate advantage of this reformulation is its improved numerical stability. Although one should expect an ideal signal the scaling $\partial_{t}^{p} F \sim \Omega^{p} F$ with $\Omega$ the precession frequency, it can be extremely hard to control numerical derivatives of $F$ (which in any realistic model will be only provided as sampled at discrete times) well enough to enforce this scaling. Here, however, one simply evaluates the original smooth function $F$ at shifted times.

Note, interestingly, that, if the validity of the comb approach was generic, and if the coefficients $a_{N,k}$ were to converge for $N\rightarrow +\infty$ to a well-defined sequence admitting an appropriate fall-off with $k$, then one would have found a magical way of computing the Fresnel transform of a signal by simply evaluating it on a comb made of at most a few dozen equally spaced times.

This seems to good to be true. In general, one could have made any ansatz for the series expansion to be matched on, and one therefore cannot expect in general nice properties to emerge from a resummation. However, the transformation~\eqref{} between the two series above has a special structure. Namely, written in matrix form, the linear system is a Vandermonde system. Defining the Vandermonde matrix $V(x_{0},\dots,x_{N})$ as $V_{ij} = (x_{j})^{i}$ (with the convention $0^{0} = 1$), setting $x_{j} = j^{2}$, and $b_{p} \equiv (-i\epsilon)^{p}(2p-1)!!$, the linear system is simply
%
\be
	b_{p} = V_{p,k} a_{N,k} \,.
\ee
%
Now, since the Vandermonde system solves the Lagrange polynomial interpolation problem, we know that . Namely, $({}^{t}V^{-1})_{ij}$ is the coefficient of $X^{i}$ in the polynomial $\prod_{k\neq j} (X-x_{k})/(x_{j} - x_{k})$. If we introduce the symmetric polynomials $\sigma$ such that
%
\be	
	\sigma(m, \{x_{1}, \dots, x_{n}\}) = \sum\limits_{1\leq i_{1}<\dots<i_{m}\leq n} x_{i_{1}}\dots x_{i_{m}} \,,
\ee
%
we obtain for the inverse of the matrix $V$
%
\be
	V^{-1}_{ij} = \frac{1}{\prod_{k\neq i} (i^{2} - k^{2})} (-1)^{N-j} \sigma(N-j, \{0,\dots,N^{2}\}\backslash \{i^{2}\}) \,.
\ee
%
This allows us to write an ``analytical'' expression for the $a_{N,k}$ for every finite order $N$:
%
\be
	a_{N,k} = \sum\limits_{p=0}^{N} (-i\epsilon)^{p}(2p-1)!! \frac{(-1)^{N-p}}{\prod_{j\neq k} (j^{2}-k^{2})} \sum\limits_{\substack{ 0 \leq i_{1} < \dots < i_{N-p} \leq N \\ i_{1}, \dots, i_{N-p} \neq k}} i_{1}^{2}\dots i_{N-p}^{2}
\ee
%
The combinatorial explosion of the number of terms in the symmetric polynomials, however, makes this expression impractical to evaluate for $N$ beyond a few dozen. It could allow, however, one to control the asymptotics and maybe prove the convergence or non-convergence of the sequence of $a_{N,k}$ for $N\rightarrow +\infty$. [wishful thinking]

\section*{---BELOW IS OUTDATED MATERIAL---}

%%%%%%%%%%%%%%%%%%%%%%%%%%%%%%%%%%%%

\subsection{Reduced order models for fast waveform generation}
\label{subsec:rom}

[review recent advances in this field: surrogates, ROQ, P\"urrer's SVD method in some more details]

[present the mode-by-mode time and phase offset tracking, specific to HM]

[specific choices: logarithmic sampling, 1D spline interpolation, smoothing spline for projection coefficients]

[give summary plot of unfaithfulness]

[compare speedup with ordinary EOB]

%%%%%%%%%%%%%%%%%%%%%%%%%%%%%%%%%%%%

\subsection{Fourier-domain response of LISA-type detectors}
\label{subsec:fdresponse}

We use the notations of the Kr\'olak \& al paper. We denote the annual frequency of the motion by $f_{0}=1/\mathrm{year}$, and $\Omega = 2\pi f_{0}$. We set $c=1$ except for numerical applications.

%%%%%%%%%%%%%%%%%%%%%%%%%%%%%%%%%%%%

\subsubsection*{Definitions}

The primary observables for a LISA-type instrument are given by (taking $y_{21}$ as an example)
%
\be
	y_{21}(t) = \left( 1-\hatk \cdot n_{3}\right) \left[ \Psi_{3}(t+\hatk\cdot p_{2} - L) - \Psi_{3}(t+\hatk\cdot p_{1}) \right] \,,
\ee
%
with the definitions
%
\be
	\Psi_{A} = \frac{\Phi_{A}}{1-(\hatk\cdot n_{A})^{2}} \,, \quad \Phi_{A} = \frac{1}{2} n_{A}^{i} H_{ij} n_{A}^{j} \,.
\ee
%
Here we follow the approximation made in~\cite{Krolak+04}: all vectors related to the positions, $n_{A}$ and $p_{A}$, will be considered evaluated at the time $t$, neglecting their variation on the time scale of light propagation in the system, which is motivated by $\Omega R/c \ll 1$, $\Omega L/c\ll 1$. It seems also that this approximation is used anyway in the first place when deriving the above form for the frequency variation of the lasers [to check]. We will also idealize the geometry of the constellation by keeping the linear order in eccentricity and neglecting higher-order orbital perturbations. Another convenient definition is
%
\be
	\Phi_{A} = P_{A}^{+} h_{+} + P_{A}^{\times}h_{\times} \,,
\ee
%
where $h_{+}$, $h_{\times}$ are the two polarizations of the wave and the $P_{A}$'s are combinations of trigonometric functions depending on $\Omega t$. 

\subsection*{Fourier transform of a delayed signal}

Let us consider, in general, the problem of determining the Fourier transform of a signal $h$ (whose Fourier transform $\tilde{h}$ is known), when applying a varying delay $d(t)$ and also multiplying by some function $F(t)$. Defining
%
\be
	h_{d}(t) = h(t+d(t)) \,, \quad s(t) = F(t)h_{d}(t) \,,
\ee
%
we have
%
\be
	h_{d}(t) = \int \ud f \, e^{-2i\pi f (t+d(t))}\tilde{h}(f) \,,
\ee
%
and
%
\begin{align}
	\tilde{s}(f) &= \int \ud t \, e^{2i\pi f t} F(t)  \int \ud f' \, e^{-2i\pi f' (t+d(t))}\tilde{h}(f') \nn\\
	&= \int \ud f' \, \tilde{h}(f-f') \int \ud t \, e^{2i\pi f' t} e^{-2i\pi (f-f') d(t)} F(t) \,.
\end{align}
%
We can rewrite the last equation as a kind of convolution with a frequency-dependent Kernel, according to
%
\begin{align}\label{eq:freqkernel}
	\tilde{s}(f) &= \int \ud f' \, \tilde{h}(f-f') G(f,f') \,, \nn\\
	G(f,f') &= \int \ud t \, e^{2i\pi f' t} e^{-2i\pi (f-f') d(t)} F(t) \,.
\end{align}
%
Since in our case both $d(t)$ and $F(t)$ are $2\pi/\Omega$-periodic, the latter Kernel will reduce to a discrete comb in frequency. If we define $g[f](t) = e^{-2i\pi f d(t)} F(t)$, a Fourier series decomposition gives
%
\begin{align}
	g[f](t) &= \sum\limits_{n\in \mathbb{Z}} c_{n}[g](f) e^{-i n\Omega t} \,, \nn\\
	c_{n}[g](f) &= \frac{\Omega}{2\pi}\int_{0}^{\frac{2\pi}{\Omega}} \ud t \, g[f](t) e^{i n \Omega t} \,,
\end{align}
%
which transforms~\eqref{eq:freqkernel} into
%
\begin{align}
	G(f,f') &= \sum\limits_{n\in \mathbb{Z}} c_{n}[g](f-f') \delta\left(f' - n f_{0}\right) \,, \nn\\
	\tilde{s}(f) &= \sum\limits_{n\in \mathbb{Z}} c_{n}[g]\left(f - n f_{0}\right) \tilde{h}\left(f - n f_{0}\right) \,.
\end{align}
%

%%%%%%%%%%%%%%%%%%%%%%%%%%%%%%%%%%%%

\subsubsection*{Application to $y_{AB}$ observables}

We now apply the above results to $y_{AB}$ observables, with the example of $y_{21}$. We will assume our starting point are the Fourier transforms of $h_{+}$ and $h_{\times}$. We get for either polarization
%
\begin{widetext}
\begin{align}
	g_{21}^{+,\times}[f](t) &= \frac{P_{3}^{+,\times}}{1+\hatk\cdot n_{3}} \left[ e^{-2i\pi f (\hatk\cdot p_{2} - L)} - e^{-2i\pi f \hatk\cdot p_{1}} \right] \,, \nn\\
	&= i\pi L f P_{3}^{+,\times} \sinc\left[ \pi L f(1+\hatk \cdot n_{3})\right]\exp\left[ i \pi f L \left(  1 - \hatk \cdot \frac{p_{1}+p_{2}}{L}\right) \right] \,,
\end{align}
\end{widetext}
%
where we used $p_{2}-p_{1} = -L n_{3}$ and the dependence in $t$ is in the vectors $n_{A}$, $p_{A}$. $(p_{2}+p_{1})/2 $ also has a simple expression in the LISA frame. Thus the complete result before any further approximation is
%
\begin{widetext}
\begin{subequations}
\begin{align}
	\tilde{y}_{21}^{+,\times}(f) &= \sum\limits_{n\in \mathbb{Z}} c_{n}[g_{21}^{+,\times}]\left(f - n f_{0}\right) \tilde{h}_{+,\times}\left(f - n f_{0}\right)\,, \\
	c_{n}[g_{21}^{+,\times}]\left(f\right) &= \frac{\Omega}{2\pi}\int_{0}^{\frac{2\pi}{\Omega}} \ud t \, e^{i n \Omega t} i\pi L f P_{3}^{+,\times} \sinc\left[ \pi L f(1+\hatk \cdot n_{3})\right]\exp\left[ i \pi f L \left(  1 - \hatk \cdot \frac{p_{1}+p_{2}}{L}\right) \right]  \,.
\end{align}
\end{subequations}
\end{widetext}
%

%%%%%%%%%%%%%%%%%%%%%%%%%%%%%%%%%%%%

\subsubsection*{Separating the orbital delay from the constellation delay}

Some numerical tests for the coefficients $c_{n}$ on the frequency band $10^{-5}$Hz-$1$Hz show that the extent in $n$ of the frequency comb rises a lot (up to several hundreds) at the high frequency end. This seems attributable to the presence of $\hatk\cdot r/c$ in the argument of te exponential and the large number of oscillations this causes. Indeed, this contribution is roughly $f R/c$ with $c/R\sim 2.10^{-3}\Hz$, whereas $c/L\sim 2.10^{-1}\Hz$, so that the oscillations due to $L/c$ terms are relatively mild, while those due to the $R/c$ term are much more rapid and yield extended harmonic content.

Luckily, it is possible to factor out this term, and to control it analytically. It corresponds to the part of the delay that corresponds to the GW propagating from the solar system barycenter ($\Omega$) to the moving center of the constellation ($O$), and has a simple expression involving only the sine and cosine of $\Omega t$. If we define
%
\be
	h_{+,\times}^{O} = h_{+,\times} (t + \hatk \cdot r)
\ee
%
as the signal as measured at the center of the constellation $O$, since $\hatk \cdot r = R \cos \beta \cos(\Omega t - \lambda)$ (choosing $\eta_{0} = 0$), applying the same calculation as above yields for this simplified retardation
%
\begin{align}
	\tilde{h}_{+,\times}^{O}\left(f\right) &= \sum\limits_{n\in \mathbb{Z}} c_{n}[g_{O}^{+,\times}]\left(f - n f_{0}\right) \tilde{h}_{+,\times}\left(f - n f_{0}\right)\,, \\
	c_{n}[g_{O}^{+,\times}]\left(f\right) &= \frac{\Omega}{2\pi}\int_{0}^{\frac{2\pi}{\Omega}} \ud t \, e^{i n \Omega t} \exp\left[ -2i \pi f R \cos \beta \cos (\Omega t - \lambda) \right] \nn\\ 
	&= e^{i n \lambda} \frac{1}{2\pi}\int_{0}^{2\pi} \ud u \, e^{i n u} \exp\left[ -2i \pi f R \cos \beta \cos u \right] \nn\\
	&= i^{n}e^{i n \lambda} J_{n}\left[ -2\pi f R \cos \beta \right] \,.
\end{align}
%
Thus we obtain, in two steps,
%
\begin{widetext}
\begin{subequations}
\begin{align}
	\tilde{h}_{+,\times}^{O}\left(f\right) &= \sum\limits_{n\in \mathbb{Z}} i^{n}e^{i n \lambda} J_{n}\left[ -2\pi \left(f - n f_{0}\right) R \cos \beta \right]\tilde{h}_{+,\times}\left(f - n f_{0}\right)\,, \\
	\tilde{y}_{21}^{+,\times}(f) &= \sum\limits_{n\in \mathbb{Z}} c_{n}[\gamma_{21}^{+,\times}]\left(f - n f_{0}\right) \tilde{h}_{+,\times}^{O}\left(f - n f_{0}\right)\,, \\
	c_{n}[\gamma_{21}^{+,\times}]\left(f\right) &= \frac{\Omega}{2\pi}\int_{0}^{\frac{2\pi}{\Omega}} \ud t \, e^{i n \Omega t} 2 i\pi L f f_{3}^{+,\times}\sinc\left[ \pi L f (1+\hatk \cdot n_{3})\right]\exp\left[ i \pi f L\left(1 - \hatk \cdot O_{2} \cdot \frac{p_{1}^{L}+p_{2}^{L}}{L} \right) \right] \,.
\end{align}
\end{subequations}
\end{widetext}
%
Now we find that the Bessel comb that relates $\tilde{h}^{O}$ to $\tilde{h}$ (which we can control analytically) contains all the spread in $n$, as the new $c_{n}$ coefficients (expressed with integrals) do not extend beyond $n=\pm 10$. These comb coefficients show a smooth variation with frequency (they vary much less than the Fourier-domain signal): in the worst case, one should be able to compute them numerically at a handful of frequencies and interpolate inbetween.

%%%%%%%%%%%%%%%%%%%%%%%%%%%%%%%%%%%%



%%%%%%%%%%%%%%%%%%%%%%%%%%%%%%%%%%%%

\subsubsection*{Direct summation}

The Newtonian estimates above indicate that the locally linear phase approximation is going to be at least marginally valid for all sources of which we observe the inspiral, merger and ringdown. We adopt for the results above the generic notation
%
\begin{align}
	\tilde{s}(f) &=  \sum\limits_{n\in \mathbb{Z}} c_{n}[G]\left(f - n f_{0}\right) \tilde{h}\left(f - n f_{0}\right)\,, \nn\\
	c_{n}[G](f) &= \frac{1}{T} \int_{0}^{T}\ud t\, e^{in\Omega t} G[f](t)
\end{align}
%
with $T=2\pi/\Omega=1\mathrm{yr}$ and $G[f](t)$ a function of frequency and time.

At the lowest order in the approximation, with $|\delta A/A| \ll 1$ and $|\delta^{2} \Psi | \ll 1$, and also neglecting $f_{0}$ in the frequency-dependence of $G$, we have $\tilde{h}(f-n f_{0}) \simeq \tilde{h}(f) \exp\left[ i n f_{0} \ud \Psi/\ud f \right]$ and (after commuting the sum and integral):
%
\be
	\tilde{s}(f) \simeq \tilde{h}(f) \frac{1}{T} \int_{0}^{T}\ud t\, G[f](t) \sum\limits_{n} e^{i n \Omega t} \exp\left[ i n f_{0} \frac{\ud \Psi}{\ud f} \right] \,.
\ee
%
Using the Dirac comb relation
%
\be
	\sum\limits_{n} e^{i n \Omega (t-t_{0})} = \frac{2\pi}{\Omega} \sum\limits_{n} \delta\left( t-t_{0} - \frac{n\Omega}{2\pi} \right) \,,
\ee
%
we have
%
\be
	\sum\limits_{n} e^{i n \Omega t} \exp\left[ i n f_{0} \frac{\ud \Psi}{\ud f }\right] = T \sum\limits_{n} \delta\left( t + \frac{1}{2\pi}\frac{\ud \Psi}{\ud f} - n T \right) \,,
\ee
%
and since only one of the Dirac deltas has support on $[0,T]$, we obtain simply
%
\be
	\tilde{s}(f) \simeq \tilde{h}(f) \times G[f]\left( -\frac{1}{2\pi} \frac{\ud \Psi}{\ud f} \right)\,.
\ee
%
Now, recall that when the SPA applies (which it does not in the merger-ringdown phase of the signal), we have the relations
%
\begin{align}
	\Psi_{\rm SPA} &= \phi(t_{f}) - 2\pi f t_{f} + \mathrm{const} \,,\nn\\
	\frac{\ud \Psi_{\rm SPA}}{\ud f} &= -2\pi t_{f} \,,
\end{align}
%
thus providing a simple physical interpretation: at leading order, we can simply use the time-domain modulation and delays, as evaluated at a fixed time $t_{f}$, the time at which the corresponding frequency is dominantly emitted. Notice however that the above result is more general, as it extends to the merger-ringdown phase, and that the limits of validity of the two approximations differ: the SPA becomes increasingly valid in the deep inspiral, whereas the LLP breaks down in the limit of almost monochromatic sources but becomes increasingly valid at the end of the inspiral, when the signal evolves quickly compared to the orbital timescale.

Notice that these Newtonian estimates are very poor to cover the merger and ringdown; for instance, above the dominant ringdown frequency the Fourier-domain amplitude drops rapidly in a way that is not captured by the power law above.

%%%%%%%%%%%%%%%%%%%%%%%%%%%%%%%%%%%%

\subsubsection*{Summation of $\tilde{h}_{O}\left(f\right)$}

The expression obtained above for $\tilde{h}_{O}\left(f\right)$ can be summed, either by applying the above general formulas with the function $G[f](t) = \exp\left[ -2i \pi f R \cos \beta \cos (\Omega t - \lambda) \right]$, either directly by means of the Jacobi-Anger expansion:
%
\begin{widetext}
\begin{align}
	\tilde{h}_{O}\left(f\right) &\simeq \sum\limits_{n\in \mathbb{Z}} i^{n}e^{i n \lambda} J_{n}\left[ -2\pi f R \cos \beta \right] A(f) \exp\left[ -i \Psi \left(f-\frac{n\Omega}{2\pi}\right) \right] \nn \\
	&\simeq A(f) e^{-i \Psi \left(f\right)} \sum\limits_{n\in \mathbb{Z}} i^{n}e^{i n \lambda} J_{n}\left[ -2\pi f R \cos \beta \right] \exp\left[ i \frac{n\Omega}{2\pi} \frac{\ud \Psi}{\ud f} + \calO(\Omega^{2})  \right]
\end{align}
\end{widetext}
%
We can then use the Jacobi-Anger expansion (with the notation $z\equiv 2\pi R f \cos\beta$)
%
\be
	e^{i z \cos \theta}= \sum\limits_{n\in \mathbb{Z}} i^{n}J_{n}[z]e^{i n \theta} \,,
\ee
%
to obtain a simple phase correction
%
\begin{subequations}
\begin{align}
	\tilde{h}_{O}\left(f\right) &\simeq \tilde{h}(f) \exp\left[ -iz \cos\left( \lambda + \frac{\Omega}{2\pi} \frac{\ud \Psi}{\ud f} \right) \right]\,.
\end{align}
\end{subequations}
%
The corrections beyond the leading order can be derived directly by a kind of Feynman trick, introducing derivatives with respect to the parameter $\lambda$.

%%%%%%%%%%%%%%%%%%%%%%%%%%%%%%%%%%%%

\subsection{Accelerated overlap computations for modeled waveforms}
\label{subsec:overlaps}

[present recursion relations - Fresnel integrals]

[discuss numerical robustness, power-law cut in the space of phase coefficients]

%%%%%%%%%%%%%%%%%%%%%%%%%%%%%%%%%%%%

\subsection{Summary \& performance}
\label{subsec:performance}

[give speedup for linear overlap and Fresnel overlap, for 22 only and HM]

%%%%%%%%%%%%%%%%%%%%%%%%%%%%%%%%%%%%
%%%%%%%%%%%%%%%%%%%%%%%%%%%%%%%%%%%%

\section{Impact of noise realizations on parameter estimation for LIGO}
\label{sec:noiseligo}

[no-noise corresponds to a geometric average on the posteriors themselves across noise realizations; how does this translate in practice ?]

[Phil's plot: example of the distribution of the MAP across noise realizations, compared to the posterior itself]

[Phil's plot: example(s) of 100 cumulative distributions compared to the no-noise case]

[evolution of results when increasing SNR: 12, 15, 20]

[table giving the ``averaged'' broadening of the posterior and the ``averaged'' departure of the MAP from injection across noise realizations, for all SNR=12 simulations]

[discussion: validation of the no-noise analysis as a first proxy to do parameter estimation estimation]

%%%%%%%%%%%%%%%%%%%%%%%%%%%%%%%%%%%%
%%%%%%%%%%%%%%%%%%%%%%%%%%%%%%%%%%%%

\section{Impact of merger/ringdown and higher modes on parameter estimation for LISA}
\label{sec:pelisa}

[connection to previous work: McWilliams\&al, Littenberg\&al]

[plot: example posterior with/without HM, EOBNRv2HMROM injection]

[plot: example posterior with/without merger/ringdown, EOBNRv2HMROM injection]

[plot: posterior with information contained in the inspiral only: inspiral-only injection and templates, vs IMR injection and template]

[discuss multimodal structure of the posterior (sky position notably), if it holds]

[improvement on ]

[sky localization errors across the sky with full bayesian inference on a few example cases: a sufficiently large number of inferences for this to be interesting ?]

[accumulation of information with time ? - future studies]

[compare to Fisher matrix analysis on some examples ?]

%%%%%%%%%%%%%%%%%%%%%%%%%%%%%%%%%%%%
%%%%%%%%%%%%%%%%%%%%%%%%%%%%%%%%%%%%

\section{Summary and Conclusions}
\label{sec:conc}

[extension to longer waveforms, lower masses]

[use of other fast FD waveforms: SEOBNRv2ROM, PhenomP]

[impact of aligned spins ? of simple precession ?]

[impact of merger-ringdown in the aligned spins case and simple precession case ?]

[validation of existing Fischer matrix analysis, at least on some examples, with full Bayesian inference]

[compare different instrument designs, across astrophysical models (Sesana-Barausse\&al)]

[extend analysis to a full population ?]

%%%%%%%%%%%%%%%%%%%%%%%%%%%%%%%%%%%%
%%%%%%%%%%%%%%%%%%%%%%%%%%%%%%%%%%%%

\vspace{4.5mm}

\hspace{0.85in}
{\bf Acknowledgments}

\vspace{3.5mm}

[Acknowledgments]

%%%%%%%%%%%%%%%%%%%%%%%%%%%%%%%%%%%%
%%%%%%%%%%%%%%%%%%%%%%%%%%%%%%%%%%%%

\appendix

\section{Details on Fourier-domain reduced order models}
\label{app:rom}

%%%%%%%%%%%%%%%%%%%%%%%%%%%%%%%%%%%%
%%%%%%%%%%%%%%%%%%%%%%%%%%%%%%%%%%%%

\section{Details on the Fourier-domain LISA response}
\label{app:fdresponse}

%%%%%%%%%%%%%%%%%%%%%%%%%%%%%%%%%%%%

\subsection{TDI observables}
\label{appsubsec:TDI}

For the constellation delay and modulation, if we associate as above a function $G_{AB}$ to each basic Doppler shift observable $y_{AB}$ (as with the example of $y_{21}$ above), for each of the polarizations we obtain:
%
\begin{align}
	G_{21}^{+,\times} &= i\pi f L \left(n_{3}\cdot P_{+,\times}\cdot n_{3}\right) \sinc\left[ \pi f L \left( 1 + k\cdot n_{3} \right) \right] \exp\left[ i\pi f L \left( 1 - k\cdot (p_{1}+p_{2})/L  \right) \right] \nn\\
	G_{12}^{+,\times} &= i\pi f L \left(n_{3}\cdot P_{+,\times}\cdot n_{3}\right) \sinc\left[ \pi f L \left( 1 - k\cdot n_{3} \right) \right] \exp\left[ i\pi f L \left( 1 - k\cdot (p_{1}+p_{2})/L  \right) \right] \nn\\
	G_{32}^{+,\times} &= i\pi f L \left(n_{1}\cdot P_{+,\times}\cdot n_{1}\right) \sinc\left[ \pi f L \left( 1 + k\cdot n_{1} \right) \right] \exp\left[ i\pi f L \left( 1 - k\cdot (p_{2}+p_{3})/L  \right) \right] \nn\\
	G_{23}^{+,\times} &= i\pi f L \left(n_{1}\cdot P_{+,\times}\cdot n_{1}\right) \sinc\left[ \pi f L \left( 1 - k\cdot n_{1} \right) \right] \exp\left[ i\pi f L \left( 1 - k\cdot (p_{2}+p_{3})/L  \right) \right] \nn\\
	G_{13}^{+,\times} &= i\pi f L \left(n_{2}\cdot P_{+,\times}\cdot n_{2}\right) \sinc\left[ \pi f L \left( 1 + k\cdot n_{2} \right) \right] \exp\left[ i\pi f L \left( 1 - k\cdot (p_{3}+p_{1})/L  \right) \right] \nn\\
	G_{31}^{+,\times} &= i\pi f L \left(n_{2}\cdot P_{+,\times}\cdot n_{2}\right) \sinc\left[ \pi f L \left( 1 - k\cdot n_{2} \right) \right] \exp\left[ i\pi f L \left( 1 - k\cdot (p_{3}+p_{1})/L  \right) \right] \,,
\end{align}
%
where the $p_{A}$ vectors refer to the positions with respect to the center of the constellation $O$ (that is to say, they are given by $O_{2}\cdot p_{A}^{L}$).

The expression for the $X_{A}$ second-generation TDI observables are given by (c.f. Kr\'olak\&al)
%
\begin{widetext}
\begin{align}
	X_{1} &= \left[ \left( y_{31} + y_{13,2} \right) - \left( y_{21} + y_{12,3} \right) - \left( y_{31} + y_{13,2} \right)_{,33} + \left( y_{21} + y_{12,3} \right)_{,22} \right] \nn\\
	& - \left[ \left( y_{31} + y_{13,2} \right) - \left( y_{21} + y_{12,3} \right) - \left( y_{31} + y_{13,2} \right)_{,33} + \left( y_{21} + y_{12,3} \right)_{,22} \right]_{,2233} \,.
\end{align}
\end{widetext}
%
Within the rigid approximation, all the delays are considered equal to $L/c$ as far as the gravitational wave response is concerned. We are applying successive constant delays to the basic observables, and the translation in Fourier domain is simply
%
\begin{widetext}
\begin{align}
	\tilde{X}_{1}(f) &= -4 \sin\left( 2\pi f L \right) \sin\left( 4\pi f L \right) e^{6 i \pi f L} \left[ \tilde{y}_{31} - \tilde{y}_{21} + e^{2i\pi f L} \left( \tilde{y}_{13} - \tilde{y}_{12} \right) \right] \,, \nn\\
	\tilde{X}_{2}(f) &= -4 \sin\left( 2\pi f L \right) \sin\left( 4\pi f L \right) e^{6 i \pi f L} \left[ \tilde{y}_{12} - \tilde{y}_{32} + e^{2i\pi f L} \left( \tilde{y}_{21} - \tilde{y}_{23} \right) \right] \,, \nn\\
	\tilde{X}_{3}(f) &= -4 \sin\left( 2\pi f L \right) \sin\left( 4\pi f L \right) e^{6 i \pi f L}\left[ \tilde{y}_{23} - \tilde{y}_{13} + e^{2i\pi f L} \left( \tilde{y}_{32} - \tilde{y}_{31} \right) \right] \,.
\end{align}
\end{widetext}
%

For the Sagnac observables, we have similarly ((8) of Shaddock\&al)
%
\begin{align}
	\alpha_{1} &= \left[ \left( y_{31} + y_{23,2} + y_{12,12} \right) - \left( y_{21} + y_{32,3} + y_{13,13} \right) \right] \nn\\
	& - \left[ \left( y_{31} + y_{23,2} + y_{12,12} \right)_{,213} - \left( y_{21} + y_{32,3} + y_{13,13} \right)_{,312} \right]
\end{align}
%
and
%
\begin{widetext}
\begin{align}
	\tilde{\alpha}_{1}(f) &= - 2i \sin\left(3\pi f L\right) e^{3 i \pi f L} \left[ \tilde{y}_{31} - \tilde{y}_{21} + e^{2i\pi f L} \left( \tilde{y}_{23} - \tilde{y}_{32} \right) + e^{4i\pi f L} \left( \tilde{y}_{12} - \tilde{y}_{13} \right) \right] \,, \nn\\
	\tilde{\alpha}_{2}(f) &= - 2i \sin\left(3\pi f L\right) e^{3 i \pi f L} \left[ \tilde{y}_{12} - \tilde{y}_{32} + e^{2i\pi f L} \left( \tilde{y}_{31} - \tilde{y}_{13} \right) + e^{4i\pi f L} \left( \tilde{y}_{23} - \tilde{y}_{21} \right) \right] \,, \nn\\
	\tilde{\alpha}_{3}(f) &= - 2i \sin\left(3\pi f L\right) e^{3 i \pi f L} \left[ \tilde{y}_{23} - \tilde{y}_{13} + e^{2i\pi f L} \left( \tilde{y}_{12} - \tilde{y}_{21} \right) + e^{4i\pi f L} \left( \tilde{y}_{31} - \tilde{y}_{32} \right) \right] \,.
\end{align}
\end{widetext}
%
The optimal second-generation combinations $A$, $E$ and $T$ are given by ((57) of Kr\'olak\&al)
%
\begin{align}
	A &= \frac{1}{\sqrt{2}} \left( \alpha_{3} - \alpha_{1} \right) \nn\\
	E &= \frac{1}{\sqrt{6}} \left( \alpha_{1} - 2\alpha_{2} + \alpha_{3} \right) \nn\\
	T &= \frac{1}{\sqrt{3}} \left( \alpha_{1} + \alpha_{2} + \alpha_{3} \right) \,,
\end{align}
%
which gives
%
\begin{widetext}
\begin{align}
	\tilde{A}(f) &= - \frac{2i}{\sqrt{2}} \sin\left(3\pi f L\right) e^{3 i \pi f L} \left[ \left( \tilde{y}_{13} + \tilde{y}_{31} \right) \left( e^{4i\pi f L} - 1 \right) + \left( \tilde{y}_{21} + \tilde{y}_{23} \right) \left( 1 - e^{2i\pi f L} \right) \right. \nn\\
	& \qquad\qquad\qquad\qquad\qquad\qquad \left. + \left( \tilde{y}_{32} + \tilde{y}_{12} \right) \left( e^{2i\pi f L} - e^{4i\pi f L} \right) \right] \nn\\
	\tilde{E}(f) &= - \frac{2i}{\sqrt{6}} \sin\left(3\pi f L\right) e^{3 i \pi f L} \left[ \left( \tilde{y}_{23} - \tilde{y}_{21} \right) \left( 1 + e^{2i\pi f L} - 2e^{4i \pi f L} \right) + \left( \tilde{y}_{31} - \tilde{y}_{13} \right) \left( 1 - 2e^{2i\pi f L} + e^{4i \pi f L} \right) \right. \nn\\
	& \qquad\qquad\qquad\qquad\qquad\qquad + \left( \tilde{y}_{12} - \tilde{y}_{32} \right) \left( -2 + e^{2i\pi f L} + e^{4i \pi f L} \right) \left. \right] \nn\\
	\tilde{T}(f) &= - \frac{2i}{\sqrt{3}} \sin\left(3\pi f L\right) e^{3 i \pi f L} \left( \tilde{y}_{31} - \tilde{y}_{13} + \tilde{y}_{12} - \tilde{y}_{21} + \tilde{y}_{23} - \tilde{y}_{32} \right) \left( 1 + e^{2i\pi f L} + e^{4i \pi f L} \right) \,.
\end{align}
\end{widetext}
%

%%%%%%%%%%%%%%%%%%%%%%%%%%%%%%%%%%%%

\subsection{Beyond the leading-order Fourier-domain response}
\label{appsubsec:nloresponse}

We derive in the following analytical corrections to this leading order formula. In practice, we expect that the first and second correction terms, at the very most, will be needed.

\subsubsection*{Frequency dependence of the modulation/delay}

If we want to keep the full frequency dependence of the modulation/delay function $G$ in the sum, then we can expand
%
\be
	G[f-n f_{0}](t) = \sum\limits_{k\geq 0} \frac{1}{k!}\left( \frac{-n \Omega}{2\pi} \right)^{k} \frac{\partial^{k} G}{\partial f^{k}}[f](t) \,,
\ee
%
and we have
%
\begin{widetext}
\begin{align}
	\tilde{s}(f) &\simeq \tilde{h}(f) \frac{1}{T} \int_{0}^{T}\ud t\, \sum\limits_{n}\sum\limits_{k} e^{i n \Omega t} \exp\left[ i n f_{0} \frac{\ud \Psi}{\ud f} \right] \frac{1}{k!}\left( \frac{-n \Omega}{2\pi} \right)^{k} \frac{\partial^{k} G}{\partial f^{k}}[f](t) \nn\\
	&= \tilde{h}(f) \sum\limits_{n}\sum\limits_{k} \exp\left[ i n f_{0} \frac{\ud \Psi}{\ud f} \right] \frac{1}{k!} \frac{1}{(2i\pi)^{k}} \frac{1}{T} \int_{0}^{T}\ud t\, (-i n \Omega)^{k}  e^{i n \Omega t}  \frac{\partial^{k} G}{\partial f^{k}}[f](t) \nn\\
	&= \tilde{h}(f) \sum\limits_{n}\sum\limits_{k} \exp\left[ i n f_{0} \frac{\ud \Psi}{\ud f} \right] \frac{1}{k!} \frac{1}{(2i\pi)^{k}} \frac{1}{T} \int_{0}^{T}\ud t\, e^{i n \Omega t}  \frac{\partial^{2k} G}{\partial f^{k}\partial t^{k}}[f](t) \nn\\
	&= \tilde{h}(f) \sum\limits_{k\geq 0} \frac{1}{k!} \frac{1}{(2i\pi)^{k}}  \frac{\partial^{2k} G}{\partial f^{k}\partial t^{k}}[f]\left(-\frac{1}{2\pi} \frac{\ud \Psi}{\ud f} \right) \,,
\end{align}
\end{widetext}
%
where we performed $k$ integrations by parts (using the periodicity of $G$ for any frequency) and summed over $n$ in the same way as in the leading order calculation.

\subsubsection*{Quadratic term in the phase}

If we keep the first correction in the phase beyond the LLP, the quadratic term $\delta^{2}\Psi$, but ignore all the subsequent terms, we can Taylor-expand the exponential according to
%
\begin{widetext}
\begin{align}
	\tilde{h}(f-n f_{0}) &\simeq \tilde{h}(f)  \exp\left[ i n f_{0} \frac{\ud \Psi}{\ud f} \right] \exp\left[ -\frac{1}{2} i (n f_{0})^{2} \frac{\ud^{2}\Psi}{\ud f^{2}} \right] \nn\\
	&= \tilde{h}(f)  \exp\left[ i n f_{0} \frac{\ud \Psi}{\ud f} \right] \sum\limits_{k\geq 0} \frac{1}{k!}\left( \frac{-i}{2} \right)^{k} \left( \frac{n\Omega}{2\pi} \right)^{2k} \left( \frac{\ud^{2}\Psi}{\ud f^{2}}  \right)^{k} \,,
\end{align}
\end{widetext}
% 
and we have
%
\begin{widetext}
\begin{align}
	\tilde{s}(f) &\simeq \tilde{h}(f) \frac{1}{T} \int_{0}^{T}\ud t\, \sum\limits_{n}\sum\limits_{k} e^{i n \Omega t} \exp\left[ i n f_{0} \frac{\ud \Psi}{\ud f} \right] \frac{1}{k!} \left( \frac{-i}{2} \right)^{k} \left( \frac{n\Omega}{2\pi} \right)^{2k} \left( \frac{\ud^{2}\Psi}{\ud f^{2}}  \right)^{k} G[f](t) \nn\\
	&= \tilde{h}(f) \sum\limits_{n}\sum\limits_{k} \exp\left[ i n f_{0} \frac{\ud \Psi}{\ud f} \right] \frac{1}{k!} \left( \frac{i}{2} \right)^{k} \frac{1}{(2\pi)^{2k}} \left( \frac{\ud^{2}\Psi}{\ud f^{2}}  \right)^{k} \frac{1}{T} \int_{0}^{T}\ud t\, (i n \Omega)^{2 k}  e^{i n \Omega t} G[f](t) \nn\\
	&= \tilde{h}(f) \sum\limits_{k\geq 0} \frac{1}{2^k k!} \left( \frac{i}{4\pi^{2}}\frac{\ud^{2}\Psi}{\ud f^{2}}  \right)^{k} \frac{\partial^{k} G}{\partial t^{k}}[f]\left(-\frac{1}{2\pi} \frac{\ud \Psi}{\ud f} \right) \,.
\end{align}
\end{widetext}
%

\subsubsection*{Linear term in the amplitude}

If we keep the first correction in the amplitude, we have
%
\be
	\tilde{h}(f-n f_{0}) \simeq \tilde{h}(f)  \exp\left[ i n f_{0} \frac{\ud \Psi}{\ud f} \right] \left( 1 - \frac{n \Omega}{2\pi} \frac{1}{A}\frac{\ud A}{\ud f} \right) \,,
\ee
%
and we obtain similarly
%
\be
	\tilde{s}(f) \simeq \tilde{h}(f) \left[ G[f]\left(-\frac{1}{2\pi} \frac{\ud \Psi}{\ud f} \right) + \frac{1}{2 i \pi} \frac{1}{A} \frac{\partial G}{\partial t}[f]\left(-\frac{1}{2\pi} \frac{\ud \Psi}{\ud f} \right) \right]\,.
\ee
%

%%%%%%%%%%%%%%%%%%%%%%%%%%%%%%%%%%%%
%%%%%%%%%%%%%%%%%%%%%%%%%%%%%%%%%%%%

%\bibliography{ListeRef.bib}
\bibliography{/Users/marsat/Documents/publications/bibliographie/ListeRef.bib}


\end{document}

