\documentclass[aps,showpacs,twocolumn,prd,superscriptaddress,nofootinbib]{revtex4}

\usepackage{amsmath}
\usepackage{amsfonts}
\usepackage{amssymb}
\usepackage{latexsym}
\usepackage{graphicx}
\usepackage{bm}
\usepackage{color}
\usepackage{enumerate}
\usepackage{ulem}

\newcommand{\be}{\begin{equation}}
\newcommand{\ee}{\end{equation}}
\newcommand\ud{{\mathrm{d}}}
\newcommand\uD{{\mathrm{D}}}
\newcommand\calO{{\mathcal{O}}}
\newcommand\bfx{\mathbf{x}}
\newcommand{\ov}[1]{\overline{#1}}
\newcommand{\ph}[1]{\phantom{#1}}
\newcommand{\cte}{\mathrm{cte}}
\newcommand{\nn}{\nonumber}
\newcommand{\hatk}{\hat{k}}
\newcommand{\Hz}{\,\mathrm{Hz}}
\newcommand{\sinc}{\,\mathrm{sinc}}
\newcommand{\Msol}{M_{\odot}}

\begin{document}

\title{Bayesian methods for black hole merger parameter estimation with LISA.}

\author{John G. Baker}
\affiliation{Gravitational Astrophysics Laboratory, NASA Goddard Space Flight Center, 8800 Greenbelt Rd., Greenbelt, MD 20771, USA}
\author{Sylvain Marsat}
\affiliation{Department of Physics, University of Maryland, College Park, MD 20742, USA}
\affiliation{Gravitational Astrophysics Laboratory, NASA Goddard Space Flight Center, 8800 Greenbelt Rd., Greenbelt, MD 20771, USA}
\affiliation{Max Planck Institute for Gravitational Physics (Albert Einstein Institute), Am M\"uhlenberg 1, Potsdam-Golm, 14476, Germany}


\date{\today}

\begin{abstract}

[Abstract]

\end{abstract}

\pacs{
04.70.Bw, % classical black holes
04.80.Nn, % Gravitational wave detectors and experiments
95.30.Sf, % relativity and gravitation
95.55.Ym, % Gravitational radiation detectors
97.60.Lf  % black holes (astrophysics)
}

\maketitle

%%%%%%%%%%%%%%%%%%%%%%%%%%%%%%%%%%%%

\section{Introduction}
\label{sec:intro}

[Introduction]

%%%%%%%%%%%%%%%%%%%%%%%%%%%%%%%%%%%%

\section{Methodology}
\label{sec:intro}

\subsection{Fast frequency domain LISA response}
\label{sec:response}

-- Geometric setting, definitions, summarize the FD response

\subsection{Reduced order model for EOBNRv2 waveforms.}
\label{sec:waveforms}

-- Waveforms: EOBNRv2HMROM + PNext

\subsection{Fast likelihood computation}
\label{sec:likelihood}

-- Likelihood computation: fast overlaps

\subsection{Fisher matrix parameter estimation}
\label{sec:Fisher}

-- Fisher matrices computation: step self-tuning,...

\subsection{Bayesian sampling}
\label{sec:samplers}

-- Bayesian samplers: Multinest, PTMCMC

%%%%%%%%%%%%%%%%%%%%%%%%%%%%%%%%%%%%

\section{Morphology of the signals}
\label{sec:morph}

- Morphology of the response

\subsection{Temporal development of SNR}
\label{sec:timeSNR}

-- SNR accumulation with time or frequency: inspiral/MRD balance across mass range

\subsection{LISAframe parameters}
\label{sec:LISAframe}

-- LISA frame angles

\subsection{The low-frequency limit}
\label{sec:low-freq}

-- make the connection to low-f response (2 interferometers)

\subsection{A simplified likelihood model}
\label{sec:simple-like}

-- simplified response and closed-form likelihood

%%%%%%%%%%%%%%%%%%%%%%%%%%%%%%%%%%%%

\section{Illustrating examples}
\label{sec:examples}

- Demonstration/examples

\subsection{A non-degenerate case}
\label{sec:high-M-non-deg}

-- High-mass system, non-degenerate, 22 and HM + Fisher - stacked SNR contours
eg:
m1,    m2,    t0,D,     phi0,i,   lam,  bet, psi
8.89e5,1.11e5,0, 3.67e4,pi/3,pi/2,3pi/4,pi/3,pi/3

-- Performance: waveform and likelihood costs, number of sampler evaluations

\subsection{Stellar origin black holes}
\label{sec:SOBH}

-- SOBH system, SNR sufficient for clean analysis + Fisher - stacked SNR contours
-- Performance: waveform and likelihood costs, number of sampler evaluations

\subsection{Highlighting degeneracies}
\label{sec:degen}

- Highlighting degeneracies

-- High-mass degenerate example, 22 vs HM - stacked SNR contours
-- show degenerate waveforms
-- simplified response and closed-form likelihood, full degeneracies
-- (explore simplified response further: parameter maps, other off-the-shelf samplers ?)

%%%%%%%%%%%%%%%%%%%%%%%%%%%%%%%%%%%%

\section{Discussion}
\label{sec:discussion}

- Summary/conclusions


----------------
Other questions:
- special orientations/sky locations ?
- Fisher matrix qualification - explore parameter choices
- continuously increase SNR - include on these examples
- continuously accumulate information over time
- full exploration of parameter space for all of the above...


%%%%%%%%%%%%%%%%%%%%%%%%%%%%%%%%%%%%

\end{document}
